\section{Signaler och anrop}
\subsection{Svenska signaler}

Svenska signaler förekommer inom ett antal prefix. Enligt ITU disponerar Sverige förljande signalserier:
7SA--7SZ samt 8SA--8SZ och vidare de mer kända SAA--SMZ. Dessa har används till varierande ändamål, exempelvis har flyget signaler i serien SE-AAA--ZZZ. Polisen har tidigare använt signaler i serien SHA plus fyra siffror, detta är nu ersatt med nytt system i.o.m. RAKEL. Räddningstjänsten använde SDA med fyra siffror. Signaler som 7SA + 4 siffror används för mindre yrkesbåtar SC + 4 siffror för fritidsbåtar.

Amatörradion använder ett antal signaler, de viktigaste är:

\begin{tabular}{ll}
	SM & Amatörradiosignal utdelad av PTS (nya signaler tilldelas ej i serien) \\
	SA & Amatörradiosignal tilldelad av SSA, ESA eller FRO                     \\
	SK & Klubbsignaler (som regel tvåställiga efter distriktsiffran)           \\
	SL & Militära signaler (som regel tvåställiga efter distriktsiffran)
\end{tabular}

Dessa signaler följs av en \textit{distriktsiffra} se särskilt avsnitt och sedan 2-ställiga eller 3-ställiga bokstavskombinationer som är den personliga signalen. Exempel är SM0UEI som är min egen signal, distriktsiffran är 0 dvs hemmavarande i Stockholms län. Ett annat exempel kan vara SK5JV tidigare Fagersta amatörradioklubb.

Repeatrar som tillhör klubbar får ofta signal efter klubben med tillägg /R för repeater.

Det finns numera även ett stort antal signaler som är tillfälliga eller knutna till särskilda event, exempelvis scoutverksamhet som ibland sänder amatörradio och särskilda forskningsfartyg, flyg- och rymdfart mm.

Som suffix används följande:

\begin{tabular}{ll}
	/M  & Mobil (rörlig) sändaramatör, även portabel \\
	/MM & Mobil till sjöss (mobil maritime)          \\
	/AM & Mobil i luften (aeromobile)                \\
	/PM & Mobil portabel (ej officiellt suffix)\\
	/R  & Repeaterstation
\end{tabular}

\subsection{Utländska signaler}

Eftersom denna handledning är för VHF/UHF tar vi inte med alla DXCC-signaler utan de som är närmast oss i Sverige. Konsultera den fulla DXCC-listan för mer information.

\begin{center}
\begin{longtable}{cl|cl}
	   \textbf{Signal}    & \textbf{Land}        & \textbf{Signal} & \textbf{Land}         \\ \hline
	\endhead
	   AMA--AOZ & Spanien              &    C3A--C3Z     & Andorra               \\
	      C4A--C4Z        & Cypern               &    DAA--DRZ     & Tyskland              \\
	      EAA--EHZ        & Spanien              &    EIA--EJZ     & Irland                \\
	      EKA--EKZ        & Armenien             &    EMA--EOZ     & Ukraina               \\
	      EBA--ERZ        & Moldav.              &    ESA--ESZ     & Estland               \\
	      EUA--EWZ        & Vitryssland          &    EXA--EXZ     & Ryssland              \\
	      EYA--EYZ        & Tajikistan           &    EZA--EZZ     & Turkmenistan          \\
	      FAA--FZZ        & Frankrike            &    GAA--GZZ     & Storbrittannien       \\
	      HAA--HZZ        & Ungern               &    HBA--HBZ     & Schweiz               \\
	      HEA--HEZ        & Schweiz              &    HFA--HFZ     & Polen                 \\
	      HGA--HGZ        & Ungern               &    HVA--HVZ     & Vatikanen             \\
	      HWA--HYZ        & Frankrike            &    H2A--H2Z     & Cypern                \\
	      IAA--IZZ        & Italien              &    JWA--JXZ     & Norge                 \\
	      J4A--J4Z        & Grekland             &    LAA--LNZ     & Norge                 \\
	      LXA--LXZ        & Luxemburg            &    LYA--LYZ     & Litauen               \\
	      LZA--LZZ        & Bulgarien            &    MAA--MZZ     & Storbrittannien       \\
	      OEA--OEZ        & Österrike            &    OFA--OJZ     & Finland               \\
	      OKA--OLZ        & Tjeckiska republiken &    OMA--OMZ     & Slovakiska republiken \\
	      ONA--OTZ        & Belgien              &    OUA--OZZ     & Danmark               \\
	      PAA--PIZ        & Nederländerna        &    P3A--P3Z     & Cypern                \\
	      RAA--RZZ        & Ryska federationen   &    SAA--SMZ     & Sverige               \\
	      SNA--SRZ        & Polen                &    TAA--TCZ     & Turkiet               \\
	      TFA--TFZ        & Island               &    THA--THZ     & Frankrike             \\
	      TKA--TKZ        & Frankrikte           &    TMA--TMZ     & Frankrike             \\
	      TOA--TQZ        & Frankrike            &    TVA--TXZ     & Frankrike             \\
	      T9A--T9Z        & Bosnien-Herzegovina  &    UAA--UIZ     & Ryska federationen    \\
	      UJA--UMZ        & Uzbekistan           &    UNA--UQZ     & Kazakstan             \\
	       URA-UTZ        & Ukraina              &    UUA--UZZ     & Ukraina               \\
	      VPA--VSZ        & Storbritannien       &    XPA--XPZ     & Danmark               \\
	      XXA--XXZ        & Portugal             &    YLA--YLZ     & Litauen               \\
	      Y2A--Y9Z        & Tyskland             &    ZBA--ZJZ     & Storbritannien        \\
	      ZNA--ZOZ        & Storbritannien       &    ZQA--ZQZ     & Storbritannien        \\
	       2AA-2ZZ        & Storbritannien       &    3YA--3YZ     & Norge                 \\
	      3ZA--3ZZ        & Polen                &    5BA--5BZ     & Cypern                \\
	      5PA--5QZ        & Danmark              &    7SA--7SZ     & Sverige               \\
	      8SA--8SZ        & Sverige              &    9AA--9AZ     & Kroatien
\end{longtable}
\end{center}

\subsection{Svenska distrikten}

Sverige delas in i följande distrikt efter sina län:

\begin{center}
\begin{tabular}{cl}
	\textbf{Distrikt} & \textbf{Län}                                     \\ \hline %\endhead
	      0        & Stockholm                                        \\
	      1        & Gotland                                          \\
	      2        & Västerbotten, Norrbotten                         \\
	      3        & Gävleborg, Jämtland, Västernorrland              \\
	      4        & Örebro, Värmland, Dalarna                        \\
	      5        & Östergötland, Södermanland, Västmanland, Uppsala \\
	      6        & Halland, Västra götaland                         \\
	      7        & Skåne, Blekinge, Kronoberg, Jönköping, Kalmar    \\
	      8        & Speciella stationer utanför landets gränser
\end{tabular}
\end{center}

Distrikten förekommen som siffra i utdelade anropssignaler. Radioamatörer byter inte distriktsiffra under resa i annat distrikt, i stället används prefix som t.ex. /M för mobil. Ofta uppger man "SM0UEI mobilt i SM3-land" för att påvisa att man befinner sig utanför ordinarie distrikt.

\section{Terminologi och trafik}

\subsection{Bokstaveringsalfabetet (Svenska)}

\begin{center}
\begin{tabular}{cl|cl|cl }
	A & Adam   & O & Olof    & 1 & Ett        \\
	B & Bertil & P & Petter  & 2 & Tvåa       \\
	C & Cesar  & Q & Qvintus & 3 & Trea       \\
	D & David  & R & Rudolf  & 4 & Fyra       \\
	E & Erik   & S & Sigurd  & 5 & Femma      \\
	F & Filip  & T & Tore    & 6 & Sexa       \\
	G & Gustav & U & Urban   & 7 & Sju        \\
	H & Helge  & V & Viktor  & 8 & Åtta       \\
	I & Ivar   & W & Wilhelm & 9 & Nia        \\
	J & Johan  & X & Xerxes  & 0 & Nolla      \\
	K & Kalle  & Y & Yngve   & . & Punkt      \\
	L & Ludvig & Z & Zäta    & , & Komma      \\
	M & Martin & Å & Åke     & - & Minus      \\
	N & Niklas & Ä & Ärlig   & + & Plus       \\
	  &        & Ö & Östen   &   & Mellanslag \\
\end{tabular}
\end{center}


\subsection{Bokstaveringsalfabetet (Internationella)}
\begin{center}
\begin{tabular}{cl|cl|cl}
	A & Alfa     &  P   & Papa       & 0 & Zero    \\
	B & Bravo    &  Q   & Quebec     & 1 & One     \\
	C & Charlie  &  R   & Romeo      & 2 & Two     \\
	D & Delta    &  S   & Sierra     & 3 & Tree    \\
	E & Echo     &  T   & Tango      & 4 & Fower   \\
	F & Foxtrot  &  U   & Uniform    & 5 & Fife    \\
	G & Golf     &  V   & Victor     & 6 & Six     \\
	H & Hotel    &  W   & Whiskey    & 7 & Seven   \\
	I & India    &  X   & X-ray      & 8 & Ait     \\
	J & Juliet   &  Y   & Yankee     & 9 & Niner   \\
	K & Kilo     &  Z   & Zulu       & . & Stop    \\
	L & Lima     & Å/AA & Alfa-Alfa  & , & Decimal \\
	M & Mike     & Ä/AE & Alfa-Echo  & - & Minus   \\
	N & November & Ö/OE & Oscar-Echo & + & Plus    \\
	O & Oscar    &      &            &   & Space   \\
\end{tabular}
\end{center}

\subsection{Q-koder}
\begin{longtable}{ll}
	\textbf{Kod} & \textbf{Fråga / Svar}                                                         \\ \hline \endhead
	QRA & Vad heter er station?                                                \\
	    & Vår station heter ...                                                \\ \hline
	QRB & Hur långt bort från min station befinner ni er?                      \\
	    & Avståndet mellan oss är ungefär ...                                  \\ \hline
	QRG & Kan ni ange min exakta frekvens?                                     \\
	    & Er exakta frekvens är ... (MHz/kHz)                                  \\ \hline
	QRH & Varierar min frekvens/våglängd?                                      \\
	    & Er frekvens/våglängd varierar.                                       \\ \hline
	QRI & Hur är min sändningston (CW)?                                        \\
	    & Er sändningston är 1--God, 2--Varierande, 3--Dålig                   \\ \hline
	QRK & Vilken uppfattbarhet har mina signaler?                              \\
	    & Uppfattbarheten hos dina signaler är:                                \\
	    & 1--Dålig, 2--Bristfällig, 3--Ganska god, 4--God, 5--Utmärkt          \\ \hline
	QRL & Är ni upptagen?                                                      \\
	    & Jag är upptagen med ... (namn/signal) stör ej.                       \\ \hline
	QRM & Är ni störd av annan station?                                        \\
	    & Störningarna är:                                                     \\
	    & 1--Obef., 2--Svaga, 3--Måttliga, 4--Starka, 5--Mycket starka         \\ \hline
	QRN & Besväras ni av atmosfäriska störningar?                              \\
	    & Störningarna är:                                                     \\
	    & 1--Obef., 2--Svaga, 3--Måttliga, 4--Starka, 5--Mycket starka         \\ \hline
	QRO & Kan jag (ska jag) öka sändareffekten?                                \\
	    & Öka sändareffekten.                                                  \\ \hline
	QRP & Kan jag (ska jag) minska sändareffekten?                             \\
	    & Minska sändareffekten.                                               \\ \hline
	QRQ & Kan jag (får jag) öka sändningshastigheten?                          \\
	    & Öka sändningshastigheten.                                            \\ \hline
	QRS & Kan jag (skall jag) sända långsammare?                               \\
	    & Sänd långsammare.                                                    \\ \hline
	QRT & Skall jag avbryta sändningen?                                        \\
	    & Avbryt sändningen                                                    \\ \hline
	QRU & Har ni något till mig?                                               \\
	    & Jag har inget till er. Se även QTC.                                  \\ \hline
	QRV & Är ni redo?                                                          \\
	    & Jag är redo.                                                         \\ \hline
	QRX & När anropar ni mig härnäst?                                          \\
	    & Jag anropar er kl ... (på ... MHz/kHz)                               \\ \hline
	QRZ & Vem anropar mig?                                                     \\
	    & Ni anropas av ... (på ... MHz/kHz).                                  \\ \hline
	QSA & Vilken styrka har mina signaler?                                     \\
	    & Era signaler är:                                                     \\
	    & 1--Ej uppf., 2--Svaga, 3--Ganska starka, 4--Starka, 5--Mycket starka \\ \hline
	QSB & Svajar styrkan på mina signaler?                                     \\
	    & Styrkan på era signaler svajar.                                      \\ \hline
	QSK & Kan du höra mig mellan dina tecken och får jag avbryta dig?          \\
	    & Jag kan höra dig mellan mina tecken och du får avbryta.              \\ \hline
	QSL & Kan ni ge mig kvittens?                                              \\
	    & Jag kvitterar.                                                       \\ \hline
	QSO & Ha ni förbindelse med ... eller ... (förmedlat)?                     \\
	    & Jag har förbindelse med ... (via ...)                                \\ \hline
	QST & Har tidigare använts som allmänt anrop men ersatts av CQ             \\ \hline
	QSY & Skall jag övergå till att sända på annan frekvens?                   \\
	    & Gå över till att sända på annan frekvens (eller ... kHz/MHz).        \\ \hline
	QTC & Hur många telegram har ni att sända?                                 \\
	    & Jag har ... telegram till dig (eller ...).                           \\ \hline
	QTH & Vilken är er geografiska position?                                   \\
	    & Min geografiska position är ...                                      \\ \hline
	QTR & Kan ni ge mig rätt tid?                                              \\
	    & Rätt tid är ...                                                      \\ \hline
\end{longtable}

\subsection{Lokator}

Lokator (Maidenhead locator) är ett praktiskt sätt att tala om sin ungefärliga position genom att ange endast sex stycken tecken. En lokator kan t.ex. se ut som JO89VK vilket täcker in nordvästa Järfälla. Det finns många verktyg för att räkna på lokator där ute, det är bra att känna sin egen. Det finns appar för detta till telefonerna som både kan räkna på bäring, distans mellan två rutor och dessutom via telefonens GPS bestämma vilken lokator du för närvarande befinner dig i.

Första paret dela in jorden i 18x18 fält, dvs 20 grader per fält longitud och 10 grader per fält latitud. Varje sådant fält delas sedan in i 10x10 rutor som numreras 0-9 på vardera axeln. Dessa i sin tur delas sedan in i 24x24 smårutor som då får storleksordningen 2.5 grader latitud och 5 grader long. vardera.

\section{Teknik}

\subsection{Effekt i dBW och dBm}

Effekter anges i W eller i decibel relaterat till 1 mW (dBm) eller relaterat 1W (dBW). Tabell över effekt och decibelwatt nedan:
\begin{center}
\begin{tabular}{rrr|rrr|rrr}
	   \textbf{Effekt} & \textbf{dBW} & \textbf{dBm} & \textbf{Effekt} & \textbf{dBW} & \textbf{dBm} & \textbf{Effekt} & \textbf{dBW} & \textbf{dBm} \\ \hline
	  1 \textmu W &          -60 &          -30 &        1 W &            0 &           30 &      100 W &           20 &           50 \\
	 10 \textmu W &          -50 &          -20 &        3 W &            5 &           35 &      250 W &           24 &           54 \\
	100 \textmu W &          -40 &          -10 &        5 W &            7 &           37 &      500 W &           27 &           57 \\
	         1 mW &          -30 &            0 &       10 W &           10 &           40 &       1 kW &           30 &           60 \\
	        10 mW &          -20 &           10 &       20 W &           13 &           43 &     1.5 kW &           32 &           62 \\
	       100 mW &          -10 &           20 &       50 W &           17 &           47 &     2.0 kW &           33 &           63
\end{tabular}
\end{center}

\subsection{S-värden}

Signalstyrkan i amatörradio uttrycks oftast som S-värden. Dessa fås i regel genom nivån på AGC hos mottagaren. Därför ser man sälla utslag vid riktigt låga signaler.

Standard kalibrering för S-metern är enligt följande skala:

\begin{center}
\begin{longtable}{r|rr|rr}
	     & \multicolumn{2}{c|}{\textbf{$<$ 30 MHz}} & \multicolumn{2}{c}{\textbf{$>$ 30 MHz}} \\ 
	   \textbf{S} &  \textbf{dBm} &                \textbf{\textmu V} &  \textbf{dBm} &               \textbf{\textmu V} \\ \hline \endhead
	   1 & -121 &                     0.21 & -141 &                    0.02 \\
	   2 & -115 &                     0.40 & -135 &                    0.04 \\
	   3 & -109 &                     0.80 & -129 &                    0.08 \\
	   4 & -103 &                     1.60 & -123 &                    0.16 \\
	   5 &  -97 &                     3.20 & -117 &                    0.32 \\
	   6 &  -91 &                     6.30 & -111 &                    0.63 \\
	   7 &  -85 &                    12.60 & -105 &                    1.26 \\
	   8 &  -79 &                    25.00 &  -99 &                    2.50 \\
	   9 &  -73 &                    50.00 &  -93 &                    5.00 \\
	9+10 &  -63 &                      160 &  -83 &                      16 \\
	9+20 &  -53 &                      500 &  -73 &                      50 \\
	9+30 &  -43 &                     1600 &  -63 &                     160 \\
	9+40 &  -33 &                     5000 &  -53 &                     500
\end{longtable}
\end{center}
\subsection{Termiska brusgolvet}

\begin{center}
\begin{tabular}{rr|rr|rr}
	\textbf{RBW} & \textbf{N$_0$} & \textbf{RBW} & \textbf{N$_0$} & \textbf{RBW} & \textbf{N$_0$} \\ \hline
	         0.5 &           -141 &         6.25 &           -136 &          100 &           -124 \\
	         1.0 &           -144 &        12.50 &           -133 &          200 &           -121 \\
	         3.0 &           -139 &        25.00 &           -130 &         5000 &           -107 \\
	         5.0 &           -137 &        50.00 &           -127 &        10000 &           -104
\end{tabular}
\end{center}

Mottagarbandbredden (RBW) anges i kHz och brusgolvet i dBm (dB relaterat en styrka om 1 mW).

\subsection{Return loss och VSWR}

Return loss och VSWR anger samma sak. VSWR är vanligare inom amatörradio medan man i profesionella sammanhang föredrar att prata om return loss. RL är storleken på den reflekterade signalen i förhållande till den framåtgående signalen. Return loss mäts alltså i dB enligt formeln $10\log(P_F/P_R)$ där $P_F$ är den framåtgående effekten (forward) och  $P_R$ är den reflekterade signalen i retur.

\begin{longtable}{rrr|rrr|rrr}
	\textbf{RL} & \textbf{VSWR} & \textbf{\%} & \textbf{RL} & \textbf{VSWR} & \textbf{\%} & \textbf{RL} & \textbf{VSWR} & \textbf{\%} \\ \hline 	\endhead
	          1 &         17,39 &       79,43 &           8 &          2,32 &       15,85 &          20 &          1,22 &        1,00 \\
	          2 &          8,72 &       63,10 &          10 &          1,92 &       10,00 &          22 &          1,17 &        0,63 \\
	          3 &          5,85 &       50,12 &          12 &          1,67 &        6,31 &          24 &          1,13 &        0,40 \\
	          4 &          4,42 &       39,81 &          14 &          1,50 &        3,98 &          25 &          1,12 &        0,32 \\
	          5 &          3,57 &       31,62 &          15 &          1,43 &        3,16 &          26 &          1,11 &        0,25 \\
	          6 &          3,01 &       25,12 &          16 &          1,38 &        2,51 &          28 &          1,08 &        0,16 \\
	          7 &          2,61 &       19,95 &          18 &          1,29 &        1,58 &          30 &          1,07 &        0,10
\end{longtable}

Acceptabelt RL är ungefär från 14 dB, riktigt bra från 20 dB och de bästa komponenterna ligger runt 30 dB. Många antenntuners som går med automatik startar avstämningen först när VSWR är 1:2 eller sämre som motsvarar ca 10 dB RL.

\subsection{CTCSS subtoner}

Inom amatörradio används ofta pilottoner (subtoner) som CTCSS\footnote{Contnuous Tone-Conded Squelch System} för repeatrar och liknande. På PMR446 används subtoner för att skapa virtuella grupper och sub-kanaler. De som används är följande toner och frekvenser:

\begin{tabular}{rr|rr|rr|rr|rr}
	 1 &  67,0 &  2 &  69,3 &  3 &  74,4 &  4 &  77,0 &  5 &  79,7 \\ \hline
	 6 &  82,5 &  7 &  85,4 &  8 &  88,5 &  9 &  91,5 & 10 &  94,8 \\ \hline
	11 &  97,4 & 12 & 100,0 & 13 & 103,5 & 14 & 107,2 & 15 & 110,9 \\ \hline
	16 & 114,8 & 17 & 118,8 & 18 & 123,0 & 19 & 127,3 & 20 & 131,8 \\ \hline
	21 & 136,5 & 22 & 141,3 & 23 & 146,2 & 24 & 151,4 & 25 & 156,7 \\ \hline
	26 & 162,2 & 27 & 167,9 & 28 & 173,8 & 29 & 179,9 & 30 & 186,2 \\ \hline
	31 & 192,8 & 32 & 203,5 & 33 & 210,7 & 34 & 218,1 & 35 & 225,7 \\ \hline
	36 & 233,6 & 37 & 241,8 & 38 & 250,3 &    &       &    &
\end{tabular}

\subsection{CTCSS-zoner i Sverige}

Rekommendationer för repeatrar i olika distrikt och län att använda CTCSS för att hindra att störningar uppkommer vid conds mm. Det ger också möjligheten för sändaramatörer att öppna just den repeater man önskar om man har flera på samma frekvens omkring sig.

\begin{tabular}{lcccc}
	\textbf{Område} & \textbf{Primär} & \textbf{Sek. 1} & \textbf{Sek. 2} & \textbf{Sek. 3} \\ \hline
%	\endhead
	Distrikt 0      &      77,0       &      123.0      &      67.0       &      100.0      \\
	Distrikt 1      &      218.1      &      233.6      &                 &  \\
	Distrikt 2      &      107.2      &      146.2      &      162.2      &      186.2      \\
	Distrikt 3      &      127.3      &      141.3      &      250.3      &  \\
	Värml. / Örebro &      74.4       &      151.4      &                 &  \\
	Dalarna         &      85.4       &      151.4      &                 &  \\
	Distrikt 5      &      82.5       &      91.5       &      103.5      &      203.5      \\
	Distrikt 6      &      114.8      &      118,8      &      94.8       &      131.8      \\
	Distrikt 7      &      79.7       &      156.7      &      210.7      &
\end{tabular}

\section{Övergripande frekvensplan}

\subsection{Indelning efter frekvens och våglängd}

\begin{tabular}{llrlrl}
\textbf{Förk.} & \textbf{Benämning}   & \textbf{Frekvens}& & \textbf{Våglängd} & \\ \hline
ELF            & Extremt låg frekvens & 3--30 &Hz          & 10--100 &Mm        \\
SLF            & Superlåg frekvens    & 30--300 &Hz        & 1--10 &Mm          \\
ULF            & Ultralåg frekvens    & 300--3000 &Hz      & 100--1000 &km      \\
VLF            & Väldigt låg frekvens & 3--30 &kHz         & 10--100 &km        \\
LF (LV)            & Låg frekvens         & 30--300 &kHz       & 1--10 &km          \\
MF (MV)            & Mellanfrekvens       & 300--3000 &kHz     & 100--1000 &m       \\
HF (KV)            & Högfrekvens          & 3--30 &MHz         & 1--10 &m           \\
VHF (UKV)           & Väldigt hög frekvens & 30--300 &MHz       & 1--10 &m           \\
UHF            & Ultrahög frekvens    & 300--3000 &MHz     & 100--1000 &mm      \\
SHF            & Superhög frekvens    & 3--30 &GHz         & 10--100 &mm        \\
EHF            & Extremt hög frekvens & 30--300 &GHz       & 1--10 &mm          \\
\end{tabular}

Benämningarna HF, MF och LF har också andra betydelser. Exempelvis an\-vänd\-s HF som beteckning av den signal en antenn tar mot eller sänder oavsett frekvensband, MF kan vara mellansignalen oavsett frekvens efter omvandling i en superheterodynmottagare och LF, ibland benämnt AF (audiofrekvens) är det hörbara ljudet, dvs den modulation som används på signalen.

På engelska används i stället benämningarna RF för radio frequency, IF för intermediate frequency and AF för audio frequency vilket rekommenderas då sammanblandningsrisk med ITU-benämningarna på spektrum inte föreligger.

\subsection{Rundradiobenämningar och frekvensband}

\begin{tabular}{llrl}
\textbf{Förk.} & \textbf{Namn} & \textbf{Frekvens Rundradio} &     \\ \hline
LW/LV          & Långvåg       & 148,5--285                  & kHz \\
MW/MV          & Mellanvåg     & 526,5--1606,5               & kHz \\
SW/KV          & Kortvåg       & 4,3--30                     & MHz \\
UKV            & Ultrakortvåg  & 88--108                     & MHz \\
\end{tabular}

\subsubsection{Egenskaper olika frekvensband}

\textbf{Långvåg} --- Markvågsutbredning, relativt höga sändareffekter, tillförlitliga förbindelser men i övre delen av frekvensbandet kortare förbindelser dagtid. På de lägsta frekvenserna erhålls med hög sändareffekt goda förbindelser på stora avstånd globalt och används även för t.ex. malmprospektering, kommunikation med ubåtar i undervattensläge.

\textbf{Kortvåg} --- God rymdvågsutbredning med mycket lång räckvidd redan med låg effekt men samtidigt starkt avhängigt radiokonditionerna. Med ökande frekvens blir jonosfärreflektionen allt flackare vilket resulterar i en alltmer uttalad död zon (skip). Särskilt utmärkande för kortvågen är att den redan med låg effekt ger under gynnsamma konditioner extremt lång räckvidd via rymdvåg, ibland globalt.

\textbf{Mellanvåg} --- Kombinerar egenskaperna hos angränsande delar av lång- och kortvåg, kan ge kraftig interferens mellan rymd- och markvåg som ofta upplevs som kraftig fädning. Särskilt utmärkande för mellanvågen är den i det närmaste avsaknanden av skipzon eftersom mark- och rymdvåg kompletterar varandra, jonosfärens D-skikt är heller inte särskilt uttalat i frekvensområdet dvs förbindelser via rymdvåg på korta avstånd mellan 100--300 km är möjliga även dagtid under perioder med kraftig solaktivitet.

\textbf{Ultrakortvåg} --- Förbindeleser med låg effekt och små antenner, oberoende av jonosfären men då endast i form av frirumsutbredning, dvs fram till horisonten och under påverkan av terränghinder mm. Särskilt utmärkande för UKV är att rymdvåg saknar, markvågsdämpningen till lands är total och kommunikation på högre frekvenser i princip därför bara sker vid fri sikt mellan sändare och mottagare.


