\section{Signaler och anrop}
\subsection{Landsprefix}

Här är inte alla länder med utan de vanligaste som körs från Sverige.

\begin{center}
\begin{longtable}{cl|cl}
	   	\caption{Tabell över landsprefix}\\
	   \textbf{Signal} & \textbf{Land}        & \textbf{Signal} & \textbf{Land}         \\ \hline
	   \endfirsthead
	\endhead
              AMA--AOZ     & Spanien              & C3A--C3Z        & Andorra               \\
	      C4A--C4Z     & Cypern               & DAA--DRZ        & Tyskland              \\
	      EAA--EHZ     & Spanien              & EIA--EJZ        & Irland                \\
	      EKA--EKZ     & Armenien             & EMA--EOZ        & Ukraina               \\
	      EBA--ERZ     & Moldav.              & ESA--ESZ        & Estland               \\
	      EUA--EWZ     & Vitryssland          & EXA--EXZ        & Ryssland              \\
	      EYA--EYZ     & Tajikistan           & EZA--EZZ        & Turkmenistan          \\
	      FAA--FZZ     & Frankrike            & GAA--GZZ        & Storbrittannien       \\
	      HAA--HZZ     & Ungern               & HBA--HBZ        & Schweiz               \\
	      HEA--HEZ     & Schweiz              & HFA--HFZ        & Polen                 \\
	      HGA--HGZ     & Ungern               & HVA--HVZ        & Vatikanen             \\
	      HWA--HYZ     & Frankrike            & H2A--H2Z        & Cypern                \\
	      IAA--IZZ     & Italien              & JWA--JXZ        & Norge                 \\
	      J4A--J4Z     & Grekland             & LAA--LNZ        & Norge                 \\
	      LXA--LXZ     & Luxemburg            & LYA--LYZ        & Litauen               \\
	      LZA--LZZ     & Bulgarien            & MAA--MZZ        & Storbrittannien       \\
	      OEA--OEZ     & Österrike            & OFA--OJZ        & Finland               \\
	      OKA--OLZ     & Tjeckiska republiken & OMA--OMZ        & Slovakiska republiken \\
	      ONA--OTZ     & Belgien              & OUA--OZZ        & Danmark               \\
	      PAA--PIZ     & Nederländerna        & P3A--P3Z        & Cypern                \\
	      RAA--RZZ     & Ryska federationen   & SAA--SMZ        & Sverige               \\
	      SNA--SRZ     & Polen                & TAA--TCZ        & Turkiet               \\
	      TFA--TFZ     & Island               & THA--THZ        & Frankrike             \\
	      TKA--TKZ     & Frankrikte           & TMA--TMZ        & Frankrike             \\
	      TOA--TQZ     & Frankrike            & TVA--TXZ        & Frankrike             \\
	      T9A--T9Z     & Bosnien-Herzegovina  & UAA--UIZ        & Ryska federationen    \\
	      UJA--UMZ     & Uzbekistan           & UNA--UQZ        & Kazakstan             \\
	       URA-UTZ     & Ukraina              & UUA--UZZ        & Ukraina               \\
	      VPA--VSZ     & Storbritannien       & XPA--XPZ        & Danmark               \\
	      XXA--XXZ     & Portugal             & YLA--YLZ        & Litauen               \\
	      Y2A--Y9Z     & Tyskland             & ZBA--ZJZ        & Storbritannien        \\
	      ZNA--ZOZ     & Storbritannien       & ZQA--ZQZ        & Storbritannien        \\
	       2AA-2ZZ     & Storbritannien       & 3YA--3YZ        & Norge                 \\
	      3ZA--3ZZ     & Polen                & 5BA--5BZ        & Cypern                \\
	      5PA--5QZ     & Danmark              & 7SA--7SZ        & Sverige               \\
	      8SA--8SZ     & Sverige              & 9AA--9AZ        & Kroatien
\end{longtable}
\end{center}

\subsection{Svenska signaler}

Svenska signaler förekommer inom ett antal prefix. Enligt ITU disponerar Sverige förljande signalserier: 7SA--7SZ samt 8SA--8SZ och vidare de mer kända SAA--SMZ. Dessa har används till varierande ändamål, exempelvis har flyget signaler i serien SE-AAA--ZZZ. Polisen har tidigare använt signaler i serien SHA plus fyra siffror, detta är nu ersatt med nytt system i.o.m. RAKEL. Räddningstjänsten använde SDA med fyra siffror. Signaler som 7SA + 4 siffror används för mindre yrkesbåtar SC + 4 siffror för fritidsbåtar.

Amatörradion använder ett antal signaler, de viktigaste är:

\begin{tabular}{ll}
	SM & Amatörradiosignal utdelad av PTS (nya signaler tilldelas ej i serien) \\
	SA & Amatörradiosignal tilldelad av SSA, ESA eller FRO                     \\
	SK & Klubbsignaler (som regel tvåställiga efter distriktsiffran)           \\
	SL & Militära signaler (som regel tvåställiga efter distriktsiffran)
\end{tabular}

Dessa signaler följs av en \textit{distriktsiffra} se särskilt avsnitt och sedan 2-ställiga eller 3-ställiga bokstavskombinationer som är den personliga signalen. Exempel är SM0UEI som är min egen signal, distriktsiffran är 0 dvs hemmavarande i Stockholms län. Ett annat exempel kan vara SK5JV tidigare Fagersta amatörradioklubb.

Repeatrar som tillhör klubbar får ofta signal efter klubben med tillägg /R för repeater.

Det finns numera även ett stort antal signaler som är tillfälliga eller knutna till särskilda event, exempelvis scoutverksamhet som ibland sänder amatörradio och särskilda forskningsfartyg, flyg- och rymdfart mm.

Som suffix används följande:

\begin{tabular}{ll}
	/M  & Mobil (rörlig) sändaramatör, även portabel \\
	/MM & Mobil till sjöss (mobil maritime)          \\
	/AM & Mobil i luften (aeromobile)                \\
	/PM & Mobil portabel (ej officiellt suffix)\\
	/R  & Repeaterstation
\end{tabular}

\subsection{Svenska distrikten}

Sverige delas in i följande distrikt efter sina län:

\begin{table}[h]
	\centering
\begin{tabular}{cl}
	\textbf{Distrikt} & \textbf{Län}                                     \\ \hline %\endhead
	      0        & Stockholm                                        \\
	      1        & Gotland                                          \\
	      2        & Västerbotten, Norrbotten                         \\
	      3        & Gävleborg, Jämtland, Västernorrland              \\
	      4        & Örebro, Värmland, Dalarna                        \\
	      5        & Östergötland, Södermanland, Västmanland, Uppsala \\
	      6        & Halland, Västra götaland                         \\
	      7        & Skåne, Blekinge, Kronoberg, Jönköping, Kalmar    \\
	      8        & Speciella stationer utanför landets gränser
\end{tabular}
\caption{Distriktssiffor i Sverige}
\end{table}
Distrikten förekommen som siffra i utdelade anropssignaler. Radioamatörer byter inte distriktsiffra under resa i annat distrikt, i stället används suffix (tillägg efter ordinarie signal) som t.ex. /M för mobil. Ofta uppger man "SM0UEI mobilt i SM3-land" för att påvisa att man befinner sig utanför ordinarie distrikt.

I de tidigare SMnXYZ-signalerna förekom amatörens bokstavskombination endast en gång, det kunde alltså inte finnas en SM4UEI och en SM0UEI, det vore i så fall samma amatör som flyttat. Med det nya systemet där SSA m.fl. utdelar signaler i stället för PTS i serien SAnXYZ kan SM4XYZ och SA7XYZ alltså vara två helt olika amatörer.
\section{Terminologi och trafik}

\subsection{Bokstaveringsalfabetet (Svenska)}

\begin{table}[H]
	\centering
\begin{longtable}{cl|cl|cl }
	A & Adam   & O & Olof    & 1 & Ett        \\
	B & Bertil & P & Petter  & 2 & Tvåa       \\
	C & Cesar  & Q & Qvintus & 3 & Trea       \\
	D & David  & R & Rudolf  & 4 & Fyra       \\
	E & Erik   & S & Sigurd  & 5 & Femma      \\
	F & Filip  & T & Tore    & 6 & Sexa       \\
	G & Gustav & U & Urban   & 7 & Sju        \\
	H & Helge  & V & Viktor  & 8 & Åtta       \\
	I & Ivar   & W & Wilhelm & 9 & Nia        \\
	J & Johan  & X & Xerxes  & 0 & Nolla      \\
	K & Kalle  & Y & Yngve   & . & Punkt      \\
	L & Ludvig & Z & Zäta    & , & Komma      \\
	M & Martin & Å & Åke     & - & Minus      \\
	N & Niklas & Ä & Ärlig   & + & Plus       \\
	  &        & Ö & Östen   &   & Mellanslag \\
\end{longtable}
\caption{Svenska bokstaveringsalfabetet}
\end{table}

\subsection{Bokstaveringsalfabetet (Internationella)}
\begin{table}[H]
\centering
\begin{tabular}{cl|cl|cl}
	A & Alfa     &  P   & Papa       & 0 & Zero    \\
	B & Bravo    &  Q   & Quebec     & 1 & One     \\
	C & Charlie  &  R   & Romeo      & 2 & Two     \\
	D & Delta    &  S   & Sierra     & 3 & Tree    \\
	E & Echo     &  T   & Tango      & 4 & Fower   \\
	F & Foxtrot  &  U   & Uniform    & 5 & Fife    \\
	G & Golf     &  V   & Victor     & 6 & Six     \\
	H & Hotel    &  W   & Whiskey    & 7 & Seven   \\
	I & India    &  X   & X-ray      & 8 & Ait     \\
	J & Juliet   &  Y   & Yankee     & 9 & Niner   \\
	K & Kilo     &  Z   & Zulu       & . & Stop    \\
	L & Lima     & Å/AA & Alfa-Alfa  & , & Decimal \\
	M & Mike     & Ä/AE & Alfa-Echo  & - & Minus   \\
	N & November & Ö/OE & Oscar-Echo & + & Plus    \\
	O & Oscar    &      &            &   & Space   \\
\end{tabular}
\caption{Internationella bokstaveringsalfabetet (ITU-alfabetet)}
\end{table}

\subsection{Q-koder}
\begin{longtable}{ll}
	\textbf{Kod} & \textbf{Fråga / Svar}                                                         \\ \hline \endhead
	QRA & Vad heter er station?                                                \\
	    & Vår station heter ...                                                \\ \hline
	QRB & Hur långt bort från min station befinner ni er?                      \\
	    & Avståndet mellan oss är ungefär ...                                  \\ \hline
	QRG & Kan ni ange min exakta frekvens?                                     \\
	    & Er exakta frekvens är ... (MHz/kHz)                                  \\ \hline
	QRH & Varierar min frekvens/våglängd?                                      \\
	    & Er frekvens/våglängd varierar.                                       \\ \hline
	QRI & Hur är min sändningston (CW)?                                        \\
	    & Er sändningston är 1--God, 2--Varierande, 3--Dålig                   \\ \hline
	QRK & Vilken uppfattbarhet har mina signaler?                              \\
	    & Uppfattbarheten hos dina signaler är:                                \\
	    & 1--Dålig, 2--Bristfällig, 3--Ganska god, 4--God, 5--Utmärkt          \\ \hline
	QRL & Är ni upptagen?                                                      \\
	    & Jag är upptagen med ... (namn/signal) stör ej.                       \\ \hline
	QRM & Är ni störd av annan station?                                        \\
	    & Störningarna är:                                                     \\
	    & 1--Obef., 2--Svaga, 3--Måttliga, 4--Starka, 5--Mycket starka         \\ \hline
	QRN & Besväras ni av atmosfäriska störningar?                              \\
	    & Störningarna är:                                                     \\
	    & 1--Obef., 2--Svaga, 3--Måttliga, 4--Starka, 5--Mycket starka         \\ \hline
	QRO & Kan jag (ska jag) öka sändareffekten?                                \\
	    & Öka sändareffekten.                                                  \\ \hline
	QRP & Kan jag (ska jag) minska sändareffekten?                             \\
	    & Minska sändareffekten.                                               \\ \hline
	QRQ & Kan jag (får jag) öka sändningshastigheten?                          \\
	    & Öka sändningshastigheten.                                            \\ \hline
	QRS & Kan jag (skall jag) sända långsammare?                               \\
	    & Sänd långsammare.                                                    \\ \hline
	QRT & Skall jag avbryta sändningen?                                        \\
	    & Avbryt sändningen                                                    \\ \hline
	QRU & Har ni något till mig?                                               \\
	    & Jag har inget till er. Se även QTC.                                  \\ \hline
	QRV & Är ni redo?                                                          \\
	    & Jag är redo.                                                         \\ \hline
	QRX & När anropar ni mig härnäst?                                          \\
	    & Jag anropar er kl ... (på ... MHz/kHz)                               \\ \hline
	QRZ & Vem anropar mig?                                                     \\
	    & Ni anropas av ... (på ... MHz/kHz).                                  \\ \hline
	QSA & Vilken styrka har mina signaler?                                     \\
	    & Era signaler är:                                                     \\
	    & 1--Ej uppf., 2--Svaga, 3--Ganska starka, 4--Starka, 5--Mycket starka \\ \hline
	QSB & Svajar styrkan på mina signaler?                                     \\
	    & Styrkan på era signaler svajar.                                      \\ \hline
	QSK & Kan du höra mig mellan dina tecken och får jag avbryta dig?          \\
	    & Jag kan höra dig mellan mina tecken och du får avbryta.              \\ \hline
	QSL & Kan ni ge mig kvittens?                                              \\
	    & Jag kvitterar.                                                       \\ \hline
	QSO & Ha ni förbindelse med ... eller ... (förmedlat)?                     \\
	    & Jag har förbindelse med ... (via ...)                                \\ \hline
	QST & Har tidigare använts som allmänt anrop men ersatts av CQ             \\ \hline
	QSY & Skall jag övergå till att sända på annan frekvens?                   \\
	    & Gå över till att sända på annan frekvens (eller ... kHz/MHz).        \\ \hline
	QTC & Hur många telegram har ni att sända?                                 \\
	    & Jag har ... telegram till dig (eller ...).                           \\ \hline
	QTH & Vilken är er geografiska position?                                   \\
	    & Min geografiska position är ...                                      \\ \hline
	QTR & Kan ni ge mig rätt tid?                                              \\
	    & Rätt tid är ...                                                      \\ \hline
\end{longtable}

\subsection{Lokator}

Lokator (Maidenhead locator) är ett praktiskt sätt att tala om sin ungefärliga position genom att ange endast sex stycken tecken. En lokator kan t.ex. se ut som JO89VK vilket täcker in nordvästa Järfälla. Det finns många verktyg för att räkna på lokator där ute, det är bra att känna sin egen. Det finns appar för detta till telefonerna som både kan räkna på bäring, distans mellan två rutor och dessutom via telefonens GPS bestämma vilken lokator du för närvarande befinner dig i.

Första paret dela in jorden i 18x18 fält, dvs 20 grader per fält longitud och 10 grader per fält latitud. Varje sådant fält delas sedan in i 10x10 rutor som numreras 0-9 på vardera axeln. Dessa i sin tur delas sedan in i 24x24 smårutor som då får storleksordningen 2.5 grader latitud och 5 grader long. vardera.

\subsection{Uppträdande}

När vi kör amatörradio finns det ett antal saker att tänka på som har att göra med hur vi beter oss mot varandra på banden. Se detta som en guide till hur man bör uppträda på banden.

En radioamatör måste vara \textbf{tolerant}. Vi delar frekvenser med många andra personer, en del av dem kommer inte ha samma uppfattning som du själv har om saker och ting. Här gäller det att vara tolerant, förstående och framför allt inte bli upprörd över personer som kanske inte beter sig som du önskade att de betedde sig.

Radioamatörer är \emph{aldrig ensamma på banden} helt oavsett om någon svarar på ditt allmänna anrop eller ej så finns det i det närmaste \textbf{garanterat någon som lyssnar}. 

Tänk på vad du säger och att du undviker diskutera ämnen som kan verka \textbf{upprörande} eller \textbf{stötande}. Ämnen som bör undvikas är \textbf{religion} och livs\-å\-skå\-d\-ni\-ng, \textbf{politisk} ideologi, \textbf{ekonomiska} eller \textbf{sociala} frågor m.m. där motparter kan ha starka åsikter som inte nödvändigtvis stämmer med dina egna. Radion är inte ett agitationsrum för sådana frågor.

\textbf{Svordomar}, \textbf{könsord} och liknande undviker vi helt. Språket skall vara vårdat men behöver inte vara strikt. Tänk på att din motpart är inte den enda som lyssnar utan det finns \textit{andra amatörer som lyssnar}, icke-amatörer som lyssnar, myndigheter som lyssnar och så vidare.

Ha \textbf{förståelse} för att andra kanske inte har dina egna detaljkunskaper, professionalism med mera. Agera \textbf{ödmjukt} gentemot andra människor på banden.

Blir du ändå upprörd, undvik att \emph{agera på det} över huvud taget. Sänd inte över annans sändning, s.k. ''gummitumme'', eller stör på annat vis för du är upprörd. Avsluta hellre QSO:t, byt frekvens eller återkom lite senare när du lugnat ned dig. Tänk på att \textit{de flesta konflikter orsakas av okunskap eller brist på förståelse}. \textbf{Agera vuxet} i sådana situationer och jobba för att \textbf{de-eskalera} situationen.

En skicklig amatör \textit{lyssnar mycket innan sändning}. Vi anropar på ett korrekt sätt och avslutar på ett korrekt sätt. Vi försöker uppge våra respektive signaler på ett \emph{tydligt och läsligt sätt}, i dag finns det en tendens att sluddra över signalerna framför allt på 2m och 70cm banden, gör inte det. Tydlighet är en vinning i sig. 

När någon ny i ringen inträder, räkna upp de deltagande signalerna så att personen tydligt får en bild av alla som är med och vem som är på turen före och efter hen.

Vi pratar inte \textbf{nedvärderande} om personer varesig de är andra amatörer eller ej, eller en viss grupp av personer. Vi undviker \textbf{sexuella anspelningar} och vitsar ''\textbf{under bältet}'' liksom allt för \textbf{personliga detaljer}. Amatörradion är främst för \textbf{tekniska diskussioner} av rent \textbf{privat natur} eller av \textbf{allmänt intresse för hobbyn}, tester och prov med mera.

Undvik väldigt \textbf{långa sändningspass}. Ibland händer det saker hos dina motstationer som att de får ett viktigt telefonsamtal eller måste springa ut i köket för katten har rivit ner något, ett barn ramlar eller annat som gör att man måste kvickt lämna radion. Att \textbf{långprata} i sådana lägen gör det svårt att tala om ''QRX --- jag måste ta hand om en sak, anropar dig igen om 5 min.''. Enstaka gånger kanske man behöver förklara något lite längre men gör det till en vana att lämna luckor så ofta som möjligt.

\textbf{Nödtrafik har alltid prioritet} och måste respekteras på alla
frekvenser.

\subsection{Repeatrar}

Repeatrars syfte är främst att förlänga kommunikationen från mobila och portabla amatörsändare. Samtal mellan fasta stationer förekommer men om ni hör varandra på direkten, övergå gärna till en simplex-frekvens i stället för att belägga repeatern.

Lämna luckor mellan er när ni växlar station som sänder. Gör det möjligt för andra att ''breaka-in'' särskilt om ert QSO fortsätter under längre tid. Ta hänsyn till att andra kanske vill använda repeatern för att nå personer som de inte kan nå annars. Hänsyn åt båda hållen förutsätts här. 

Repeatern är en begränsad resurs. Det är inte okay att lägga beslag på den under långa perioder när andra kanske behöver den, var ödmjuk inför att någon driver repeatern och har satt upp den i första hand för att supporta mobila stationer.

Nödtrafik har alltid prioritet.

\subsection{QSO}

Konsten att genomföra ett radiosamtal (QSO) i olika sammanhang.

\subsubsection{Radiosamtalets delar}

Ett radiosamtal består som regel av tre delar. Först sker ett anrop, när kontakt etablerats utväxlas ett antal meddelande (dialog) och när man är klarar avslutas samtalet. Dessa tre delar är ganska standard. Man följer detta ganska strikt t.ex. på kortvågen där telefoni oftast innebär SSB. Anledningen är enkel, det går inte höra när någon släpper sändtangenten eller bara är tyst och tänker.

När man kör FM över repeatrar på VHF/UHF är det inte lika vanligt att man både öppnar och avslutar varje sändning med motparten och sin egen signal. Men man skall regelbundet upprepa signalerna och i praktiken är det lämpligt att göra kanske var femte minut eller oftare.

\subsubsection{Anropet}

Ett anrop kan se ut ungefär såhär:

<<<<<<< HEAD
--- SM0MAD från SM0UEI, SA0MAD kom!
=======
--- SA0MAD från SM0UEI, SA0MAD kom!
>>>>>>> 9b0d699b43f86e40d49f75e6dd4d94047b3fe129

Här är det SA0MAD som anropas av SM0UEI. 

<<<<<<< HEAD
=======
Svaret kan se ut ungefär såhär:

>>>>>>> 9b0d699b43f86e40d49f75e6dd4d94047b3fe129
--- SM0UEI från SA0MAD kom!

Därefter övergår radiosamtalet i dialog eller meddelandesändning.

\subsubsection{Allmänt anrop} 

Används när man inte ropar på någon särskild motstation utan önskar samtal med vem som helst. På svenska använder man ofta just orden ''allmänt anrop'' medan på engelska är det vanligare att man uttalar CQ (seek you). Ett allmänt androp kan se ut såhär:

--- Allmänt anrop, allmänt anrop, allmänt androp från SM0UEI SM0UEI SM0UEI kallar allmänt anrop och lyssnar.

Eller på engelska:

--- CQ CQ CQ this is SM0UEI calling CQ CQ CQ and standing by.

\subsubsection{Meddelandesändning}

--- SA0MAD från SM0UEI, tack för svaret. Din signal är 59 hos mig, mitt QTH är JO89WA och namnet är Anders. SA0MAD från SM0UEI kom.

--- SM0UEI från SA0MAD, tack för rapporten. Din signal är 57 hos mig, jag befinner mi i JO89VK men kommer under kvällen byta QTH. Jag kommer då vara QRV på 3663 kHz. QSL? SM0UEI från SA0MAD.

--- SA0MAD från SM0UEI, QSL på det, QRX 19.30 på frekvens 3663 kHz. 

\subsubsection{Avslutning}

--- SA0MAD från SM0UEI, tack för rapport och vi hörs senare, 73, slut kom

--- SM0UEI från SA0MAD, 73 tillbaka, klart slut.

\subsection{Contest}

Under contest är det vanligtvis så att man kör ett relativt kort QSO för att hinna så många som möjligt. Under constestförhållanden utväxlar man som regel signal, sekvensnummer, signalrapport med varandra. Ofta lägger man till "contest" i sitt anrop exempelvis:

--- CQ Contest CQ Contest CQ Contest SM0UEI SM0UEI CQ Contest

Efter man etablerat kontakt utväxlar man signalrapporter och sekvensnummer

--- SM0UEI from SA0MAD, you are 59 here and sequence 28, SM0UEI from SA0MAD.

--- SA0MAD from SM0UEI, QSL your signals are 58 and my sequence is 112, QSL? SA0MAD from SM0UEI

--- QSL and good luck, SM0UEI from SA0MAD

\subsubsection{Pile-up}

Ibland kan det bli väldigt många motstationer samtidigt som ropar. Nu gäller det att spetsa öronen! Först gäller det att sålla. Rara signaler från långtbortistan ger mer poäng i en contest som regel eller från länder du inte kört osv beroende på regler. Försök att sålla med "du som sänder från Florida" eller "VK7 kom igen" osv till det är en station kvar. Kör den snabbt, ropa CQ igen och börja sålla.

Direkt när det uppstår en pile-up är det effektivt att köra split. Dvs du lyssnar 5-10 kHz upp eller ned från den frekvens du sänder på. Det gör det lättare för dig att behålla kommandot under pile-up. Ligger du och sänder i ett frekvensområde som är särskilt ägnat för DX är det smart att lägga Rx-frekvensen strax utanför. Det undviker att man stökar ned i DX-bandet.

Kör du split skall du säga det efter varje sändning. "CQ CQ CQ de Sierra Mike Zeor Uniform Echo India listening 5 up" exempelvis. På CW bör en split vara minst 2 kHz och på SSB bör den vara minst 5 kHz ännu hellre 10 kHz. Tänk på att när du startar din split måste du kolla så att båda frekvenserna är ok. Låt inte din pile-up sprida ut sig för mycket även om det är kanske enklare för dig så är risken stor att den stör någon annan. 

Kör korta QSO. Utbyt snabbt den information som behövs och ta sedan nästa. Ha förståelse för att det kan bli krockar i en pile-up. När du hör en partiell signal eller station du vill prata med håll fast vid den. Om du har svårt att läsa den be den repetera tills ni är klara. Genom att du är auktoriteten på frekvensen kommer pile-up:en att lugna ned sig och vänta på sin tur. Om du ''hattar omkring'' är risken att all radiodisciplin far ut genom fönstret.

Använd ett standardmönster när du kör:

--- SM0UEI CQ CQ CQ de SM0UEI 10 UP

--- SM0UEI de ON3XYZ you are 59 sequence 122, QSL?

--- ON3XYZ SM0UEI QSL, 59 back to you, sequence 312 QSL?

--- QSL. CQ CQ CQ de SM0UEI 10 UP ...

Om du försöker nå en motstation med pile-up var uppmärksam på dennes sändningar och vänta på din tur. Tala gärna om signal och var du sänder från men släpp sedan fram andra. Tänk på hur du själv skulle vilja att en pile-up på din egen station skulle vilja agera. Den gyllene regeln är också alltid lyssna först --- sänd sedan!


\section{Teknik}

\subsection{Effekt i dBW och dBm}

Effekter anges i W eller i decibel relaterat till 1 mW (dBm) eller relaterat 1W (dBW). Tabell över effekt och decibelwatt nedan:
\begin{table}[h]
\begin{tabular}{rrr|rrr|rrr}
	\textbf{Effekt} & \textbf{dBW} & \textbf{dBm} & \textbf{Effekt} & \textbf{dBW} & \textbf{dBm} & \textbf{Effekt} & \textbf{dBW} & \textbf{dBm} \\ \hline
	    1 \textmu W &          -60 &          -30 &             1 W &            0 &           30 &           100 W &           20 &           50 \\
	   10 \textmu W &          -50 &          -20 &             3 W &            5 &           35 &           250 W &           24 &           54 \\
	  100 \textmu W &          -40 &          -10 &             5 W &            7 &           37 &           500 W &           27 &           57 \\
	           1 mW &          -30 &            0 &            10 W &           10 &           40 &            1 kW &           30 &           60 \\
	          10 mW &          -20 &           10 &            20 W &           13 &           43 &          1.5 kW &           32 &           62 \\
	         100 mW &          -10 &           20 &            50 W &           17 &           47 &          2.0 kW &           33 &           63
\end{tabular}
\caption{Tabell över effekt och decibelskalor}
\end{table}

\subsection{S-värden, signalvärde, S-meter}

Signalstyrkan i amatörradio uttrycks oftast som S-värden. Dessa fås i regel genom nivån på AGC hos mottagaren. Därför ser man sälla utslag vid riktigt låga signaler.

Standard kalibrering för S-metern är enligt skalan i tabellen \ref{tab:s-varden}

\begin{table}[h]
\begin{tabular}{r|rr|rr||r|rr|rr}
      & \multicolumn{2}{c|}{\textbf{$<$ 30 MHz}} &
  \multicolumn{2}{c}{\textbf{$>$ 30 MHz}}       && \multicolumn{2}{c|}{\textbf{$<$ 30 MHz}} &
  \multicolumn{2}{c}{\textbf{$>$ 30 MHz}}\\ \textbf{S} & \textbf{dBm}
  & \textbf{\textmu V} & \textbf{dBm} & \textbf{\textmu V}&   \textbf{S} & \textbf{dBm}
  & \textbf{\textmu V} & \textbf{dBm} & \textbf{\textmu V} \\\hline
          
	   1 & -121 & 0.21  & -141 & 0.02 & 9+10 & -63 & 160  & -83 & 16  \\
	   2 & -115 & 0.40  & -135 & 0.04 & 9+20 & -53 & 500  & -73 & 50  \\
	   3 & -109 & 0.80  & -129 & 0.08 & 9+30 & -43 & 1600 & -63 & 160 \\
	   4 & -103 & 1.60  & -123 & 0.16 & 9+40 & -33 & 5000 & -53 & 500 \\
	   5 & -97  & 3.20  & -117 & 0.32 &      &     &      &     &     \\
	   6 & -91  & 6.30  & -111 & 0.63 &      &     &      &     &     \\
	   7 & -85  & 12.60 & -105 & 1.26 &      &     &      &     &     \\
	   8 & -79  & 25.00 & -99  & 2.50 &      &     &      &     &     \\
	   9 & -73  & 50.00 & -93  & 5.00 &      &     &      &     &     \\
\end{tabular}
\caption{Tabell över S-värden, effekt och spänning}
\label{tab:s-varden}
\end{table}

\subsection{Modulationer}

\subsubsection{Bandbredd olika modulationer}

Olika modulationer upptar olika bandbredd. Detta är mycket viktigt att förstå när man ställer in sin radiostation. Detta gäller särskilt att beakta i närheten av nödfrekvenser eller bandkanten. När vi talar om bandbredder här förstås den bandbredd vari minst 98\% av signalens effekt befinner sig.

\begin{tabular}{lrl}
	\textbf{Modulation} & \textbf{Bandbredd} & \textbf{Kommentarer}                  \\ \hline
	CW                  &          $<$500 Hz & Smalbandigt                           \\
	AM                  &              6 kHz & Amplitudmodulering med fullt sidband  \\
	SSB*                &              <3 kHz & Amplitudmodulering med enkelt sidband \\
	NFM                 &           7-12 kHz & Smalbandig FM                         \\
	FM                  &             16 kHz & Normal FM                             \\
	WFM                 &          16-75 kHz & Bredbandig FM (t.ex. rundradio)
\end{tabular}

*) För SSB gäller att USB och LSB fungerar lite olika. När man beräknar den högsta eller lägsta frekvensen utgår man från den inställda frekvensen $f$. För USB gäller då att högsta frekvensen är $f+3$\,kHz. För LSB blir det $f-3$\,kHz. Detta innebär att om du sänder på 80\,m-bandet och du får sända telefoni från 3600--3800\,kHz och vill lägga dig i under bandkanten och köra LSB skall du ställa in din radio på 3603\,kHz som lägsta frekvens. Använd gärna lite marginal och kör exempelvis 3605\,kHz i stället.

Den egentliga modulationsfrekvensen är dock lite mer komplicerad. Normalt anges den verkliga modulationsfrekvensen som ca 2,7\,kHz och det beror på att man i regel filtrerar bort ljudet under 300\,Hz och det över 3000\,Hz. Detta innebär att det akustiska frekvensomfånget blir 300--3000\,Hz och därmed upptar signalen inte mer än 2,7\,kHz.

Det är vanligt att man märker stationer som kör överdriven bandbredd. Antingen som en följd av att man vill öka sin modulationsvinst, okunskap eller man har skruvat i sin radio. Syftet kan var att få bättre genomslag vid långväga förbindelser.

\subsubsection{Telegrafi, CW}

CW står för continuous waves och innebär en rent omodulerad bärvåg. I mottagaren används en oscillator för att återskapa hörbar signal. Detta används för telegrafi och modulationsslaget är oftast A1A. Ibland sänds telegrafi som modulerad AM-bärvåg också som då moduleras med t.ex. 700\,Hz ton. Det är dock mindre vanligt.

Bandbredden för CW är i teorin mycket smal. I praktiken blir den lite beroende på frekvens från några Hz till något hundratal Hz beroende på frekvensband och sändarens beskaffenhet i form av jitter och frekvensstabilitet.

Bandbredden hos CW består av fasbruset vilket normalt är så undertryckt att det egentligen inte betyder så mycker samt stig- respektive falltiden när man nycklar eller släpper nyckeln. Sker detta mjukt är bandbredden låg, har man skarp in- eller urkoppling av bärvågen nyttjar man mer bandbredd.

\subsubsection{Amplitudmodulering, AM}

Amplitudmodulering finns i flera olika varianter. Vanlig AM består av en bärvåg vars styra varieras i takt med signalen som skall sändas. Denna förändring av bärvågen producerar sidband och det är i dessa som den egentliga informationen återfinns. Bärvågen i sig får dock lejonparten av signalen varför det är en sändningsklass som nästan aldrig används inom amatörradiobanden.

Bandbredden hos AM-modulerad signal kan beräknas genom att man tar två gånger högsta modulationsfrekvensen. Detta ger t.ex. vid en modulationsfrekvens som går från 300-3000\,Hz en bandbredd som varierar med talet från upp till 6\,kHz.

$$B=2f_m$$

Där $f_m$ är högsta modulationsfrekvensen.

\subsubsection{SSB/ESB -- Enkelt sidband, en AM-variant}

Enkelt sidband används av radioamatörer för att minska på bandbredden samt lägga radioenergin där den behövs mest. Eftersom båda sidbanden innehåller samma information kan man filtrera bort dessa samt bärvågen innan man matar sändarens förstärkarsteg med resultatet. I mottagaren behöver man dock återskapa en referenssignal, en så kallad beat-oscillator gör detta. När man ställer in frekvensen så försöker man därmed matcha den ursprungliga frekvensen. Ligger man för långt från låter det kalle anka, kommer man för nära sidbandet låter det dovt och basigt. 

Enkelt sidband förkortas ESB eller SSB (single side-band) och man kan välja vilket sidband man vill använda sig av. På amatörradiofrekvenser under 10 MHz använder man LSB (lägre/lower sidbandet) och på frekvenser över 10 MHz används USB/ÖSB (upper/övre sidbandet). 

Detta är mycket av tradition. Använder man fel sorts sidband hörs det inget vettigt när man försöker lyssna. Språkrytmerna gör dock att vi uppfattar det som att mänskligt tal förekommer. I dag händer det att amatörer bryter mot regeln och sänder med ``fel'' sidband på fel frekvens.

Bandbredden hos SSB är halva den för normal AM egentligen. Den kan därmed beräknas som för AM och halveras.

$$B=f_m$$

Där $f_m$ är högsta modulationsfrekvensen.

\subsubsection{Frekvensmodulering, FM}

Frekvensmodulering består av att man tar en bärvåg och modulerar den med talet genom att skifta dess frekvens. Om skiftet i frekvens är mycket litet kallas moduleringen för fasmodulation. FM-modulering indelas i lite olika klasser beroende på hur stor deviation som används. På amatörradions VHF- och UHF-band talar vi om FM och NFM (Narrow FM, andra namn förekommer också). Ibland talar man om bred FM, normal FM och smal FM på svenska.

Normal FM innebär att deviationen (hur mycket signalen avviker från grundfrekvensen) är lika stor som den högsta modulationsfrekvensen. Det är vanligt att kommunikationsradio använder sig av 3 kHz som högsta modulationsfrekvens och 5 kHz deviation. Deviationen är då något bredare och ger upphov till en viss modulationsvinst. När man talar om FM-radio på UKV-bandet för rundradio så är deviationen ca 75\,kHz och högsta modulationsfrekvens ca 16\,kHz. Där är alltså svinget betydligt bredare än modulationen och detta är bred FM.

Nu för tiden förordas en minskning av bandbredden för FM-sändningar på amatörbanden, främst är det väl VHF och UHF där FM-sändning är vanligast förekommande och där vill man ha en kanalindelning om 12,5\,kHz i stället för som tidigare 25\,kHz. Om man studerar bandbredden hos olika FM-signaler kan man använda sig av Carsons bandbreddsbegrepp:

$$B=2(f_M+f_D)$$

Där $B$ är bandbredden $f_M$ högsta modulationsfrekvensen och $f_d$ är FM-signalens maximala deviation (även kallat sving). Carsons bandbreddsbegrepp säger att 98\% av energin förekommer inom den stipulerade bandbredden. Det betyder att att grannkanalen kan få ungefär 17\,dB lägre signal under sändning vilket fortfarande inte är enormt bra. Carson var för övrigt den som faktiskt uppfan SSB-modulationen.

\begin{center}
\begin{tabular}{rrrr}
Deviation & Modulation & Bandbredd & Kanaldelning\\ \hline
5 kHz & 3 kHz & 16 kHz & 25 kHz\\
2.5 kHz & 3 kHz & 11 kHz & 12.5 kHz\\
\end{tabular}
\end{center}

\subsection{Termiska brusgolvet}

\begin{center}
\begin{tabular}{rr|rr|rr}
	\textbf{RBW} & \textbf{N$_0$} & \textbf{RBW} & \textbf{N$_0$} & \textbf{RBW} & \textbf{N$_0$} \\ \hline
	         0.5 &           -141 &         6.25 &           -136 &          100 &           -124 \\
	         1.0 &           -144 &        12.50 &           -133 &          200 &           -121 \\
	         3.0 &           -139 &        25.00 &           -130 &         5000 &           -107 \\
	         5.0 &           -137 &        50.00 &           -127 &        10000 &           -104
\end{tabular}
\end{center}

Mottagarbandbredden (RBW) anges i kHz och brusgolvet i dBm (dB relaterat en styrka om 1 mW).

\subsection{Return loss och VSWR}

Return loss och VSWR anger samma sak. VSWR är vanligare inom amatörradio medan man i profesionella sammanhang föredrar att prata om return loss. RL är storleken på den reflekterade signalen i förhållande till den framåtgående signalen. Return loss mäts alltså i dB enligt formeln $10\log(P_F/P_R)$ där $P_F$ är den framåtgående effekten (forward) och  $P_R$ är den reflekterade signalen i retur.

\begin{longtable}{rrr|rrr|rrr}
	\textbf{RL} & \textbf{VSWR} & \textbf{\%} & \textbf{RL} & \textbf{VSWR} & \textbf{\%} & \textbf{RL} & \textbf{VSWR} & \textbf{\%} \\ \hline 	\endhead
	          1 &         17,39 &       79,43 &           8 &          2,32 &       15,85 &          20 &          1,22 &        1,00 \\
	          2 &          8,72 &       63,10 &          10 &          1,92 &       10,00 &          22 &          1,17 &        0,63 \\
	          3 &          5,85 &       50,12 &          12 &          1,67 &        6,31 &          24 &          1,13 &        0,40 \\
	          4 &          4,42 &       39,81 &          14 &          1,50 &        3,98 &          25 &          1,12 &        0,32 \\
	          5 &          3,57 &       31,62 &          15 &          1,43 &        3,16 &          26 &          1,11 &        0,25 \\
	          6 &          3,01 &       25,12 &          16 &          1,38 &        2,51 &          28 &          1,08 &        0,16 \\
	          7 &          2,61 &       19,95 &          18 &          1,29 &        1,58 &          30 &          1,07 &        0,10
\end{longtable}

Acceptabelt RL är ungefär från 12\,dB, riktigt bra från 20 dB och de bästa komponenterna ligger runt 30\,dB. Många antenntuners som går med automatik startar avstämningen först när VSWR är 1:2 eller sämre som motsvarar ca 10\,dB\,RL.

\subsection{CTCSS subtoner}

Inom amatörradio används ofta pilottoner (subtoner) som CTCSS\footnote{Contnuous Tone-Conded Squelch System} för repeatrar och liknande. På PMR446 används subtoner för att skapa virtuella grupper och sub-kanaler. De som används är följande toner och frekvenser:

\begin{tabular}{rr|rr|rr|rr|rr}
	 1 &  67,0 &  2 &  69,3 &  3 &  74,4 &  4 &  77,0 &  5 &  79,7 \\ \hline
	 6 &  82,5 &  7 &  85,4 &  8 &  88,5 &  9 &  91,5 & 10 &  94,8 \\ \hline
	11 &  97,4 & 12 & 100,0 & 13 & 103,5 & 14 & 107,2 & 15 & 110,9 \\ \hline
	16 & 114,8 & 17 & 118,8 & 18 & 123,0 & 19 & 127,3 & 20 & 131,8 \\ \hline
	21 & 136,5 & 22 & 141,3 & 23 & 146,2 & 24 & 151,4 & 25 & 156,7 \\ \hline
	26 & 162,2 & 27 & 167,9 & 28 & 173,8 & 29 & 179,9 & 30 & 186,2 \\ \hline
	31 & 192,8 & 32 & 203,5 & 33 & 210,7 & 34 & 218,1 & 35 & 225,7 \\ \hline
	36 & 233,6 & 37 & 241,8 & 38 & 250,3 &    &       &    &
\end{tabular}

\subsection{CTCSS-zoner i Sverige}

Rekommendationer för repeatrar i olika distrikt och län att använda CTCSS för att hindra att störningar uppkommer vid conds mm. Det ger också möjligheten för sändaramatörer att öppna just den repeater man önskar om man har flera på samma frekvens omkring sig.

\begin{tabular}{lcccc}
	\textbf{Område}    & \textbf{Primär} & \textbf{Sek. 1} & \textbf{Sek. 2} & \textbf{Sek. 3} \\ \hline
	Distrikt 0         & 77,0            & 123.0           & 67.0            & 100.0           \\
	Distrikt 1         & 218.1           & 233.6           &                 &                 \\
	Distrikt 2         & 107.2           & 146.2           & 162.2           & 186.2           \\
	Distrikt 3         & 127.3           & 141.3           & 250.3           &                 \\
	D4 Värml. / Örebro & 74.4            & 151.4           &                 &                 \\
	D4 Dalarna         & 85.4            & 151.4           &                 &                 \\
	Distrikt 5         & 82.5            & 91.5            & 103.5           & 203.5           \\
	Distrikt 6         & 114.8           & 118,8           & 94.8            & 131.8           \\
	Distrikt 7         & 79.7            & 156.7           & 210.7           & 
\end{tabular}

Lägg märke till att decimalsiffran är i de flesta fall samma som distriktssiffran.

\section{Bandbredd olika modulationer}

Olika modulationer upptar olika bandbredd. Detta är mycket viktigt att förstå när man ställer in sin radiostation. Detta gäller särskilt att beakta i närheten av nödfrekvenser eller bandkanten. När vi talar om bandbredder här förstås den bandbredd vari minst 98 \% av signalens effekt befinner sig.

\begin{tabular}{lrl}
	\textbf{Modulation} & \textbf{Bandbredd} & \textbf{Kommentarer}                  \\ \hline
	CW                  &          $<$500 Hz & Smalbandigt                           \\
	AM                  &              6 kHz & Amplitudmodulering med fullt sidband  \\
	SSB*                &              3 kHz & Amplitudmodulering med enkelt sidband \\
	NFM**               &           7-12 kHz & Smalbandig FM                         \\
	FM                  &             16 kHz & Normal FM                             \\
	WFM                 &          16-75 kHz & Bredbandig FM (t.ex. rundradio)
\end{tabular}

*) För SSB gäller att USB och LSB fungerar lite olika. När man beräknar den högsta eller lägsta frekvensen utgår man från den inställda frekvensen $f$. För USB gäller då att högsta frekvensen är $f+3$ kHz. För LSB blir det $f-3$ kHz. Detta innebär att om du sänder på 80m bandet och du får sända telefoni från 3600--3800 kHz och vill lägga dig i under bandkanten och köra LSB skall du ställa in din radio på 3603 kHz som lägsta frekvens. Använd gärna lite marginal och kör exempelvis 3605 kHz i stället.

*) På kortvågen används smalbandig FM med max 10 kHz modulationsbandbredd. På VHF/UHF är smalbandig FM ca 11 kHz bred med Carssons bandbreddsbegrepp.

Det är vanligt att t.ex. utländska stationer kör överdriven bandbredd. Antingen som en följd av att man vill öka sin modulationsvinst, okunskap eller man har skruvat i sin radio.

\section{Övergripande frekvensplan}

\subsection{Indelning efter frekvens och våglängd}
\label{frekvens-vaglangd}

\begin{tabular}{llrlrl}
	\textbf{Förk.} & \textbf{Benämning}   & \textbf{Frekvens} &     & \textbf{Våglängd} &  \\ \hline
	ELF            & Extremt låg frekvens &             3--30 & Hz  &           10--100 & Mm \\
	SLF            & Superlåg frekvens    &           30--300 & Hz  &             1--10 & Mm \\
	ULF            & Ultralåg frekvens    &         300--3000 & Hz  &         100--1000 & km \\
	VLF            & Väldigt låg frekvens &             3--30 & kHz &           10--100 & km \\
	LF (LV)        & Låg frekvens         &           30--300 & kHz &             1--10 & km \\
	MF (MV)        & Mellanfrekvens       &         300--3000 & kHz &         100--1000 & m  \\
	HF (KV)        & Högfrekvens          &             3--30 & MHz &             1--10 & m  \\
	VHF (UKV)      & Väldigt hög frekvens &           30--300 & MHz &             1--10 & m  \\
	UHF            & Ultrahög frekvens    &         300--3000 & MHz &         100--1000 & mm \\
	SHF            & Superhög frekvens    &             3--30 & GHz &           10--100 & mm \\
	EHF            & Extremt hög frekvens &           30--300 & GHz &             1--10 & mm \\
	THF            & Terahertsfrekvens    &          300-3000 & GHz &         100--1000 & µm
\end{tabular}

Benämningarna HF, MF och LF har också andra betydelser. Exempelvis an\-vänd\-s HF som beteckning av den signal en antenn tar mot eller sänder oavsett frekvensband, MF kan vara mellansignalen oavsett frekvens efter omvandling i en superheterodynmottagare och LF, ibland benämnt AF (audiofrekvens) är det hörbara ljudet, dvs den modulation som används på signalen.

På engelska används i stället benämningarna RF för radio frequency, IF för intermediate frequency and AF för audio frequency vilket rekommenderas då sammanblandningsrisk med ITU-benämningarna på spektrum inte föreligger.

Amatörradioband finns inom de flesta av dessa frekvensband utom de högsta och lägsta frekvenserna.

\subsection{Rundradiobenämningar och frekvensband}

\begin{tabular}{llrl}
\textbf{Förk.} & \textbf{Namn} & \textbf{Frekvens Rundradio} &     \\ \hline
LW/LV          & Långvåg       & 148,5--285                  & kHz \\
MW/MV          & Mellanvåg     & 526,5--1606,5               & kHz \\
SW/KV          & Kortvåg       & 4,3--30                     & MHz \\
UKV            & Ultrakortvåg  & 88--108                     & MHz \\
\end{tabular}

\subsection{Radarband och benämningar enligt ITU}

\begin{tabular}{lrl}
	\textbf{Band} & \textbf{Frekvens} & \textbf{Benämning}          \\ \hline
	HF            &   0.003--0.03 GHz & High frequency              \\
	VHF           &     0.03--0.3 GHz & Very hgh frequency          \\
	UHF           &        0.3--1 GHz & Ultra high frequency        \\
	L             &          1--2 GHz & Long wave                   \\
	S             &          2--4 GHz & Short wave                  \\
	C             &          4--8 GHz &  \\
	X             &         8--12 GHz & Anv. under 2:a världskriget \\
	Ku            &        12--18 GHz & ''Kurz under''              \\
	K             &        18--27 GHz & Tyska ''Kurz'' (kortvåg)    \\
	Ka            &        27--40 GHz & Kurz-above (över)           \\
	V             &        40--75 GHz &  \\
	W             &       75--110 GHz &  \\
	mm            &      110--300 GHz & Millimetervågor
\end{tabular}

\subsection{Egenskaper olika frekvensband}

För radioamatörer delar man in frekvensbanden i långvåg, mellanvåg, kortvåg, VHF, UHF och SHF beroende på frekvens, se tabellen under avsnitt \ref{frekvens-vaglangd}. Dessa har lite olika utbredningsegenskaper.

\textbf{Långvåg} -- Markvågsutbredning, relativt höga sändareffekter, tillförlitliga för\-bind\-el\-ser men i övre delen av frekvensbandet kortare förbindelser dagtid. På de lägsta frekvenserna erhålls med hög sändareffekt goda förbindelser på stora avstånd globalt och används även för t.ex. malmprospektering, kommunikation med ubåtar i undervattensläge.

\textbf{Kortvåg} -- God rymdvågsutbredning med mycket lång räckvidd redan med låg effekt men samtidigt starkt avhängigt radiokonditionerna. Med ökande frekvens blir jonosfärreflektionen allt flackare vilket resulterar i en alltmer uttalad död zon (skip). Särskilt utmärkande för kortvågen är att den redan med låg effekt ger under gynnsamma konditioner extremt lång räckvidd via rymdvåg, ibland globalt.

\textbf{Mellanvåg} -- Kombinerar egenskaperna hos angränsande delar av lång- och kortvåg, kan ge kraftig interferens mellan rymd- och markvåg som ofta upplevs som kraftig fädning. Särskilt utmärkande för mellanvågen är den i det närmaste avsaknanden av skipzon eftersom mark- och rymdvåg kompletterar varandra, jonosfärens D-skikt är heller inte särskilt uttalat i frekvensområdet dvs förbindelser via rymdvåg på korta avstånd mellan 100--300\,km är möjliga även dagtid under perioder med kraftig solaktivitet.

\textbf{Ultrakortvåg (VHF)} -- Förbindeleser med låg effekt och små antenner, oberoende av jonosfären men då endast i form av frirumsutbredning, dvs fram till horisonten och under påverkan av terränghinder mm. Särskilt utmärkande för UKV är att rymdvåg saknar, markvågsdämpningen till lands är total och kommunikation på högre frekvenser i princip därför bara sker vid fri sikt mellan sändare och mottagare.

