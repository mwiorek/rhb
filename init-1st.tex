%%%%%%%%%%%%%%%%%%%%%%%%%%%%%%%%%%%%%%%%%%%%%%%%%%%%%%%%%%%%%%%%%%%%%%%%%%%%
% diverse saker här som används i dokumentet %%%%%%%%%%%%%%%%%%%%%%%%%%%%%%%
%%%%%%%%%%%%%%%%%%%%%%%%%%%%%%%%%%%%%%%%%%%%%%%%%%%%%%%%%%%%%%%%%%%%%%%%%%%%
\newcommand{\SubtitleText}{För Sändaramatörer, privatradioanvändare\\
	och andra radiointresserade}
\newcommand{\Forfattare}{Täpp-Anders Sikvall}
\newcommand{\Initialer}{SM5UEI}
\newcommand{\DokYear}{19}
\newcommand{\DokVersion}{2.2.3} % Glöm inte fixa versionen här!
\newcommand{\DokumentDatum}{\today}
%%%%%%%%%%%%%%%%%%%%%%%%%%%%%%%%%%%%%%%%%%%%%%%%%%%%%%%%%%%%%%%%%%%%%%%%%%%
\documentclass[10pt,swedish,a4paper]{article}    % De viktigaste inställningarna för dokumentet
\usepackage[a4paper,bindingoffset=1cm]{geometry} % Särskilda dokumentinställningar 
\usepackage[utf8]{inputenc}            % Indata som UTF-8 är väldigt bra för t.ex. svenska
\usepackage[T1]{fontenc}               % Fontkodning enligt modern standard
\usepackage[swedish]{babel}            % Svensk avstavning och mycket mer
\usepackage[colorlinks,linkcolor=black,urlcolor=blue]{hyperref}
                                       % Hyperref möjliggör länkar utanför dokumentet
\usepackage{enumitem}                  % Enumrering 
\usepackage{multirow}                  % Tabeller med tabellrader som spänner över mer än en textrad
\usepackage{amsmath}                   % Utökade matematiska symboler
\usepackage{tabularx}                  % Tabulering med fler optioner än original
\usepackage{graphicx}                  % Kunna sätta bilder bättre
\usepackage[table,x11names]{xcolor}    % Referera till färger med deras X11-namn
\usepackage{fancyhdr}                  % Kunna snyggare typsätta huvud och fot på varje sida
\usepackage[yyyymmdd]{datetime}        % Mer europeiskt datumformat (ISO)
  \renewcommand{\dateseparator}{-}
\usepackage{lastpage}                  % Kunna referera till sista sidan i dokumentet
\usepackage{pdfpages}                  % Kunna lägga in PDF-kommentarer
\usepackage{icomma}                    % Intelligenta kommatteringar för stora och små tal
\usepackage{pdflscape}                 % Kunna göra vissa sidor liggande i PDF
\usepackage{wrapfig}                   % Kunna wrappa figurer
\usepackage{float}                     % Utökade optioner för floats
\usepackage[bottom]{footmisc}          % Optioner för fotnoter, sätt alltid på slutet av samma sida
\usepackage{longtable}                 % Tabeller som sidbryter
\usepackage{fbb}                       % En font inspirerad av Bembo
\newif\iftitlefoot
\raggedbottom                          % Stretcha inte sidor mer än nödvändigt
\setlist{nosep}                        % Separera inte numrerade och punktlistor mer
\usepackage{todonotes}                 % Kunna lägga in att-göra i dokumentet
\renewcommand{\arraystretch}{1.15}

