\subsection{PR-bandet 27\,MHz}

Detta är det enda bandet som allmänheten kan använda på HF-bandet. Det delar många egenskaper med 31\,MHz jaktradiobandet men är ett band som är äldre och mer etablerat.

Maximal uteffekt på bandet är 4W RMS ERP dvs antennvinst överstigande en 1/2-vågs dipol (0 dBd, 2.12 dBi) måste inräknas i effekten efter avdrag för matningsförlust. Modulationsslag AM, FM och SSB (primärt används USB) är tillåtet på alla kanaler i dag. Traditionellt används kanal 24 för USB men i dag får vilken kanal som helst användas.

Kanalerna med A efter är upplåtna för radiostyrning och inte för telefoni. Undvik därför att använda dessa om du har en sändare som kan använda dessa frekvenser. De är med i tabellen för den skall vara komplett. 

\clearpage

\begin{longtable}{rrl|rrl}
	\textbf{Frekvens}& \textbf{Kanalnr}& \textbf{Övrigt}        
	& \textbf{Frekvens} & \textbf{Kanalnr} & \textbf{Övrigt}  \\
	\hline \endhead
	  26,965 &       1 &                &   27,215 &      21 &          \\
	  26,975 &       2 &                &   27,225 &      22 &          \\
	  26,985 &       3 &                &   27,255 &      23 &          \\
	  26,995 &      3A & Radiostyrning  &   27,235 &      24 & SSB      \\
	  27,005 &       4 &                &   27,245 &      25 &          \\
	  27,015 &       5 &                &   27,265 &      26 &          \\
	  27,025 &       6 &                &   27,275 &      27 &          \\
	  27,035 &       7 &                &   27,285 &      28 &          \\
	  27,045 &      7A & Radiostyrning  &   27,295 &      29 &          \\
	  27,055 &       8 &                &   27,305 &      30 &          \\
	  27,065 &       9 &                &   27,315 &      31 &          \\
	  27,075 &      10 &                &   27,325 &      32 &          \\
	  27,085 &      11 &                &   27,335 &      33 &          \\
	  27,095 &     11A & Tid. nödfrekv. &   27,345 &      34 &          \\
	  27,105 &      12 &                &   27,355 &      35 &          \\
	  27,115 &      13 &                &   27,365 &      36 &          \\
	  27,125 &      14 &                &   27,375 &      37 &          \\
	  27,135 &      15 &                &   27,385 &      38 &          \\
	  27,155 &      16 &                &   27,395 &      39 &          \\
	  27,165 &      17 &                &   27,405 &      40 &          \\
	  27,175 &      18 &                &          &         &          \\
	  27,185 &      19 &                &          &         &          \\
	  27,195 &     19A & Radiostyrning  &          &         &          \\
	  27,205 &      20 &                &          &         &        
\end{longtable}

Många apparater är endast FM i dag men det finns de som också har SSB. Äldre apparater hade oftast AM och FM och ibland även SSB. Telegrafi körs i princip inte på PR-bandet, troligen för att det aldrig varit några krav på det och de som kör heller inte haft möjlighet förr i tiden att DX-a på bandet. 

Innan Televerket släppte upp bestämmelserna var det väldigt hårda bestämmelser på bandet, i princip var det bara kommunikation inom familjen som tilläts. I dag kan bandet användas som man vill och det är på sina håll god aktivitet. 

Kom ihåg att inte överskrida effektbegränsningarna bara.