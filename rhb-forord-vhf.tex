\section*{Förord}
	
Det du nu håller i din hand eller läser på din skärm är resultatet av
en omarbetning av den tidigare radiohandboken. Denna version kommer
sig av att det var dags att uppdatera layouten och även passa på att
städa upp lite i boken så att den blev lite mer strukturerad.
	
Denna del innehåller VHF/UHF men inte HF. Boken finns i två andra
delar, en som innehåller HF men inte VHF/UHF och en som innehåller
båda sektionerna.
	
De tidigare versionerna särskilt tänkta att läsas på platta,
ebokläsare eller mobil har utgått. Detta för att de inte var särskilt
populära eller ofta nedladdade liksom att det tar en del tid i anspråk
att underhålla flera olikva versioner av samma material. Eftersom
sidstorleken för läs- och surfplattor är bäst som A5 ungefär innebär
det också en del problem med layout.
	
Hur som helst hoppas jag att ni ska tycka att denna version innehåller
mycket matnyttigt. Som vanligt är det bara att maila era synpunkter
till anders@sikvall.se så kan vi se om vi kan uppdatera dessa till
kommande versioner.
	
Bidrag till boken tas tacksamt mot men jag kommer bedöma ifall
materialet är lämpligt att ta med. I denna utgåva har även PTS
bandplaner bedömts som överflödigt material eftersom vi ändå uppnått
ganska många sidor med det viktigaste materialet. PTS bandplan
återfinns ändå på deras hemsida.
	
Och kör radio där ute. Mobilt. Stabilt. Med stil.\\[4em]
	
Karlholm, \today\\
\textit{Täpp-Anders Sikvall}

\clearpage
