\subsection{Amatörradiocertifikat}

Amatörradio har en lång historia och sträcker sig tillbaka till
radions barndom. Rent formellt så var det i USA i slutet av 1890-talet
som radioamatörer började sända telegram till varandra med den teknik
som fanns till buds då. Det blev mycket populärt bland elingenjörer
och andra teknikintresserade och runt 1910 började man få problem med
interferens och störningar ochg beslutade sig för att formalisera det
hela. Olika restriktioner infördes men i och med detta regelverk fick
vi också vissa krav på kunskaper för att operera radiosändare.

\subsubsection{Amatörradiocertifikat HAREC}

Om det fulla amatörradiocertifikatet är ett certifikat som är utformat
efter de principer som anges i HAREC T/R 61-02, Harmonised Amateur
Radio Examination Certificate, Vilnius 2004, uppdaterad 2014 och 2016.

Detta är ett ganska omfattande dokument och i Sverige så har vi t.ex.
KonCEPT-boken som används för ubildning av nya radioamatörer, den kan
laddas ned som PDF från ssa.se eller beställas som papperbok i deras
webshop. För att bli radioamatör behöver man svara tillräckligt många
rätt på två stycken delprov, ett teknikprov som avhandlar elektriska
kretsar, radioteknik, sändare och mottagare, vågutbredning,
eletromagnetiska fält, grundläggande matematik och fysik som är
tillämplig, viss komponentlära, förståelse för störningar och att bli
störd, filter och antenner mm. Det andra provet är ett prov över
reglementet som tillämpas, både lagar och författningar som reglerar
amatörradio men även saker som mer praktiska som exempelvis
bokstaveringsalfabetet och Q-koder mm.

Ett godkänt sådant prov ger möjlighet att operera som radioamatör på
en stor mängd olika frekvensband, alla de som i Sverige är utmärkta
som amatörradioband.

\subsubsection{Amatörradiocertifikat insteg}
\label{sec:instegscertifikat}

Instegscertifikatet är ett förenklat teknikprov men ungefär samma prov
vad gäller reglementen osv. Det förenklade teknikprovet innebär dock
en del begränsningar eftersom instegsamatören inte kan ges riktigt
samma förtroende att sända på alla frekvenser. Exempelvis har man
därmed begränsat de frekvensband som får användas till sådana som är
exklusiva för amatörradio i Sverige.

Konskapskraven i Sverige skall motsvara det som finns i ''ECC Report 89'': \\
\url{https://docdb.cept.org/download/409}. 

PTS förväntas delegera examineringen till SSA och deras utsedda provförrättare 
och en särskild utbildningbok för instegscertet håller på att tas fram just 
nu och ska snart gå till tryck (Mars 2025).

Tyvärr fick inte SSA gehör för sin remiss att även öppna 70 cm bandet 
för instegscertifikat. Bandet är inte länge ett exklusivt amatörband 
men vi delar det med lågeffektssändare för fjärrstyrning mm och det
bör inte finnas hinder då fullcertare får sända med 200\,W i bandet och
kan söka högeffektstillstånd på upp till 1\,kW.

Radioamatörer med instegscertifikat tilldelas anropssignaler i serien SH 
som även tidigare använts för noviser. Vid uppgradering till fullt cert
får man i stället en signal i SA-serien är tanken.

\subsubsection{Amatörradioband för instegscertifikat}

\begin{table}[H]
\centering
\begin{tabular}{rrlr}
	\textbf{Frekvens} & \textbf{Våglängd} & \textbf{Notiser}               & \textbf{Max Effekt} \\ \hline
	            7 MHz &          40 meter &                                &           25\,W PEP \\
	           14 MHz &          20 meter & 25\,W PEP                      &                     \\
	           21 MHz &          15 meter &                                &           25\,W PEP \\
	           28 MHz &          20 meter & Liknande utbredning som 27 MHz &           25\,W PEP \\
	           50 MHz &           6 meter &                                &           25\,W PEP \\ \hline
	          144 MHz &           2 meter &                                &           25\,W ERP
\end{tabular}
\caption{Amatörradioband i sverige upp till 70 cm}
\end{table}

50\,MHz är egentligen ett VHF-band men räknas ofta till kortvåg (HF) 
då de flesta amatörradiostationer för kortvåg omfattar även 6\,m-bandet.

Myndigheterna har gett instegscertare tillträde till amatörradioband
som är exklusiva för amatörradio vilket innebär att ett antal populära
band inte finns med. Arbete fortsätter på att försöka tillföra även 70
cm bandet (432--438 MHz) till de band som är tillåtna för
instegsamatörer men det har inte kommit med i denna utgåva av PTS
undantagsföreskrift.

\subsubsection{Effektrestriktioner instegscertifikat}

På frekvensbanden 7, 14, 21, 28 och 15 MHz tillåter man 25 W PEP
tillfört antennsystemet. Detta är effekten vid en key-down på
morsenyckel. Antennvinsten är inte medräknad på dessa frekvensband.

För 144 MHz däremot är det 25\,W ERP vilket innebär att du får mata en
halvvågsdipol direkt med 25\,W men om du ska använda riktantenner med
högre antennförstärkning än 0\,dBd eller 2,15\,dBi så måste du sänka
sändareffekten i motsvarande grad.

Anledningen till denna skillnad är att det är ganska svårt att bygga
antenner med hög förstärkning, eller att bestämma dess faktiska
förstärkning på kortvågsbanden medan det för 145\,MHz och uppåt är
betydligt enklare att få till det och dessutom mäta och bestämma
antennens förstärkning.
 
\subsubsection{Restriktioner på utrustning för instegscertifikat}

Som instegsamatör får man inte operera hemmabyggda sändare, modifierade
sändare eller andra sändare som saknar CE-märkning. Det innebär att man måste
hålla sig till fabriksbyggd utrustning som säljs på den europeiska marknaden. 

\subsubsection{Instegscertifikat utomlands}

Certifikatet äger heller ingen giltighet utomlands så som det fulla,
HAREC-baserade, amatörradiocertifikatet gör. Du får därför i de flesta fall
inte ta med din radiosändare till andra länder och använda den där på 
instegscertifikatet.

