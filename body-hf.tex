\clearpage

\section{Trafik- och tumregler HF}
\subsection{Kort sammanfattning av reglemente}

OBS! Detta är inte fullständigt radioreglemente naturligtvis utan endast sammanfattning av några viktiga punkter.

\subsubsection{Begrepp i bandplanerna}

\begin{itemize}
\item QRP: Aktivitetscentrum för låg effekt ($<$5W), svaga signaler förekommer, visa hänsyn.
\item QRS: Aktivitetscenter för långsam CW.
\item QRSS: Extremt långsam CW med dator.
\item DV: Digital Voice.
\item Image: Bildmoder exempelvis SSTV och Fax som ryms inom den specificerade maximala bandbredden.
\end{itemize}

\subsubsection{Trafikregler och tumregler}

\begin{itemize}
\item Vid SSB-telefoni används LSB på frekvenser under 10 MHz och USB på frekvenser över 10 MHz.
\item Lägsta acceptabla inställda frekvens för LSB är 3 kHz över under bandkant! 
\item Högsta acceptable inställda frekvens för USB är 3 kHz under övre bandkant!
\item IBP är International Beacon Project. Fyrarna sänder med 3 min intervaller och används för att studera utbredningen av radiosignaler globalt. Fyrarna sänder anrop och fyra 1 s toner. Anropet och första tonen sänds med 100W, därefter sänds tonerna med 10W, 1W samt 100mW.
\item Vid AM (A3J) skall hänsyn tas så att störningar på annan trafik ej fö\-re\-kom\-mer med de sidband som då uppstår, det gäller då både övre och undre sidbandet.
\item Ingen som helst sändning är tillåtet inom fyrsegmenten. Detta skall respekteras. Lyssna gärna på nödfrekvenserna men används dem icke, om det inte är du som svarar på ett nödsamtal! Undvik QSO allt för nära dessa också.
\item Var särskilt uppmärksam på satelliters nerlänksfrekvenser på 10 m-bandet. I detta segment skall endast lyssning ske. Ingen sändning är tillåten här eller i skyddssegmentet strax ovanför satellitsegmentet. Tänk på att satelliters frekvens kan dopplerskiftas uppåt en hel del när de rör sig mot mottagaren.
\end{itemize}

\subsection{Fyrar}

\subsubsection{IBP---International Beacon Project}

Det finns flera olika typer av fyrar men för HF är IBP (International Beacon Project) intressant eftersom det ger operatören möjlighet att utröna hur utbredningen ser ut för stunden genom att lyssna efter fyrar. Tabellen nedan visar anropssignaler och första sändningsslotten som fyren sänder, dvs SCHED för olika fyrar och frekvenser.

Fyrarna har gemensam hårdvara och synkroniseras mot tidsreferens. Fyrar kan vara offline av olika skäl, kontrollera mot IBP:s hemsida om du inte hör en fyr du brukar höra, den kan vara off line.


\subsubsection{Lista över IBP-fyrar}
\begin{tabular}{llrrrrr}
\textbf{Signal} & \textbf{QTH} & \textbf{14 100} & \textbf{18 110} & \textbf{21 150} & \textbf{24 930} & \textbf{28 200} \\ \hline

4U1UN  & United Nations & 00:00  & 00:10  & 00:20  & 00:30  & 00:40  \\ 
VE8AT  & Canada         & 00:10  & 00:20  & 00:30  & 00:40  & 00:50  \\
W6WX   & United States  & 00:20  & 00:30  & 00:40  & 00:50  & 01:00  \\
KH6RS  & Hawaii         & 00:30  & 00:40  & 00:50  & 01:00  & 01:10  \\
ZL6B   & New Zealand    & 00:40  & 00:50  & 01:00  & 01:10  & 01:20  \\
VK6RBP & Australia      & 00:50  & 01:00  & 01:10  & 01:20  & 01:30  \\
JA2IGY & Japan          & 01:00  & 01:10  & 01:20  & 01:30  & 01:40  \\
RR9O   & Russia         & 01:10  & 01:20  & 01:30  & 01:40  & 01:50  \\
VR2B   & Hong Kong      & 01:20  & 01:30  & 01:40  & 01:50  & 02:00  \\
4S7B   & Sri Lanka      & 01:30  & 01:40  & 01:50  & 02:00  & 02:10  \\
ZS6DN  & South Africa   & 01:40  & 01:50  & 02:00  & 02:10  & 02:20  \\
5Z4B   & Kenya          & 01:50  & 02:00  & 02:10  & 02:20  & 02:30  \\
4X6TU  & Israel         & 02:00  & 02:10  & 02:20  & 02:30  & 02:40  \\
OH2B   & Finland        & 02:10  & 02:20  & 02:30  & 02:40  & 02:50  \\
CS3B   & Madeira        & 02:20  & 02:30  & 02:40  & 02:50  & 00:00  \\
LU4AA  & Argentina      & 02:30  & 02:40  & 02:50  & 00:00  & 00:10  \\
OA4B   & Peru           & 02:40  & 02:50  & 00:00  & 00:10  & 00:20  \\
YV5B   & Venezuela      & 02:50  & 00:00  & 00:10  & 00:20  & 00:30  \\
\end{tabular}

\begin{landscape}
\section{Frekvenser Amatörradio LF/MF/HF}
\subsection{Bandplaner LF/MF/HF}
Alla frekvenser i kHz, bandbredder i Hz.

\subsubsection{Bandplan 2.2\,km, 135,7--137,8\,kHz}
\begin{tabular}{rrrll}
\textbf{Frekvens} &  & \textbf{BW} & \textbf{Trafik} & \textbf{Noteringar} \\ \hline
135,7 & 135,8 & 200 & CQ, QRSS, Digi & OBS! Högsta effekt 1W ERP. \\ \hline
\end{tabular}

\subsubsection{Bandplan 600\,m, 472--479\,kHz}
\begin{tabular}{rrrll}
\multicolumn{2}{c}{\textbf{Frekvens}} & \textbf{BW} & \textbf{Trafik} & \textbf{Noteringar} \\ \hline
472 & 479 & 200 & CW, QRSS, Digi & OBS! Högsta utstrålad effekt 1W EIRP \\ \hline
\end{tabular}

\subsubsection{Bandplan 160\,m, 1810--2000\,kHz}
\begin{tabular}{rrrll}
\multicolumn{2}{c}{\textbf{Frekvens}} & \textbf{BW} & \textbf{Trafik} & \textbf{Noteringar} \\ \hline
1 810 & 1 838 & 200  & CW         & Exklusivt för CW. Interkontinental trafik har prio. \\ \hline
1 838 & 1 840 & 500  & Smalband   & Ej packet på 160m, PSK 1 838,150                    \\ \hline
1 840 & 1 850 & 2700 & Alla moder & Även digimode. SSB QRP 1 843 kHz                    \\ \hline
1 850 & 1 900 & 2700 & Alla moder & OBS! Max 10 W till ant.                             \\ \hline
1 900 & 1 950 & 2700 & Alla moder & OBS! Max 100 W till ant.                            \\ \hline
1 950 & 2 000 & 2700 & Alla moder & OBS! Max 10 W till ant.                             \\ \hline
\end{tabular}

\subsubsection{Bandplan 80\,m, 3500--3800\,kHz}
\begin{tabular}{rrrll}
\multicolumn{2}{c}{\textbf{Frekvens}} & \textbf{BW} & \textbf{Trafik} & \textbf{Noteringar} \\ \hline
3 500 & 3 510 & 200  & CW             & Exklusivt CW                         \\ 
      &       &      &                & Interkontinental DX-trafik har prio  \\ \hline
3 510 & 3 580 & 200  & CW             & Exklusivt CW contest 3510-–560       \\ 
      &       &      &                & CW QRS 3 555 kHz, CW QRP 3 560       \\ \hline
3 580 & 3 600 & 500  & Smalband, Digi & PSK 3 580,150                        \\
      &       &      &                & Automatiska Digimoder 3 590--600     \\ \hline
3 600 & 3 620 & 2700 & Alla moder     & Digimoder Automatiska Digimoder      \\ \hline
3 600 & 3 650 & 2700 & Alla moder     & SSB contest 3 600--650               \\
      &       &      &                & DV 3 630                             \\ \hline
3 650 & 3 700 & 2700 & Alla moder     & SSB QRP 3 690                        \\ \hline
3 700 & 3 800 & 2700 & Alla moder     & Contest 3 700-–800                   \\
      &       &      &                & Image 3 775                          \\
      &       &      &                & Region 1 nödfrekvens 3 760           \\ \hline
3 775 & 3 800 & 2700 & Alla moder     & Interkontinental DX-trafik prioritet \\ \hline
\end{tabular}

\subsubsection{Bandplan 40\,m, 7000--7200\,kHz}
\begin{tabular}{rrrll}
\multicolumn{2}{c}{\textbf{Frekvens}} & \textbf{BW} & \textbf{Trafik} & \textbf{Noteringar} \\ \hline
7\,000 & 7\,040 & 200  & CW         & Exklusivt CW.                             \\
      &       &      &            & QRP aktivitetscentrum 7\,030\,kHz           \\ \hline
7\,040 & 7\,050 & 500  & Smalband   & Digimoder Automatiska inom 7\,047–-050\,kHz \\ \hline
7\,050 & 7\,060 & 2700 & Alla moder & Digimoder Automatiska inom 7\,050–-053\,kHz \\ \hline
7\,060 & 7\,100 & 2700 & Alla moder & SSB contest i segmentet                   \\
      &       &      &            & DV 7 070 kHz, SSB QRP 7\,090 kHz           \\ \hline
7\,100 & 7\,130 & 2700 & Alla moder & Region 1 nödfrekvens 7\,110 kHz            \\ \hline
7\,130 & 7\,200 & 2700 & Alla moder & SSB contest i segmentet                   \\
      &       &      &            & Image 7\,165\,kHz                           \\ \hline
7\,175 & 7\,200 & 2700 & Alla moder & Interkontinental DX-trafik prio           \\ \hline
\end{tabular}

\subsubsection{Bandplan 30 m, 10100--10150 kHz}
\begin{tabular}{rrrll}
\multicolumn{2}{c}{\textbf{Frekvens}} & \textbf{BW} & \textbf{Trafik} & \textbf{Noteringar} \\ \hline
10\,100 & 10\,140 & 200 & CW       & CW exkl. Max 150 Watt på 30 m    \\
       &        &     &          & CW QRP 10\,116\,kHz                     \\ \hline
10\,140 & 10\,150 & 500 & Smalband & Digimoder PSK 10142,150\,kHz. Ej Packet \\ \hline
\end{tabular}

\subsubsection{Bandplan 20 m, 14000--14350 kHz}
\begin{tabular}{rrrll}
\multicolumn{2}{c}{\textbf{Frekvens}} & \textbf{BW} & \textbf{Trafik} & \textbf{Noteringar} \\ \hline
14\,000 & 14\,070 & 200  & CW         & Exklusivt CW                            \\
       &        &      &            & Conctest 14\,000-–060                     \\
       &        &      &            & CW QRS 14 055, CW QRP 14\,060            \\ \hline
14\,070 & 14\,099 & 500  & Smalband   & PSK 14 070,150                          \\
       &        &      &            & Auto Digimoder 14 089-–099              \\ \hline
14\,099 & 14\,101 & 200  & Fyrar      & Exklusivt IBP, endast fyrar             \\ \hline
14\,101 & 14 \,12 & 2700 & Alla moder & Digitala moder och obevakade Digimoder  \\ \hline
14\,112 & 14\,350 & 2700 & Alla moder & SSB Contest 14 125--300                 \\
       &        &      &            & DV 14 130, DXpedition prio 14\,195$\pm$5 \\ \hline
14\,300 & 14\,350 & 2700 & Alla moder & Image 14\,230, SSB QRP 14\,285            \\
       &        &      &            & Global nödfrekvens 14 300               \\ \hline
\end{tabular}

\subsubsection{Bandplan 17 m, 18068--18168 kHz}
\begin{tabular}{rrrll}
\multicolumn{2}{c}{\textbf{Frekvens}} & \textbf{BW} & \textbf{Trafik} & \textbf{Noteringar} \\ \hline
18 068 & 18 095 & 200  & CW         & CW exklusivt. QRP 18 086             \\ \hline
18 095 & 18 109 & 500  & Smalband   & Digimoder PSK 18 100,150             \\
       &        &      &            & Automatiska Digimoder 18 105-–18 109 \\ \hline
18 109 & 18 111 & 200  & Fyrar      & Exklusivt fyrar, IBP fyrnät          \\ \hline
18 111 & 18 168 & 2700 & Alla moder & Digi 18 111–-18 120                  \\
       &        &      &            & SSB QRP 18 130, DV 18 150            \\
       &        &      &            & Global nödfrekv. 18 160\\ \hline
\end{tabular}

\subsubsection{Bandplan 15 m, 21000--21450 kHz}
\begin{tabular}{rrrll}
\multicolumn{2}{c}{\textbf{Frekvens}} & \textbf{BW} & \textbf{Trafik} & \textbf{Noteringar} \\ \hline
21 000 & 21 070 & 200  & CW         & Exklusivt CW, QRS 21 055, CW QRP 21 060          \\ \hline
21 070 & 21 110 & 500  & Smalband   & PSK 21080.150, Automatiska Digimoder 21 090–-110 \\
21 110 & 21 120 & 2700 & Alla moder & Alla moder utom SSB!                             \\
       &        &      &            & Digimoder, och Automatiska Digimoder             \\ \hline
21 120 & 21 149 & 500  & Smalband   &                                                  \\ \hline
21 149 & 21 151 & 200  & Fyrar      & Exklusivt fyrar. IBP fyrnät                      \\ \hline
21 151 & 21 450 & 2700 & Alla moder & DV 21 180, SSB QRP 21 285, Image 21 340          \\
       &        &      &            & Global nödfrekv. 21 360                          \\ \hline
\end{tabular}

\subsubsection{Bandplan 12 m, 24890--24990 kHz}
\begin{tabular}{rrrll}
\multicolumn{2}{c}{\textbf{Frekvens}} & \textbf{BW} & \textbf{Trafik} & \textbf{Noteringar} \\ \hline
24 890 & 24 915 & 200  & CW         & Exklusivt CW, QRP 24 906                             \\ \hline
24 915 & 24 929 & 500  & Smalband   & PSK 24 920.150, Automatiska Digimoder 24 925–-24 929 \\ \hline
24 929 & 24 931 & 200  & Fyrar      & Fyrar, IBP fyrnät                                    \\ \hline
24 931 & 24 990 & 2700 & Alla moder & Auto Digimoder 24 931-–24 940                        \\
       &        &      &            & SSB QRP 24 950, DV 24 960                            \\ \hline
\end{tabular}

\subsubsection{Bandplan 10 m, 28000-29700 kHz}
\begin{tabular}{rrrll}
\multicolumn{2}{c}{\textbf{Frekvens}} & \textbf{BW} & \textbf{Trafik} & \textbf{Noteringar} \\ \hline
28 000 & 28 070 & 200  & CW         & Exklusivt CW, QRS 28 055, CW QRP 28 060                \\ \hline
28 070 & 28 190 & 500  & Smalband   & PSK 28 120.150, Auto Digimoder inom 28 120--150        \\ \hline
28 190 & 28 199 & 200  & Fyrar IBP  & Regionala fyrar med tidsdelning                        \\ \hline
28 199 & 28 201 & 200  & Fyrar IBP  & IBP fyrnät                                             \\ \hline
28 201 & 28 225 & 200  & Fyrar IBP  & kontinuerligt sändande fyrar                           \\ \hline
28 225 & 28 300 & 2700 & Alla moder & Övriga fyrar                                           \\ \hline
28 300 & 28 320 & 2700 & Alla moder & Digimoder och Automatiska Digimoder                    \\ \hline
28 320 & 29 100 & 2700 & Alla moder & DV 28 330 kHz, SSB QRP 28 360 kHz                      \\
       &        &      &            & Image 28 680 kHz                                       \\ \hline
29 100 & 29 200 & 6000 & Alla moder & FM simplex, 10 kHz kanaler                             \\
       &        &      &            & Maximalt ±2.5 kHz dev., max 2.5 kHz mod.frek.          \\ \hline
29 200 & 29 300 & 6000 & Alla moder & Digimoder och Automatiska Digimoder                    \\ \hline
29 300 & 29 510 & 6000 & Satellit   & Nerlänk fr. satellit. EJ SÄNDNING I SEGMENTET          \\ \hline
29 510 & 29 520 & 6000 & Skydd      & Skyddsfrekvens för satelliter. EJ SÄNDNING I SEGMENTET \\ \hline
29 520 & 29 590 & 6000 & Alla moder & FM Repeater in RH1--8, 100 kHz duplex, 2.5 kHz NBFM    \\ \hline
29 600 & 29 620 & 6000 & Alla moder & FM simplex, anrop 29 600                               \\
       &        &      &            & FM simplex repeater 29 610                             \\ \hline
29 620 & 29 700 & 6000 & Alla moder & FM Repeater ut RH1--8, 100 kHz duplex                  \\ \hline
\end{tabular}
\end{landscape}

\clearpage


\begin{landscape}
\small
\subsection{PTS Bandplan för VLF, 3-30 kHz}
\begin{longtable}{lll}
\textbf{Frekvens} & \textbf{Användning}              & \textbf{Anmärkning}             \\ \hline \endhead
0,009--0,014	 & Radionavigering                 & Sjöfart  \\	 
0,009--0,014	 & Radionavigering                 & Luftfart \\	 
0,009--0,01995	 & Militär användning              &          \\
0,014--0,01995	 & Fast radio                      &          \\
0,014--0,01995	 & Sjöfartsradio                   &          \\
0,01995--0,02005 & Standardfrekvens och tidssignal &          \\ 
\end{longtable}

\subsection{PTS Bandplan för LF, 30-300 kHz}
\begin{longtable}{lll}
\textbf{Frekvens} & \textbf{Användning}  & \textbf{Anmärkning}                     \\ \hline \endhead
0,1176--0,126     & Militär användning   &                                         \\
0,1176--0,126     & Sjöfartsradio        &                                         \\
0,129--0,1485     & Militär användning   &                                         \\
0,129--0,1485     & Sjöfartsradio        &                                         \\
0,1357--0,1378    & \textbf{Amatörradio} & Undantag från tillståndsplikt (Max 1 W) \\
0,1485--0,2835    & Rundradio            & GE75                                    \\
\end{longtable}

\subsection{PTS Bandplan för MF, 300-3000 kHz}
\begin{longtable}{lll}
\textbf{Frekvens} & \textbf{Användning}              & \textbf{Anmärkning}             \\ \hline \endhead

0,315--0,6	 & Djurimplantat                   & Undantag från tillståndsplikt             \\
0,4--0,6	 & RFID                            & Undantag från tillståndsplikt             \\
0,405--0,415	 & Militär användning              &                                           \\	 
0,4065--0,4135	 & Radionavigering                 & Sjöfart                                   \\
0,415--0,435	 & Radionavigering                 & Luftfart GE85M                            \\
0,415--0,435	 & Militär användning              & GE85M                                     \\
0,415--0,495	 & Sjöfartsradio                   & GE85M                                     \\
0,415--0,435	 & Radiofyrar                      & Luftfart	GE85M                          \\
0,435--0,5265	 & Militär användning              &                                           \\ 
0,4569--0,4571	 & Lokalisering av nödställda      & Undantag från tillståndsplikt             \\
0,472--0,479	 & \textbf{Amatörradio}            & Undantag från tillståndsplikt (Max 1 W)   \\
0,495--0,5265	 & Sjöfartsradio                   &                                           \\
0,495--0,505	 & Luftfartsradio                  &                                           \\
0,51775--0,51825 & Sjöfartsradio                   & NAVTEX                                    \\
0,5265--1,6065	 & Rundradio                       & GE75                                      \\
1,6065--1,625	 & Militär användning              & GE85M                                     \\
1,6065--1,625	 & Sjöfartsradio                   & GE85M                                     \\
1,625--1,635	 & Militär användning              &                                           \\
1,625--1,635	 & Radiolokalisering               &                                           \\
1,635--1,8	 & Militär användning              & GE85M                                     \\
1,635--1,8	 & Sjöfartsradio                   & GE85M                                     \\
1,8--1,81	 & Militär användning              &                                           \\ 
1,8--1,81	 & Radiolokalisering               &                                           \\
1,81--1,85	 & \textbf{Amatörradio}                     & Undantag från tillståndsplikt             \\
1,85--2,045	 & Militär användning              &                                           \\
1,85--2,045	 & Fast radio                      &                                           \\
1,85--2,045	 & Sjöfartsradio                   &                                           \\
1,85--2	 	 & \textbf{Amatörradio}                     & Undantag från tillståndsplikt (Max 10 W)  \\
1,85--2,045	 & Landmobil radio                 &                                           \\
2,045--2,16	 & Militär användning              & GE85M                                     \\
2,045--2,16	 & Sjöfartsradio                   & GE85M                                     \\
2,16--2,1735	 & Militär användning              &                                           \\	 
2,16--2,17	 & Radiolokalisering               &                                           \\	 
2,17--2,1735	 & Sjöfartsradio                   &                                           \\
2,1735--2,1905	 & Sjöfartsradio                   & Internationella nöd- och anropsfrekvenser \\
                 &                                 & 2182 kHz (telefoni) och 2187,5 kHz (DSC)  \\
2,1905--2,498	 & Militär användning              &                                           \\	 
2,1905--2,498	 & Sjöfartsradio                   &                                           \\
2,194--2,498	 & Fast radio                      &                                           \\
2,194--2,498	 & Landmobil radio                 &                                           \\	 
2,498--2,502	 & Standardfrekvens och tidssignal &                                           \\
2,502--2,85	 & Militär användning              &                                           \\
2,502--2,625	 & Fast radio                      &                                           \\
2,502--2,85	 & Sjöfartsradio                   &                                           \\
2,502--2,625	 & Landmobil radio                 &                                           \\	 
2,625--2,65	 & Radionavigering för sjöfart     &                                           \\
2,65--2,85	 & Fast radio                      &                                           \\
2,65--2,85	 & Landmobil radio                 &                                           \\	 
\end{longtable}

\subsection{PTS Bandplan för HF, 3-30 MHz}
\begin{longtable}{lll}
\textbf{Frekvens} & \textbf{Användning}              & \textbf{Anmärkning}             \\ \hline \endhead
3,0215--3,0245    & Sjöfartsradio                    & Nöd- och säkerhetstrafik        \\
3,025--3,155      & Militär användning               & RR AP26                         \\
3,025--3,155      & Luftfartsradio                   & RR AP26                         \\
3,155--3,4        & Militär användning               &                                 \\
3,155--3,4        & Induktiva tillämpningar          & Undantag från tillståndsplikt   \\ 
                  &                                  & 2006/771/EG 2013/752/EU         \\
3,155--3,4        & Fast radio                       &                                 \\
3,155--3,4        & Sjöfartsradio                    &                                 \\
3,155--3,4        & Landmobil radio                  &                                 \\
3,4--3,5          & Luftfartsradio                   & RR AP27                         \\
3,5--3,9          & Militär användning               &                                 \\	 
3,5--3,9          & Fast radio                       &                                 \\
3,5--3,8          & Sjöfartsradio                    &                                 \\
3,5--3,8          & \textbf{Amatörradio}             & \textbf{80-metersbandet}        \\
3,5--3,9          & Landmobil radio                  &                                 \\
3,8--3,9          & Luftfartsradio                   &                                 \\
3,9--3,95         & Militär användning               & RR AP26                         \\
3,9--3,95         & Luftfartsradio                   & RR AP26                         \\
3,95--4,995       & Militär användning               &                                 \\
3,95--4,063       & Fast radio                       &                                 \\	 
3,95--4           & Rundradio                        &                                 \\
4--4,063          & Sjöfartsradio                    & Radiotelefoni                   \\
4,063--4,438      & Sjöfartsradio                    & RR AP25 gäller i del av bandet  \\
4,438--4,65       & Fast radio                       &                                 \\
4,438--4,65       & Sjöfartsradio                    &                                 \\
4,438--4,65       & Landmobil radio                  &                                 \\
4,65--4,7         & Luftfartsradio                   & RR AP27                         \\
4,7--4,75         & Luftfartsradio                   & RR AP26                         \\
4,75--4,995       & Fast radio                       &                                 \\
4,75--4,85        & Luftfartsradio                   &                                 \\
4,75--4,995       & Landmobil radio                  &                                 \\
4,995--5,005      & Fyr                              & Standardfrekvens och tidssignal \\
5,005--5,9        & Militär användning               &                                 \\
5,005--5,48       & Fast radio                       &                                 \\
5,06--5,45        & Sjöfartsradio                    &                                 \\
5,06--5,48        & Landmobil radio                  &                                 \\
5,45--5,48        & Luftfartsradio                   &                                 \\
5,48--5,68        & Luftfartsradio                   & RR AP27                         \\
5,6785--5,6815    & Sjöfartsradio                    & Nöd- och säkerhetstrafik.       \\
5,68--5,73        & Luftfartsradio                   & RR AP26                         \\
5,73--5,9         & Fast radio                       &                                 \\
5,73--5,9         & Landmobil radio                  &                                 \\
5,9--6,2          & Rundradio                        &                                 \\
6,2--6,525        & Militär användning               & RR AP25 gäller i del av bandet  \\
6,2--6,525        & Sjöfartsradio                    & RR AP25 gäller i del av bandet  \\
6,525--7          & Militär användning               &                                 \\
6,525--6,685      & Luftfartsradio                   & RR AP27                         \\
6,685--6,765      & Luftfartsradio                   & RR AP26                         \\
6,765--6,795      & Induktiva tillämpningar          & Undantag från tillståndsplikt   \\
6,765--7          & Fast radio                       &                                 \\
6,765--7          & Landmobil radio                  &                                 \\
6,765--6,795      & Allmän kortdistansradio          & Undantag från tillståndsplikt   \\
7--7,2            & \textbf{Amatörradio}             & \textbf{40-metersbandet}        \\
7,2--7,45         & Rundradio                        &                                 \\
7,4--8,8          & Induktiva tillämpningar          & Undantag från tillståndsplikt   \\
7,45--9,4         & Militär användning               &                                 \\
7,45--8,195       & Fast radio                       &                                 \\
7,45--8,1         & Landmobil radio                  &                                 \\
8,1--8,195        & Sjöfartsradio                    &                                 \\
8,195--8,815      & Sjöfartsradio                    & RR AP25 gäller i del av bandet  \\
8,815--8,965      & Luftfartsradio                   & RR AP27                         \\
8,965--9,04       & Luftfartsradio                   & RR AP26                         \\
9,04--9,4         & Fast radio                       &                                 \\
9,4--9,9          & Rundradio                        &                                 \\
9,9--9,995        & Militär användning               &                                 \\
9,9--9,995        & Fast radio                       &                                 \\
9,995--10,005     & Standardfrekvens och tidssignal  &                                 \\
10,005--10,1      & Luftfartsradio                   & RR AP27                         \\
10,1--11,175      & Militär användning               &                                 \\
10,1--11,175      & Fast radio                       &                                 \\
10,1--10,15       & \textbf{Amatörradio}             & \textbf{30-metersbandet}        \\
10,15--11,175     & Sjöfartsradio                    &                                 \\
10,15--11,175     & Landmobil radio                  &                                 \\
10,2--11          & Induktiva tillämpningar          & Undantag från tillståndsplikt   \\
11,175--11,275    & Luftfartsradio                   & RR AP26                         \\
11,275--11,4      & Luftfartsradio                   & RR AP27                         \\
11,4--11,6        & Militär användning               &                                 \\
11,4--11,6        & Fast radio                       &                                 \\
11,6--12,1        & Rundradio                        &                                 \\
12,1--12,23       & Militär användning               &                                 \\
12,1--12,23       & Fast radio                       &                                 \\
12,23--13,2       & Militär användning               & RR AP25 gäller i del av bandet  \\
12,23--13,2       & Sjöfartsradio                    & RR AP25 gäller i del av bandet  \\
12,5--20          & Djurimplantat                    & Undantag från tillståndsplikt   \\
13,2--13,57       & Militär användning               &                                 \\
13,2--13,26       & Luftfartsradio                   & RR AP26                         \\
13,26--13,36      & Luftfartsradio                   & RR AP27                         \\
13,36--13,57      & Fast radio                       &                                 \\
13,36--13,41      & Radioastronomi                   & Onsala rymd - observatorium     \\
13,41--13,57      & Sjöfartsradio                    &                                 \\
13,41--13,57      & Landmobil radio                  &                                 \\
13,553--13,567    & Induktiva tillämpningar          & Undantag från tillståndsplikt   \\
13,553--13,567    & RFID                             & Undantag från tillståndsplikt   \\
13,553--13,567    & Allmän kortdistansradio          & Undantag från tillståndsplikt   \\
13,553--13,567    & ISM                              &                                 \\
13,57--13,87      & Rundradio                        &                                 \\
13,87--14         & Militär användning               &                                 \\
13,87--14         & Fast radio                       &                                 \\
13,87--14         & Sjöfartsradio                    &                                 \\
13,87--14         & Landmobil radio                  &                                 \\
14--14,35         & \textbf{Amatörradio}             & \textbf{20-metersbandet}        \\
14,35--14,99      & Militär användning               &                                 \\
14,35--14,99      & Fast radio                       &                                 \\
14,35--14,99      & Sjöfartsradio                    &                                 \\
14,35--14,99      & Landmobil radio                  &                                 \\
14,99--15,01      & Standardfrekvens och tidssignal  &                                 \\
15,01--15,1       & Militär användning               &                                 \\
15,01--15,1       & Luftfartsradio                   & RR AP26                         \\
15,1--15,8        & Rundradio                        &                                 \\
15,8--16,36       & Militär användning               &                                 \\
15,8--16,36       & Fast radio                       &                                 \\
16,36--17,41      & Militär användning               & RR AP25 gäller i del av bandet  \\
16,36--17,41      & Sjöfartsradio                    & RR AP25 gäller i del av bandet  \\
17,41--17,48      & Militär användning               &                                 \\
17,41--17,48      & Fast radio                       &                                 \\
17,48--17,9       & Rundradio                        &                                 \\
17,9--17,97       & Luftfartsradio                   & RR AP27                         \\
17,97--18,068     & Militär användning               &                                 \\
17,97--18,03      & Luftfartsradio                   & RR AP26                         \\
18,03--18,068     & Fast radio                       &                                 \\
18,068--18,168    & \textbf{Amatörradio}             & \textbf{17-metersbandet}        \\
18,168--18,78     & Militär användning               &                                 \\
18,168--18,78     & Fast radio                       &                                 \\
18,168--18,78     & Sjöfartsradio                    &                                 \\
18,168--18,78     & Landmobil radio                  &                                 \\
18,78--18,9       & Militär användning               & RR AP25 gäller i del av bandet  \\
18,78--18,9       & Sjöfartsradio                    & RR AP25 gäller i del av bandet  \\
18,9--19,02       & Rundradio                        &                                 \\
19,02--19,68      & Militär användning               &                                 \\
19,02--19,68      & Fast radio                       &                                 \\
19,68--19,8       & Militär användning               & RR AP25 gäller i del av bandet  \\
19,68--19,8       & Sjöfartsradio                    & RR AP25 gäller i del av bandet  \\
19,8--19,99       & Militär användning               &                                 \\
19,8--19,99       & Fast radio                       &                                 \\
19,99--20,01      & Standardfrekvens och tidssignal  &                                 \\
20,01--21         & Militär användning               &                                 \\
20,01--21         & Fast radio                       &                                 \\
20,01--21         & Landmobil radio                  &                                 \\
21--21,45         & \textbf{Amatörradio}             & \textbf{14-metersbandet}        \\
21,45--21,85      & Rundradio                        &                                 \\
21,85--21,924     & Militär användning               &                                 \\
21,85--21,924     & Fast radio                       &                                 \\
21,924--22        & Luftfartsradio                   & RR AP27 gäller i del av bandet  \\
22--22,855        & Militär användning               & RR AP25 gäller i del av bandet  \\
22--22,855        & Sjöfartsradio                    & RR AP25 gäller i del av bandet  \\
22,855--24,89     & Militär användning               &                                 \\
22,855--24,89     & Fast radio                       &                                 \\
23--23,2          & Sjöfartsradio                    &                                 \\
23--23,2          & Landmobil radio                  &                                 \\
23,2--23,35       & Luftfartsradio                   &                                 \\
23,35--24         & Sjöfartsradio                    &                                 \\
23,35--24         & Landmobil radio                  &                                 \\
24,89--24,99      & \textbf{Amatörradio}             & \textbf{12-metersbandet}        \\
24,99--25,01      & Standardfrekvens och tidssignal  &                                 \\
25,01--25,55      & Militär användning               &                                 \\
25,01--25,07      & Fast radio                       &                                 \\
25,01--25,07      & Sjöfartsradio                    &                                 \\
25,01--25,07      & Landmobil radio                  &                                 \\
25,07--25,21      & Sjöfartsradio (GMDSS)            & RR AP25 gäller i del av bandet  \\
25,21--25,55      & Fast radio                       &                                 \\
25,21--25,55      & Sjöfartsradio                    &                                 \\
25,21--25,55      & Landmobil radio                  &                                 \\
25,55--25,67      & Radioastronomi                   & Onsala rymd - observatorium     \\
25,67--26,1       & Rundradio                        &                                 \\
26,1--26,175      & Militär användning               & RR AP25 gäller i del av bandet  \\
26,1--26,175      & Sjöfartsradio                    & RR AP25 gäller i del av bandet  \\
26,175--28        & Militär användning               &                                 \\
26,175--27,5      & Fast radio                       &                                 \\
26,175--28        & Landmobil radio                  &                                 \\
26,175--27,5      & Personsökning                    &                                 \\
26,82--27,2       & Telemetri och radiostyrning      & Undantag från tillståndsplikt   \\
26,85--26,86      & Larmöverföring                   & Undantag från tillståndsplikt   \\
26,957--27,283    & Induktiva tillämpningar          & Undantag från tillståndsplikt   \\
                  &                                  & 2006/771/EG 2013/752/EU         \\
26,957--27,2      & Allmän kortdistansradio          & Undantag från tillståndsplikt   \\
                  &                                  & 2006/771/EG 2013/752/EU         \\
26,957--27,283    & ISM                              &                                 \\
26,96--27,41      & \textbf{Privatradio (CB 27 MHz)} & Undantag från tillståndsplikt   \\
26,99--27,2       & Trådlösa barnvaktssystem         & Undantag från tillståndsplikt   \\
28--29,7          & \textbf{Amatörradio}             & \textbf{10-metersbandet}        \\
\end{longtable}
\normalsize
\end{landscape}
