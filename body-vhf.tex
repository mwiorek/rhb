\section{Frekvenser VHF--UHF}

\subsection{Frekvenser ej amatörradio}

Dessa frekvenser är avsedda för allmänhet eller för specifika ända\-mål som anges. Det innebär att de kan brukas för de ändamål som anges i PTS för\-fatt\-nings\-sam\-ling\-ar och sammanställning över ej tillståndspliktiga frekvenser. Observera att du är skyldig att själv kontrollera bestämmelserna
innan en frekvens brukas.

Effekten i tabellen är ustrålad effekt PEP om inte annat anges.

\subsubsection{Jaktfrekvenser 31\,MHz}

Frekvenserna på detta band var tidigare till för enbart jakt. I dag är de öppna för övrig landmobil trafik och kan nyttjas till fritidskommunikation av annat slag.

Högsta effekt är 5\,W och maximal sändningscykel är 10\% vilket betyder att under en timme får man sända maximalt 6 minuter. 

I oktober 2012 utökades de gamla jaktkanalerna med ett antal nya kanaler vilket skedde i oktober 2012. De har ingen officiell kanalnumrering eller egentlig benämning men jag har valt att numrera upp dem efter de traditionella numren med början på 25.

Kanal 24 har dock tidigare haft en frekvens som inte längre är i bruk, så det vore förvirrande att använda den -- den saknas därför i listan. 
Nya kanaler är markerade i listan med asterisk och har fått nummer från kanal 25 och uppåt efter frekvens. Detta gör att listan blir en smula oordnad.

\begin{longtable}{rll|rll}
	\caption{Jaktfrekvenser 31\,MHz tabell}\\
	\textbf{Frekvens} & \textbf{Benämning} & \textbf{Tidigare} & \textbf{Frekvens} & \textbf{Benämning} & \textbf{Tidigare} \\ \hline
		\endfirsthead
	\textbf{Frekvens} & \textbf{Benämning} & \textbf{Tidigare} & \textbf{Frekvens} & \textbf{Benämning} & \textbf{Tidigare} \\ \hline
	\endhead
	           30,930 & Jakt 1             &                   &   31,180          &   Jakt 14          &                   \\
	           30,940 & Jakt 25*           &                   &   31,190          &   Jakt 15          &                   \\
	           30,950 & Jakt 26*           &                   &   31,200          &   Jakt 16          &                   \\
	           30,960 & Jakt 27*           &                   &   31,210        &     Jakt 17        &                   \\
	           30,970 & Jakt 28*           &                   &   31,220          &   Jakt 18          &                   \\
	           31,030 & Jakt 29*           &                   &   31,230        &     Jakt 32*       &                   \\
	           31,040 & Jakt 2             &                   &   31,240         &    Jakt 33*        &                   \\
	           31,050 & Jakt 3             & Kanal 1 Eller D   &   31,250          &   Jakt 19          &  Kanal 4 eller E  \\
	           31,060 & Jakt 4             & Kanal 2 Eller A   &   31,260          &   Jakt 20         &   Kanal 5 eller C \\
	           31,070 & Jakt 5             &                   &   31,270          &   Jakt 21          &                   \\
	           31,080 & Jakt 6             &                   &   31,280          &   Jakt 34*         &                   \\
	           31,090 & Jakt 7             &                   &   31,290          &   Jakt 35*         &                   \\
	           31,100 & Jakt 8             &                   &   31,300          &   Jakt 36*         &                   \\
	           31,110 & Jakt 9             &                   &   31,310          &   Jakt 37*         &                   \\
	           31,120 & Jakt 10            &                   &   31,320          &   Jakt 22          & Kanal 6 eller F   \\
	           31,130 & Jakt 30*           &                   &   31,330          &   Jakt 23          &                   \\
	           31,140 & Jakt 11            &                   &   31,340          &   Jakt 38*         &                   \\
	           31,150 & Jakt 12            &                   &   31,350          &   Jakt 39*        &                   \\
	           31,160 & Jakt 13            & Kanal 3 Eller B   &   31,360          &   Jakt 40*         &                   \\
	           31,170 & Jakt 31*           &                   &   31,370          &                    &
\end{longtable}

\subsubsection{PR-bandet 69 MHz}

Sedan några år tillbaka finns nu ett nytt band som kan användas för
privatradio (PR). Bandet kallas allmänt för 69\,MHz-bandet och har
blivit mycket populärt på sina ställen.

Anledningen är bland annat en stor tillgång på FM-radio för bandet
från gamla åkeriradio som säljs för billiga pengar på diverse
begagnatsajter och som därmed gör det enkelt att komma igång.

Antennstorlekarna är moderata och det är ett ypperligt band för mobilradio där våglängden är ungefär den dubbla mot 2-metersbandet och fungerar bra i många sammanhang.

Nackdelen som den delar med 27\,MHz är att många antenner för fordon är förkortade vilket minskar verkningsgraden på dessa en del men trots detta fungerar det bra. Antennerna är dock betydligt mindre skrymmande än de för 27\,MHz.

På bandet kör man FM uteslutande och det rekommenderas att man skaffar en radio med signalstyrkemätare då man på FM inte kan höra lika väl om man är störd, däremot syns det ju på S-metern om man har störningar. Bandet lider något av störningar i urbana miljöer men på landsbygden brukar det vara tyst och fint.

Användningen av bandet regleras i PTS föreskrift Undantag från Tillståndsplikt och innebär att man får använda max 25\,W ERP (dvs för en dipolantenn), max 10\% sändningscykel (dvs 6 min/timme), en kanalbredd om 25\,kHz och det finns 8 stycken kanaler upplåtna för landmobil radio. I strikt mening är inte kommunikation bas-bas egentligen tillåten eftersom det är landmobil trafik som avses i PTS bestämmelser. Kanal 1 får enbart användas för mobil-mobil trafik inom Västra Götaland och Hallands län.

\begin{table}[h]
  \centering
\begin{tabular}{rrl}
  Kanal & Frekvens & Noteringar                         \\ \hline
  1     & 69,0125  & End. mobil i V. Götaland o Halland \\
  2     & 69,0375  &                                    \\
  3     & 69,0625  &                                    \\
  4     & 69,0875  &                                    \\
  5     & 69,1125  &                                    \\
  6     & 69,1375  &                                    \\
  7     & 69,1625  &                                    \\
  8     & 69,1875  & Anv. som anropskanal               \\
\end{tabular}
\caption{Frekvenser 69 MHz}
\end{table}




\subsubsection{Jakt och jordbruksfrekvenser 155\,MHz}

Observera att kanalnumren som är traditionella och frekvenserna inte kommer helt i ordning. Fyra kanaler är markerade med $^R$ och har särskilda restriktioner på svenskt innanvatten och territorialvatten.

\begin{table}[H]
\centering
\begin{tabular}{rlrl}
	\textbf{Frekvens} & \textbf{Benämning} & \textbf{Effekt} & \textbf{Användningsområde}           \\ \hline
	          155,400 & Jakt K6            &             5 W & Jakt, Jordbruk, Skogsbruk$^R$        \\
	          155,425 & Jakt K1            &             5 W & Jakt, Jordbruk$^R$                   \\
	          155,450 & Jakt K7            &             5 W & Jakt, Jordbruk, Skogsbruk$^R$        \\
	          155,475 & Jakt K2            &             5 W & Jakt, Jordbruk$^R$                   \\
	          155,500 & Jakt K3 VHF-M L1   &             5 W & Jakt, Jordbruk, Skogsbruk, Marin$^M$ \\
	          155,525 & Jakt K4 VHF-M L2   &             5 W & Jakt, Jordbruk, Skogsbruk Marin$^M$  \\
	          156,000 & Jakt K5            &             5 W & Jakt, PMR, Friluftskanal$^P$
\end{tabular}
\caption{Jakt- och jordbruksfrekvenser 155\,MHz}
\end{table}

\footnotesize
\begin{itemize}
	\item[$^M$] Delas med marina VHF-bandet, kanalerna L1 och L2 för fritidsbåtar.
	\item[$^P$] PMR-kanal som kan användas till allmän privatradio.
	\item[$^R$] Dessa kanaler \textit{får ej användas} på svenskt territorialvatten eller svenskt inre vatten. Se \href{https://pts.se/globalassets/startpage/dokument/legala-dokument/foreskrifter/radio/beslutade_ptsfs-2018-3-undantagsforeskrifter.pdf}{PTSFS2018:3} för mer information.
\end{itemize}
\normalsize

\subsubsection{Öppna PMR-bandet på 446\,MHz}

I nya författningssamlingen står det uttryckligen att repeateranvändning är 
förbjuden. De exakta kanalerna har också inte heller bestämts utan bandet är 
upplåtet 446,0--446,2\,MHz. Traditionellt används nedanstående kanaler. Max 
effekt är 500\,mW och antennen får ej vara av löstagbar sort. Utrustningen 
skall vara godkänd för ändamålet.

Sedan sist har ytterligare spektrum tillförts och bandet har nu 16 kanaler. Det medges också digital PMR på alla frekvenserna men rekommendationen är att använda K1--K8 för analogt och K9--K16 för digitalt eftersom äldre apparater inte kan gå på de nya kanalerna medan alla digitala kan det.

Vi vissa numreringar numreras de digitala kanalerna med start på kanalnummer 1 på K9. I listan står de som D1--D8 där D står för digitalt.

Endast smalbandig modulation med FM-deviation max 2.5 kHz skall användas för att inte störa närliggande kanaler. Kanalrastret är 12,5\,kHz så modulationen bör rymmas inom den bandbredden.
 
\begin{table}[H]
\centering
\begin{tabular}{rll|rll}
	\textbf{Frekvens} & \textbf{Benämning} & \textbf{Rek. Anv.}& 
	\textbf{Frekvens} & \textbf{Benämning} & \textbf{Rek. Anv.}      \\ \hline
	446,00625 & PMR446 K1          & PMR                                   &          446,10625 & PMR446 K9\ \,\,/D1       & DPMR \\
	446,01875 & PMR446 K2          & PMR                                   &          446,11875 & PMR446 K10/D2      & DPMR \\
	446,03125 & PMR446 K3          & PMR                                   &          446,13125 & PMR446 K11/D3      & DPMR \\
	446,04375 & PMR446 K4          & PMR                                   &          446,14375 & PMR446 K12/D4      & DPMR \\
	446,05625 & PMR446 K5          & PMR                                   &          446,15625 & PMR446 K13/D5      & DPMR \\
	446,06875 & PMR446 K6          & PMR                                   &          446,16875 & PMR446 K14/D6      & DPMR \\
	446,08125 & PMR446 K7          & PMR                                   &          446,18125 & PMR446 K15/D7      & DPMR \\
	446,09375 & PMR446 K8          & PMR                                   &          446,19375 & PMR446 K16/D8      & DPMR
\end{tabular}
\caption{PMR-frekvenser}
\label{tab:pmr-frekvenser}
\end{table}

\subsubsection{Kortdistansradio (KDR, SRBR)}

Kallas även SRBR för Short Range Business Radio. 
Den traditionella frekvenslistan ser ut som följer. En ny variant med frekvenser för 12,5\,kHz samt 6,25\,kHz kanaler finns också ute nu och kan ses i tabell \ref{tab:SRBR-frekvenser}.

\begin{table}[h]
	\centering
\begin{tabular}{rlrl}
\textbf{Frekvens} & \textbf{Benämning} & \textbf{Effekt} & \textbf{Användningsområde} \\ \hline
444,600 & SRBR K1            & 2 W             & Short range business radio \\
444,625 & SRBR K2            & 2 W             & Short range business radio \\
444,800 & SRBR K3            & 2 W             & Short range business radio \\
444,825 & SRBR K4            & 2 W             & Short range business radio \\
444,850 & SRBR K5            & 2 W             & Short range business radio \\
444,875 & SRBR K6            & 2 W             & Short range business radio \\
444,925 & SRBR K7            & 2 W             & Short range business radio \\
444,975 & SRBR K8            & 2 W             & Short range business radio
\end{tabular}
\caption{Frekvenser för SRBR}
\end{table}	

SRBR är ett ej tillståndspliktigt frekvenssegment som används för yrkesmässig radiotrafik.

Rekommendationen är att man skall använda CTCSS eller motsvarande för att undvika störa och bli störd av andra stationer som delar frekvenserna.

Från PTSFS2018:3 så har bandet fått nya bärvågsfrekvenser och det har blivit öppet för att köra med 25, 12,5 eller 6,25\,kHz Kanalraster. Denna frekvenstabell blir lite mer komplicerad.

% Please add the following required packages to your document preamble:
% \usepackage{multirow}
\begin{table}[h]
	\centering
	\begin{tabular}{|l|l|l|l|l|l|}
		\hline
		\textbf{25 kHz}                                & \textbf{12,5 kHz}                               & \textbf{6,25 kHz}               & \textbf{25 kHz}                               & \textbf{12,5 kHz}                               & \textbf{6,25 kHz}               \\ \hline
		\multicolumn{1}{|c|}{\multirow{4}{*}{444,600}} & \multicolumn{1}{c|}{\multirow{2}{*}{444,59375}} & \multicolumn{1}{c|}{444,590625} & \multicolumn{1}{l|}{\multirow{4}{*}{444,850}} & \multicolumn{1}{l|}{\multirow{2}{*}{444,84375}} & \multicolumn{1}{l|}{444,840625} \\ \cline{3-3}\cline{6-6}
		\multicolumn{1}{|c|}{}                         & \multicolumn{1}{c|}{}                           & \multicolumn{1}{c|}{444,596875} & \multicolumn{1}{l|}{}                         & \multicolumn{1}{l|}{}                           & \multicolumn{1}{l|}{444,846875} \\ \cline{2-3}\cline{5-6}
		\multicolumn{1}{|c|}{}                         & \multicolumn{1}{c|}{\multirow{2}{*}{444,60625}} & \multicolumn{1}{c|}{444,603125} & \multicolumn{1}{l|}{}                         & \multicolumn{1}{l|}{\multirow{2}{*}{444,85625}} & \multicolumn{1}{l|}{444,853125} \\ \cline{3-3}\cline{6-6}
		\multicolumn{1}{|c|}{}                         & \multicolumn{1}{c|}{}                           & \multicolumn{1}{c|}{444,609375} & \multicolumn{1}{l|}{}                         & \multicolumn{1}{l|}{}                           & \multicolumn{1}{l|}{444,859375} \\ \hline
		\multirow{4}{*}{444,650}                       & \multirow{2}{*}{444,64375}                      & 444,640625                      & \multicolumn{1}{l|}{\multirow{4}{*}{444,875}} & \multicolumn{1}{l|}{\multirow{2}{*}{444,86875}} & \multicolumn{1}{l|}{444,865625} \\ \cline{3-3}\cline{6-6}
		                                               &                                                 & 444,646875                      & \multicolumn{1}{l|}{}                         & \multicolumn{1}{l|}{}                           & \multicolumn{1}{l|}{444,871875} \\ \cline{2-3}\cline{5-6}
		                                               & \multirow{2}{*}{444,65625}                      & 444,653125                      & \multicolumn{1}{l|}{}                         & \multicolumn{1}{l|}{\multirow{2}{*}{444,88125}} & \multicolumn{1}{l|}{444,878125} \\ \cline{3-3}\cline{6-6}
		                                               &                                                 & 444,659375                      & \multicolumn{1}{l|}{}                         & \multicolumn{1}{l|}{}                           & \multicolumn{1}{l|}{444,884375} \\ \hline
		\multirow{4}{*}{Saknas}                        & \multirow{2}{*}{444,66875}                      & 444,665625                      & \multicolumn{1}{l|}{\multirow{4}{*}{444,925}} & \multicolumn{1}{l|}{\multirow{2}{*}{444,91875}} & \multicolumn{1}{l|}{444,915625} \\ \cline{3-3}\cline{6-6}
		                                               &                                                 & 444,671875                      & \multicolumn{1}{l|}{}                         & \multicolumn{1}{l|}{}                           & \multicolumn{1}{l|}{444,921875} \\ \cline{2-3}\cline{5-6}
		                                               & \multirow{2}{*}{444,68125}                      & 444,678125                      & \multicolumn{1}{l|}{}                         & \multicolumn{1}{l|}{\multirow{2}{*}{444,93125}} & \multicolumn{1}{l|}{444,928125} \\ \cline{3-3}\cline{6-6}
		                                               &                                                 & 444,684375                      & \multicolumn{1}{l|}{}                         & \multicolumn{1}{l|}{}                           & \multicolumn{1}{l|}{444,934375} \\ \hline
		\multirow{4}{*}{444,800}                       & \multirow{2}{*}{444,79375}                      & 444,790625                      & \multicolumn{1}{l|}{\multirow{4}{*}{444,975}} & \multicolumn{1}{l|}{\multirow{2}{*}{444,91875}} & \multicolumn{1}{l|}{444,915625} \\ \cline{3-3}\cline{6-6}
		                                               &                                                 & 444,796875                      & \multicolumn{1}{l|}{}                         & \multicolumn{1}{l|}{}                           & \multicolumn{1}{l|}{444,921875} \\ \cline{2-3}\cline{5-6}
		                                               & \multirow{2}{*}{444,80625}                      & 444,803125                      & \multicolumn{1}{l|}{}                         & \multicolumn{1}{l|}{\multirow{2}{*}{444,93125}} & \multicolumn{1}{l|}{444,928125} \\ \cline{3-3}\cline{6-6}
		                                               &                                                 & 444,809375                      & \multicolumn{1}{l|}{}                         & \multicolumn{1}{l|}{}                           & \multicolumn{1}{l|}{444,934375} \\ \hline
		\multirow{4}{*}{444,825}                       & \multirow{2}{*}{444,81875}                      & 444,815625                      & \multicolumn{3}{l}{\multirow{4}{*}{}}                                                                                             \\ \cline{3-3}
		                                               &                                                 & 444,821875                      & \multicolumn{3}{l}{}                                                                                                              \\ \cline{2-3}
		                                               & \multirow{2}{*}{444,83125}                      & 444,828125                      & \multicolumn{3}{l}{}                                                                                                              \\ \cline{3-3}
		                                               &                                                 & 444,834375                      & \multicolumn{3}{l}{}                                                                                                              \\ \cline{1-3}
	\end{tabular}
\caption{Nya frekvensindelningen på kortdistansradiobandet}
\label{tab:SRBR-frekvenser}
\end{table}

\clearpage

\subsection{Maritima VHF-frekvenser}

Marinbandet på VHF består både av duplex- och simplexkanaler. Simplexkanalerna används skepp-till-skepp och även ibland mot kustradio. Duplexfrekvenserna används t.ex. vid telefonsamtal som sätts upp av kuststation till skepp eller liknande. På dessa arbetskanaler sänder man även ut sjörapporter, navigationsvarningar och annan information t.ex. säkerhetsvarningar som är viktiga för sjöfarten.

\subsubsection{Kanalnummer och frekvens maritima kanaler}

\begin{table}[H]
\centering
\begin{tabular}{rrr|rrr}
\textbf{Kanal} & \textbf{Skepp} & \textbf{Kust} & \textbf{Kanal} & \textbf{Skepp} & \textbf{Kust} \\ \hline
01 & 156,050 & 160,650 & 60 & 156,025 & 160,625 \\
02 & 156,100 & 160,700 & 61 & 156,075 & 160,675 \\
03 & 156,150 & 160,750 & 62 & 156,125 & 160,725 \\
04 & 156,200 & 160,800 & 63 & 156,175 & 160,775 \\
05 & 156,250 & 160,850 & 64 & 156,225 & 160,825 \\
06 & 156,300 &         & 65 & 156,275 & 160,875 \\
07 & 156,350 & 160,950 & 66 & 156,325 & 160,925 \\
08 & 156,400 &         & 67 & 156,375 & \\
09 & 156,450 &         & 68 & 156,425 & \\
10 & 156,500 &         & 69 & 156,475 & \\
11 & 156,550 &         & 70 & 156,525 & DSC \\
12 & 156,600 &         & 71 & 156,575 & \\
13 & 156,650 &         & 72 & 156,625 & \\
14 & 156,700 &         & 73 & 156,675 & \\
15 & 156,750 &         & 74 & 156,725 & \\
16 & 156,800 & Anrop/Nöd   & 75 & 156,775 & \\
17 & 156,850 &         & 76 & 156,825 & \\
18 & 156,900 & 161,500 & 77 & 156,875 & \\
19 & 156,950 & 161,550 & 78 & 156,925 & 161,525 \\
20 & 157,000 & 161,600 & 79 & 156,975 & 161,575 \\
21 & 157,050 & 161,650 & 80 & 157,025 & 161,625 \\
22 & 157,100 & 161,700 & 81 & 157,075 & 161,675 \\
23 & 157,150 & 161,750 & 82 & 157,125 & 161,725 \\
24 & 157,200 & 161,800 & 83 & 157,175 & 161,775 \\
25 & 157,250 & 161,850 & 84 & 157,225 & 161,825 \\
26 & 157,300 & 161,950 & 85 & 157,325 & 161,925 \\
27 & 157,350 & 161,950 & 86 & 157,325 & 161,925 \\
28 & 157,400 & 162,000 & 87 & 157,375 & \\
   &         &         & 88 & 157,425 & \\
   &         &         &    &         & \\
L1 & 155,500 & Leisure        & F1 & 155,625 &Fishing \\
L2 & 155,525 & Leisure        & F2 & 155,775 &Fishing \\
   &         &         & F3 & 155,825 & Fishing\\
\end{tabular}
\caption{Marin VHF, frekvenslista}
\end{table}

I tabellen listas de kanaler som gäller i svenska farvatten. Andra länder kan ha andra kanaler eller för olika ändamål. Det krävs en särskild licens från PTS för att få nyttja dessa frekvenser och radiooperatören skall ha ett SRC-certifikat (Short Range Communication).

Anropskanal och nödkanal är kanal 16.

Vid duplextrafik är skiftet -4,6\,MHz.

I tabellen är kanaler som saknar kustfrekvens alltså simplexkanaler. DSC står för ''Digital Selective Call'' ett sätt att digitalt anropa skepp eller kuststationer, kanaler vikta för DSC får inte användas för vanliga samtal.

Kanal 16 är anropsfrekvens om man inte vet motstationen passar en annan kanal. Den är också nödfrekvens eftersom den passas av de flesta. 

Kanalerna L1--L2 är frekvenser avsedda för fritidsbåtar (Leisure) och frekvenserna F1--F3 osv är avsedda för yrkesfiske. L1 och L2 delas med kanal 3 och 4 på jaktradion vilket kan vara bra att känna till.

\subsubsection{Transponderkanaler}
\begin{longtable}{rrl}
	\textbf{Kanal} & \textbf{Skepp} & \textbf{Not} \\ \hline
	   \endhead
AIS1 & 161,975 & Digital trafik, transponder\\
AIS2 & 162,025 & Digital trafik, transponder\\
\end{longtable}

\subsubsection{Stockholm radio}

Radiohorisonten är beräknad i nautiska mil, skeppet lägger till sin egen radiohorisont för att bestämma om det går att nå kuststationen eller ej.

\textbf{Ostkusten}

\begin{longtable}{lrr|lrr}
\textbf{Kuststation} & \textbf{Kanal} & \textbf{Horisont} & \textbf{Kuststation} & \textbf{Kanal}& \textbf{Horisont}\\
\hline
\endhead
Kalix          & 25 & 39 & Luleå         & 24 & 26 \\
Skellefteå     & 23 & 44 & Umeå          & 26 & 54 \\
Örnsköldsvik   & 28 & 42 & Mjällom       & 64 & 43 \\
Kramfors       & 84 & 43 & Härnösand     & 23 & 36 \\
Sundsvall      & 24 & 36 & Hudiksvall    & 25 & 54 \\
Gävle          & 23 & 37 & Östhammar     & 24 & 44 \\
Väddö          & 78 & 32 & Nacka         & 26, 23* & 43 \\
Sv. högarna    & 84 & 15 & Södertälje    & 66 & 30 \\
Torö           & 24 & 26 & Gotska sandön & 65 & 22 \\
Norrköping     & 64 & 43 & Västervik     & 23 & 45 \\
Fårö           & 28 & 25 & Visby         & 25 & 23 \\
Hoburgen       & 24 & 25 & Kalmar        & 26 & 40 \\
Ölands s. udde & 78 & 23 & Karlskrona    & 81 & 24 \\
Karlshamn      & 25 & 48 & Kivik         & 21 & 39\\
\end{longtable}
*) Sänder ej väder, varningar eller andra listor

\clearpage
\textbf{Västkusten}

\begin{longtable}{lrr|lrr}
\textbf{Kuststation} & \textbf{Kanal} & \textbf{Horisont} & \textbf{Kuststation} & \textbf{Kanal} & \textbf{Horisont} \\
\hline
\endhead

Strömstad   & 22 & 25 & Grebbestad & 26 & 25 \\
Kungshamn   & 23 & 23 & Uddevalla  & 84 & 47 \\
Tjörn       & 81 & 26 & Göteborg   & 24 & 43 \\
Grimeton    & 22 & 35 & Halmstad   & 62 & 52 \\
Helsingborg & 24 & 28 & Malmö      & 27 & 25 \\
\end{longtable}

\textbf{Insjöarna}

\begin{longtable}{lrr|lrr}
\textbf{Kuststation} & \textbf{Kanal} & \textbf{Horisont} & \textbf{Kuststation} & \textbf{Kanal} & \textbf{Horisont} \\
\hline
\endhead

Västerås  & 25 & 40 & Trollhättan & 25 & 32 \\
Bäckefors & 78 & 50 & Kinnekulle  & 01 & 43 \\
Karlstad  & 65 & 36 & Jönköping   & 23 & 49 \\
Motala    & 26 & 47 &             &    &    \\
\end{longtable}

\subsection{Frekvenser amatörradio VHF--UHF}

I denna skrift försöker vi omfatta de viktigaste VHF och UHF-banden för amatörradio vilket inkluderar 6\,m-bandet, 2\,m-bandet, 70\,cm-bandet och 23\,cm-bandet.

\subsubsection{Kanalnumrering VHF/UHF}

Denna typ av kanalnumrering är överenskommen inom IARU region 1 för 6\,m, 2\,m och 70\,cm banden på amatörradiofrekvenser. Kanalnumreringen består av ett prefix som anger vilket band och här används F--6\,m, V--2\,m, U--70\,cm. Därefter används 2 siffror på 6m och 2m banden och tre siffror på 70cm bandet för
att ange kanal.

Repeaterfrekvenser anges med tillägget R före kanalnumret och innebär då normalt duplex med det skift som normalt används för bandet. Vid repeatrar är det repeaterns utfrekvens som anges, dvs den som mobilstationen lyssnar på. Exempel: RV48.

\begin{tabular}{crrll}
	\textbf{Band} & \textbf{Startfrekvens} & \textbf{Kanalraster} & \textbf{Första kanal} & \textbf{Beräknas}    \\ \hline

6\,m   & 51.000\,MHz  & 10.0\,kHz & F00  & $f=51+k\cdot0.01$    \\
      &             &          &      & $k=(f-51)/0,01$      \\ \hline
2\,m   & 145.000\,MHz & 12.5\,kHz & V00  & $f=145+k\cdot0.0125$ \\
      &             &          &      & $k=(f-145)/0,0125$   \\ \hline
70\,cm & 430.000\,MHz & 12.5\,kHz & U000 & $f=430+k\cdot0.0125$ \\
      &             &          &      & $k=(f-430)/0,0125$   \\ \hline
\end{tabular}

Eftersom amatörradiobanden ser lite olika ut i olika länder förekommer det kanaler i numreringen som inte är tillåtna på vissa ställen. Det är därför viktig att kontrollera att man fortfarande följer bandplanerna i den region man är.

\begin{itemize}
\item I 6\,m bandet finns inga FM-kanaler definierade under 51\,MHz.
\item För 2\,m-bandet är FM-kanaler endast definierade från 145\,MHz och uppåt.
\item I 70\,cm-bandet är inga kanaler definierade i intervallet 432.000--433.000\,MHz. Observera att startfrekvensen är utanför 70\,cm bandplanen i IARU region 1.
\end{itemize}

\textit{OBS!\\ Information om kanalnumreringen för 23\,cm-bandet tas tacksamt mot. Maila mig på anders@sikvall.se om du har korrekt information.}

\subsubsection{Införande av 12.5~kHz kanalavstånd}

För ett antal år sedan beslutade man sig att gå mot ett smalare kanalraster på VHF och UHF och införde härmed kanalavstånd på 12.5~kHz i bandplanerna. Ustrustning med 25~kHz kanalraster är fortsatt tillåten och detta är en rekommendation. Vid införandet av detta så kom även ett nytt numreringsalternativ för kanalsystemen baserat på en basfrekvens (som ibland på svenska band ligger utanför vårt band) och därefter numrerade man i ordning för respektive 12.5~kHz steg och 10~kHz för kortvåg.

\begin{table}[h]
\centering
\begin{tabular}{rrrr}
Kod & Basfrekvens & Kanalavstånd & Repeaterskift \\
    & [MHz]       & [kHz]        & [kHz] \\ \hline
H & 29,500 & 10,0 & -100 \\
F & 51,000 & 10,0 & -600 \\
V & 145,000& 12,5 & -600 \\
U & 430,000& 12,5 & -2000 \\
M & 1240,000 & 25 & -6000 \\
\end{tabular}
\label{tab:kanalavstand}
\caption{Kanalavstånd och beteckning olika frekvensband}
\end{table}

\subsubsection{FM-kanaler 6m-bandet}

\begin{longtable}{rrl|rrl}
\textbf{Kanal} & \textbf{Tidigare} & \textbf{Anm}   
&  \textbf{Kanal} & \textbf{Tidigare} & \textbf{Anm} \\ \hline
	51,500 &      F50 &       & 51,750 &      F75 &  \\
	51,510 &      F51 & Anrop & 51,760 &      F76 &  \\
	51,520 &      F52 &       & 51,770 &      F77 &  \\
	51,530 &      F53 &       & 51,780 &      F78 &  \\
	51,540 &      F54 &       & 51,790 &      F79 &  \\
	51,550 &      F55 &       & 51,800 &      F80 &  \\
	51,560 &      F56 &       & 51,810 &     RF81 &  \\
	51,570 &      F57 &       & 51,820 &     RF82 &  \\
	51,580 &      F58 &       & 51,830 &     RF83 &  \\
	51,590 &      F59 &       & 51,840 &     RF84 &  \\
	51,600 &      F60 &       & 51,850 &     RF85 &  \\
	51,610 &      F61 &       & 51,860 &     RF86 &  \\
	51,620 &      F62 &       & 51,870 &     RF87 &  \\
	51,630 &      F63 &       & 51,880 &     RF88 &  \\
	51,640 &      F64 &       & 51,890 &     RF89 &  \\
	51,650 &      F65 &       & 51,900 &     RF90 &  \\
	51,660 &      F66 &       & 51,910 &     RF91 &  \\
	51,670 &      F67 &       & 51,920 &     RF92 &  \\
	51,680 &      F68 &       & 51,930 &     RF93 &  \\
	51,690 &      F69 &       & 51,940 &     RF94 &  \\
	51,700 &      F70 &       & 51,950 &     RF95 &  \\
	51,710 &      F71 &       & 51,960 &     RF96 &  \\
	51,720 &      F72 &       & 51,970 &     RF97 &  \\
	51,730 &      F73 &       & 51,980 &     RF98 &  \\
	51,740 &      F74 &       & 51,990 &     RF99 &
\end{longtable}

\clearpage
\subsubsection{FM-kanaler 2m-bandet}

\begin{longtable}{rrl|rrl}

\textbf{Frekvens} & \textbf{Kanal} & \textbf{Anm} & 
\textbf{Frekvens} & \textbf{Kanal} & \textbf{Anm} \\ \hline

145,2125 & V17 &              & 145,5000 & V40  & S20  FM Anrop \\
145,2250 & V18 & S9           & 145,5125 & V41  &               \\
145,2375 & V19 & INET GW      & 145,5250 & V42  & S21           \\
145,2500 & V20 & S10          & 145,5375 & V43  &               \\
145,2625 & V21 &              & 145,5500 & V44  & S22           \\
145,2750 & V22 & S11          & 145,5625 & V45  &               \\
145,2875 & V23 & INET GW      & 145,5750 & V46  & S23           \\
145,3000 & V24 & S12  RTTY    & 145,5875 & V47  &               \\
145,3125 & V25 &              & 145,6000 & RV48 & R0            \\
145,3250 & V26 & S13          & 145,6125 & RV49 & R0X           \\
145,3375 & V27 & INET GW      & 145,6250 & RV50 & R1            \\
145,3500 & V28 & S14          & 145,6375 & RV51 & R1X           \\
145,3625 & V29 &              & 145,6500 & RV52 & R2            \\
145,3750 & V30 & S15 DV Anrop & 145,6625 & RV53 & R2X           \\
145,3875 & V31 &              & 145,6750 & RV54 & R3            \\
145,4000 & V32 & S16          & 145,6875 & RV55 & R3X           \\
145,4125 & V33 &              & 145,7000 & RV56 & R4            \\
145,4250 & V34 & S17 Scout    & 145,7125 & RV57 & R4X           \\
145,4375 & V35 &              & 145,7250 & RV58 & R5            \\
145,4500 & V36 & S18          & 145,7375 & RV59 & R5X           \\
145,4625 & V37 &              & 145,7500 & RV60 & R6            \\
145,4750 & V38 & S19          & 145,7625 & RV61 & R6X           \\
145,4875 & V39 &              & 145,7750 & RV62 & R7            \\
         &     &              & 145,7875 & RV63 & R7X
\end{longtable}

X-kanalerna uppstod när man fick platsbrist och man övergick till en 12.5\,kHz kanaldelning för repeatrar. Först senare övergick man även till samma kanaldelning på övriga FM-kanaler. De gamla simplexkanalerna hade inte så stor spridning i Sverige men förekom rikligt t.ex. i Tyskland med S20 som anropsfrekvens (eller aktivitetscenter som det numera kallas).

\clearpage
\subsubsection{FM-kanaler 70cm-bandet}

\begin{longtable}{rrl|rrl}
\textbf{Frekvens} & \textbf{Kanal} & \textbf{Anm} &  
\textbf{Frekvens} & \textbf{Kanal} & \textbf{Anm} \\ \hline

433,4000 & U272 & SSTV    & 433,7125 & U297 &      \\
433,4125 & U273 &         & 433,7250 & U298 &      \\
433,4250 & U274 &         & 433,7375 & U299 &      \\
433,4375 & U275 &         & 433,7500 & U300 &      \\
433,4500 & U276 & Digital & 433,7625 & U301 &      \\
433,4625 & U277 &         & 433,7750 & U302 &      \\
433,4750 & U278 &         & 433,7875 & U303 &      \\
433,4875 & U279 &         & 433,8000 & U304 & APRS \\
433,5000 & U280 & Anrop   & 433,8125 & U305 &      \\
433,5125 & U281 &         & 433,8250 & U306 &      \\
433,5250 & U282 &         & 433,8375 & U307 &      \\
433,5375 & U283 &         & 433,8500 & U308 &      \\
433,5500 & U284 &         & 433,8625 & U309 &      \\
433,5625 & U285 &         & 433,8750 & U310 &      \\
433,5750 & U286 &         & 433,8875 & U311 &      \\
433,5875 & U287 &         & 433,9000 & U312 &      \\
433,6000 & U288 & RTTY    & 433,9125 & U313 &      \\
433,6125 & U289 &         & 433,9250 & U314 &      \\
433,6250 & U290 &         & 433,9375 & U315 &      \\
433,6375 & U291 &         & 433,9500 & U316 &      \\
433,6500 & U292 &         & 433,9625 & U317 &      \\
433,6625 & U293 &         & 433,9750 & U318 &      \\
433,6750 & U294 &         & 433,9875 & U319 &      \\
433,6875 & U295 &         & 434,0000 & U320 &      \\
433,7000 & U296 & FAX     &          &      &      \\

\end{longtable}

\clearpage
\begin{longtable}{rrl|rrl}
\textbf{Frekvens} & \textbf{Kanal} & \textbf{Anm}   
&  \textbf{Frekvens} & \textbf{Kanal} & \textbf{Anm} \\ \hline

434,6000 & RU368 & RU0  & 434,8000 & RU384 & RU8   \\
434,6125 & RU369 & RU0X & 434,8125 & RU385 & RU8X  \\
434,6250 & RU370 & RU1  & 434,8250 & RU386 & RU9   \\
434,6375 & RU371 & RU1X & 434,8375 & RU387 & RU9X  \\
434,6500 & RU372 & RU2  & 434,8500 & RU388 & RU10  \\
434,6625 & RU373 & RU2X & 434,8625 & RU389 & RU10X \\
434,6750 & RU374 & RU3  & 434,8750 & RU390 & RU11  \\
434,6875 & RU375 & RU3X & 434,8875 & RU391 & RU11X \\
434,7000 & RU376 & RU4  & 434,9000 & RU392 & RU12  \\
434,7125 & RU377 & RU4X & 434,9125 & RU393 & RU12X \\
434,7250 & RU378 & RU5  & 434,9250 & RU394 & RU13  \\
434,7375 & RU379 & RU5X & 434,9375 & RU395 & RU13X \\
434,7500 & RU380 & RU6  & 434,9500 & RU396 & RU14  \\
434,7625 & RU381 & RU6X & 434,9625 & RU397 & RU14X \\
434,7750 & RU382 & RU7  & 434,9750 & RU398 & RU15  \\
434,7875 & RU383 & RU7X & 434,9875 & RU399 & RU15X \\
         &       &      & 435,0000 & RU400 &       \\

\end{longtable}

RU0X osv är här en efterkonstruktion. Egentligen så användes sällan ``X-frekvenserna'' på 70cm eftersom man dels hade nästan dubbla antalet frekvenser för repeatrar och sedan gammalt ville man egentligen inte ha ett smalare kanalraster, i tidernas begynnelse körde många amatörer 70cm genom frekvenstrippling från 2m. $144,000 \cdot 3 = 432,000$\,MHz och $144,025 \cdot 3 = 432,075$\,MHz varför man till och med hade bredare kanalraster de-facto.


\subsection{Scouters frekvenser, JOTA}

Scouter finns ofta QRV under vissa helger, \textit{Jamboree On The Air, JOTA}, förekommer några gånger per år. Här är en sammanställning av de standardfrekvenser scouter nyttjar om de inte kör repeatrar eller leta upp motstationer själva. Scouter kan antingen ha egna signaler, köra under
tillfälliga signaler eller vara second operator åt med någon klubbsignal.

\subsection{Nordiska scoutfrekvenser VHF}

\begin{center}
\begin{tabular}{lrr}
	\textbf{Mode} & \textbf{Frekvens} & \textbf{Kanal} \\ \hline
	FM            &      145.425  MHz &   V34 \\
	SSB           &      144.240  MHz &  \\
	CW            &      144.050  MHz &
\end{tabular}
\end{center}

Jotan hålls alltid den 3:e hela (lördag och söndag) helgen i oktober varje år. Jotan startar officiellt vid invigningen på lördag förmiddag och slutar natten till måndagen klockan 00:00. Många börjar redan på fredagskvällen och avslutar på söndagseftermiddagen.

Sändningar under denna tid förekommer från ocertifierade scouter som lånar 
klubbsignal, har en tillfällig signal utdelad, ibland lånar enskilda 
sändaramatörer ut sina signaler. 

Sändningarna skall dock alltid ske under direkt överinseende av en radioamatör 
men var beredd på att det kommer vara en viss ovana och ske en del misstag. 
Strunta i det och ge scouterna en kul radioupplevelse.

% Radioberäkningar för VHF och UHF

\subsection{Radioberäkningar för VHF och UHF}

\subsubsection{Beräkning av radiohorisonten}

Radiohorisonten är den sträcka som markvågen kan nå utan särskilda hjälpmedel och i frånvaro av andra effekter som särskilda kondisioner (tropo eller duktning) och liknande. Avståndet kan beräknas med hjälp av en enkel formel. Radiohorisonten gäller egentligen bara när inget annat är i vägen men kan ge en ledning till den längsta utbredning man kan förvänta sig med markvåg givet en viss höjd.

För skepp på havet stämmer radiohorisonten ganska väl så man hittar denna formel ofta i utbildningsmaterial för marin VHF men då med distansen i nautiska mil i stället för km. För att få detta byter man konstanten 3,57 till 2,2 i stället.

\begin{equation*}
	r = 3,57 \left(\sqrt{h_1}+\sqrt{h_2}\right)
\end{equation*}

Där $r$ är avståndet till radiohorisonten givet i kilometer, $h_1$ är den ena stationens antennhöjd över marken givet i meter och $h_2$ är den andra stationens antennhöjd över marken också givet i meter.

\subsubsection{Sträckdämpning}

Sträckdämpningen beror på flera olika faktorer, inte minst terrängen och det som finns mellan sändaren och mottagaren. I den fria rymden följer den en enkel geometrisk utbredning men närmare marken behöver man stoppa in en del kompensationsfaktorer.

\begin{equation*}
	PL_0 = 20 \cdot \log(f) + 20 \cdot \log(d) - 27,55
\end{equation*}

Där $PL_{0}$ är sträckdämpningen i decibel(dB) (Eng: Path Loss) mellan två sändare givet avståndet $d$ i meter och frekvensen $f$ i MHz. Om man anger $d$ i kilometer i stället adderar man 60 till konstanten och får då 32,45.

För sträckdämpning vid mark får man mäta eller skatta en utbredningsdämpning som en konstant $k$ som man använder för att modifiera formeln med och får då följande variant:


\begin{equation*}
	PL_m = 20 \cdot \log(f) + (20+k) \cdot \log(d) - 27,55
\end{equation*}

Där $PL_m$ är sträckdämpningen vid marken. Faktorn $k$ kan uppskattas enligt följande tabell:

\begin{table}[h]
	\begin{centering}
		\begin{tabular}{r|l}
			\textbf{k} & \textbf{Beskrivning} \\ \hline
			0 & Över öppen terräng med högre frekvenser och fri sikt\\
			5 & Lättare terräng, mindre kullar, gräs och få träd \\
			10 & Tuffare terräng med mer höjdvariation, klippblock, tätare skog \\
			15 & Urban miljö, större hus, höghus \\
			20 & Extremt urband miljö (tänk Manhattan)\\
		\end{tabular}
	\end{centering}
	\label{tab:frirum-faktor}
	\caption{Tabell över korrigeringsfaktor för frirumsutbredning vid marken}
\end{table}

I vanlig svensk terräng är det nog vanligast man hamnar i storleksordningen 5--10.

\clearpage

\begin{landscape}
\subsection{Repeatrar, länkar och fyrar VHF/UHF}
\subsubsection{Svenska fyrar VHF/UHF}
\begin{longtable}{rlrllrrrc}
\textbf{Signal} & \textbf{QTH}  & \textbf{Frekvens} & \textbf{Status} & \textbf{Lokator} & 
\textbf{Masl}   & \textbf{Magl} & \textbf{Effekt}   & \textbf{Ritktning} \\ \hline

SK4MPI  & Borlänge           & 144.4120  & QRV & JP70NJ & 520 & 10 & 200    & NV+NO    \\
SK7MHH  & Färjestaden        & 432.4400  & QRV & JO86GP & 45  & 15 &        &          \\
SK4BX/B & Garphyttan/Storst. & 432.4600  & QRV & JO79LH & 270 & 10 & 50     & NESW     \\
SK4BX/B & Garp./Ånnaboda     & 1296.9600 & QRV & JO79LI & 270 & 10 &        &          \\
SK4BX/B & Garp./Ånnaboda     & 2320.9600 & QRT & JO79LI & 270 & 10 &        &          \\
SK6MHI  & Göteborg           & 2320.8000 & QRV & JO57XQ & 135 & 40 & 150    & Omni     \\
SK6MHI  & Hönö               & 1296.8000 & QRV & JO57TQ & 40  & 30 & 30     & Omni     \\
SK2VHG  & Kiruna/Esrange     & 144.4350  & QRT & KP07MU & 630 &    &        & S        \\
SK2CP/B & Kiruna/Esrange     & 50.0520   & QRV & KP07MU & 630 &    & 30     & Omni     \\
SK1UHF  & Klintehamn         & 432.4050  & QRV & JO97CJ & 65  & 60 & 30     & Omni     \\
SK1VHF  & Klintehamn         & 144.4470  & QRV & JO97CJ & 65  & 60 & 10     & Omni     \\
SK1UHG  & Klintehamn         & 1296.9500 & QRV & JO97CJ & 65  & 60 & 30     & Omni     \\
SK6QW/B & Mariestad          & 50.0600   & QRV & JO68WR &     & 8  & 8      & Omni     \\
SK3UHF  & Nordingrå/Rävsön   & 432.4550  & QRV & JP92FW & 200 & 8  & 50     & Omni     \\
SK3UHG  & Nordingrå/Rävsön   & 1296.8550 & QRV & JP92FW & 200 & 10 & 30     & Omni     \\
SK3UHH  & Nordingrå/Rävsön   & 2320.9000 & QRV & JP92FW & 200 & 5  &        & 220 deg. \\
SK7VHF  & Sjöbo              & 144.4610  & QRV & JO65UQ & 25  & 25 & 10     & Omni     \\
SK6VHF  & Tjörn Island       & 144.4480  & QRV & JO57TX & 120 & 12 & 10     & Omni     \\
SK6UHI  & Tjörn Island       & 1296.8050 & QRV & JO57TX & 125 & 18 & 30     & Omni     \\
SK6UHF  & Varberg/Veddige    & 432.4120  & QRV & JO67EH & 175 & 25 & 10     & Omni     \\
SK2VHF  & Vindeln/Bub.       & 144.4570  & QRV & JP94TF & 300 & 10 & 80     & N+SV     \\
SKØEN/B & Väddö              & 1296.8350 & QRV & JO99JX & 70  & 40 & 4      & Omni     \\
SK2SHF  & Vännäs/Granlundsb. & 2320.9850 & QRV & JP93VU & 250 & 50 & 10 / 5 &          \\
SK2SHF  & Vännäs/Granlundsb. & 1296.9850 & QRV & JP93VU & 250 & 50 & 10 / 5 &          \\
SK7GH/B & Värnamo            & 28.2980   & QRV & JO77BF & 230 & 10 & 5      & Omni     \\
SK3SIX  & Östersund          & 50.0700   & QRV & JP73HC & 470 & 7  & 10     & Omni     \\

\end{longtable}

\subsubsection{Svenska fyrar SHF}
\begin{tabular}{rlrllrrrc}

\textbf{Signal} & \textbf{QTH}  & \textbf{Frekvens} & \textbf{Status} & \textbf{Lokator} & 
\textbf{Masl}   & \textbf{Magl} & \textbf{Effekt}   & \textbf{Ritktning} \\ \hline

SK4BX/B & Garphyttan/Ånnaboda & 10368.9600 & QRV & JO79LI & 270 & 10  &       &         \\
SK4BX/B & Garphyttan/Ånnaboda & 5760.8000  & QRT & JO79LI & 270 & 10  &       &         \\
SK6YH/B & Göteborg            & 10368.8100 & QRV & JO57XQ & 135 & 40  & 10000 & 190 deg \\
SK6MHI  & Göteborg            & 24048.8000 & QRV & JO57XQ & 135 & 408 & 10    & Omni    \\
SK6MHI  & Göteborg            & 10368.800  & QRV & JO57XQ & 130 & 10  & 1     & Omni    \\
SK1SHH  & Klintehamn          & 10368.8500 & QRV & JO97CJ & 52  & 52  & 10000 & 360 deg \\
SKØCT/B & Stockholm           & 5760.903   & QRV & JO99JX & 60  & 30  & 80    & Omni    \\
SK6SHG  & Tjörn Island        & 24048.8830 & QRV & JO57TX & 110 & 8   & 2x1W  & N / S   \\
SKØEN/B & Väddö               & 10368.8470 & QRV & JO99JX & 60  & 30  & 10000 & 360 deg \\

\end{tabular}

*) SHF-fyrarna får vara med på VHF/UHF-delen till SHF har fått en egen del i listan.

\clearpage


\subsubsection{Repeatrar distrikt 0}
\footnotesize
\begin{longtable}{llllrrlcl}
	\textbf{Funktion}  & \textbf{Typ} & \textbf{Call} & \textbf{QTH} & \textbf{Frekvens} & \textbf{Skift} & \textbf{Access} & \textbf{Status} & \textbf{Locator} \\ \hline
	\endhead

Repeater & FM           & SMØWIU/R      & Botkyrka     &          434.8750 &          -2MHz & 77,0Hz          &       QRV       & JO89WF           \\
	Repeater           & FM           & SAØAZT/R      & Brottby      &          434.8000 &          -2MHz & 1750/77 Hz$^1$  &       QRV       & JO99DA           \\
	Repeater           & FM           & SMØRDZ        & Brottby      &          145.6500 &        -600kHz & 1750            &       QRV       & JO99DN           \\
	Repeater           & DMR          & SKØRYG        & Färingsö     &          434.6625 &          -2MHz & DV Carrier      &      Plan       & JO89VI           \\
	Repeater           & FM           & SKØVR/R       & Gustavsberg  &          434.9750 &          -2MHz & 1750            &       QRV       & JO99EH           \\
	Link               & FM           & SKØMM         & Gustavsberg  &          434.2250 &                & 91,5Hz          &       QRV       & JO99EH           \\
	Repeater           & D-Star       & SKØQO-B       & Haninge      &          434.5750 &          -2MHz & DV Carrier      &       QRV       & JO99CF           \\
	Repeater           & FM           & SKØQO/R       & Haninge      &          145.6875 &        -600kHz & 77,0Hz          &       QRV       & JO99BE           \\
	Repeater           & FM           & SKØQO/R       & Haninge      &          434.7500 &          -2MHz & 77,0Hz          &       QRV       & JO99BE           \\
	Repeater           & FM           & SKØNN/R       & Haninge      &          434.7750 &          -2MHz & Carrier         &       QRV       & JO99BE           \\
	Hotspot            & D-Star       & SGØBON        & Haninge      &          433.4500 &                & DV Carrier      &       QRV       & JO99CE           \\
	Repeater           & DMR          & SGØSZK        & Hemmesta     &          434.5875 &          -2MHz & DMR 240002      &       QRV       & JO99FH           \\
	Repeater           & FM           & SMØWAJ/R      & Johanneshov  &          434.6500 &          -2MHz & Carrier         &       QRV       & JO99AH           \\
	Link               & FM           & SMØUAO        & Kopparmora   &          434.4875 &                & 91,5 Hz         &       QRV       & JO99HI           \\
	Hotspot            & D-Star       & SGØYOS-C      & L. Essingen  &          145.2375 &                & DV Carrier      &       QRV       & JO99AH           \\
	Hotspot            & D-Star       & SEØYOS-C      & M/Y Erika    &          434.4500 &                & DV Carrier      &       QRV       & JO99AH           \\
	Repeater           & FM           & SKØMK/R       & Mariefred    &          145.7000 &        -600kHz & 1750            &       QRV       & JO89OG           \\
	Repeater           & FM           & SLØDZ/R       & Muskö        &           51.9500 &        -600kHz & 77,0 Hz         &       QRV       & JO98BX           \\
	Repeater           & FM           & SMØGJK/R      & Norrtälje    &          145.6875 &        -600kHz & 1750            &       QRV       & JO99IS           \\
	Repeater           & FM           & SMØGJK/R      & Norrtälje    &          434.9000 &          -2MHz & 1750            &       QRV       & JO99IS           \\
	Repeater           & FM           & SKØBJ/R       & Nynäshamn    &          145.7125 &        -600kHz & 77,0 Hz         &       QRV       & JO88XV           \\
	Repeater           & FM           & SEØE/R        & Saltsjöbo    &           51.8100 &        -600kHz & 1750            &       QRV       & JO99CG           \\
	Link               & FM           & SKØMM         & Sandhamn     &          434.3750 &                & 91,5 Hz         &       QRV       & JO99KG           \\
	Hotspot            & D-Star       & SKØAI-B       & Segeltorp    &          433.4625 &                & DV Carrier      &       QRV       & JO89XG           \\
	Repeater           & DMR/FM       & SKØRPF        & Sigtuna      &          434.8875 &          -2MHz & DMR/123,0Hz     &       QRV       & JO89VP           \\
	Repeater           & FM           & SMØWAJ/R      & Skärholmen   &          434.9250 &        +1,6MHz & Carrier         &       QRV       & JO89WG           \\
	Repeater           & FM           & SMØHGS/R      & Skärholmen   &          145.7250 &        -600kHz & 77,0 Hz         &       QRV       & JO89WG           \\
	Repeater           & FM           & SMØWHP        & Sollentuna   &          434.6000 &          -2MHz & 1750 Hz         &       QRV       & JO89XL           \\
	Repeater           & FM           & SKØCT/R       & Solna        &          434.9500 &          -2MHz & 77,0 Hz         &       QRV       & JO89XJ           \\
	Repeater           & FM           & SKØZA/R       & Solna        &          434.8500 &          -2MHz & 123,0 Hz        &       QRV       & JO89XI           \\
	Repeater           & FM           & SMØOFV/R      & Solna        &          145.7625 &        -600kHz & 123,0 Hz        &       QRV       & JO99AI           \\
	Repeater           & FM           & SKØCT/R       & Solna        &           29.6200 &        -100kHz & 77,0 Hz         &        ?        & JO89XJ           \\
	Repeater           & FM           & SMØKOT        & Stockholm    &          434.8250 &          -2MHz & 1750 Hz         &       QRT       &  \\
	Repeater           & DMR          & SMØRMQ        & Stockholm    &          434.5125 &          -2MHz & DMR 240010      &       QRV       & JO99CH           \\
	Repeater           & FM           & SMØWAJ/R      & Stockholm    &          434.8375 &          -2MHz &                 &      Plan       & JO99AH           \\
	Hotspot            & D-Star       & SKØVF-C       & Bandhagen    &          145.3625 &                & DV Carrier      &       QRV       & JO99AG           \\
	Hotspot            & D-Star       & SKØVF-B       & Bandhagen    &         1296.7000 &                & DV Carrier      &       QRV       & JO99AG           \\
	Repeater           & FM           & SKØCT/R       & Kista        &         1297.0250 &          -6MHz & Carrier         &       QRV       & JO89XJ           \\
	Repeater           & FM           & SKØCT/R       & Kista        &          434.6250 &          -2MHz & 77,0 Hz         &       QRV       & JO89XJ           \\
	Repeater           & FM           & SKØPQ/R       & Kista        &          145.6750 &        -600kHz & 123,0 Hz        &       QRV       & JO86XV           \\
	Repeater           & DMR          & SAØAZT/R      & Kista        &          434.9875 &          -2MHz & DMR             &       QRV       & JO89XJ           \\
	Repeater           & D-Star       & SKØPT         & Kungsängen   &          434.5500 &          -2MHz & DV Carrier      &      Plan       & JO89UL           \\
	Repeater           & FM           & SMØYIX/R      & Söder        &          434.7250 &          -2MHz & 77,0 Hz         &       QRV       & JO99BH           \\
	Repeater           & DMR          & SKØNN         & Södertörn    &          434.5375 &          -2MHz & DMR             &       QRV       & JO99CF           \\
	Hotspot            & D-Star       & SBØX-B        & Uppl. Väsby  &          433.5750 &                & DV Carrier      &       QRV       & JO89XM           \\
	Repeater           & DMR          & SKØRYG        & Uppl. Väsby  &          434.7625 &          -2MHz & DMR/123,0Hz     &       QR        & JO89XM           \\
	Hotspot            & D-Star       & SGØAMO        & Åkersberga   &          145.3625 &                & DV Carrier      &       QRV       & JO99DL           \\
	Repeater           & FM           & SKØRIX        & Sthlm city   &          145.6250 &        -600kHz & Carrier         &       QRV       & JO99AH           \\
	Repeater           & DMR          & SKØMG         & Sthlm city   &          434.6875 &          -2MHz & DMR 240099      &       QRV       & JO99AI           \\
	Repeater           & FM           & SM5DWC/R      & Södertälje   &          434.8250 &          -2MHz & 1750/77,0Hz     &       QRV       & JO89TE           \\
	Repeater           & FM           & SMØMMO/R      & Tullinge     &          145.6625 &        -600kHz & 77,0 Hz         &       QRV       & JO89XF           \\
	Repeater           & FM           & SKØMT/R       & Täby         &          434.7000 &        +1,6MHz & Carrier         &       QRV       & JO99AK           \\
	Repeater           & FM           & SKØMT/R       & Täby         &          434.7375 &          -2MHz & 77,0 Hz         &      Plan       & JO99AK           \\
	Repeater           & DMR          & SKØMQ         & Täby         &          434.5625 &          -2MHz & DMR             &       QRV       & JO99AK           \\
	Repeater           & FM           & SKØRYG        & Uppl. Väsby  &          434.6750 &          -2MHz & 1750/77,0Hz     &       QRV       & JO89XM           \\
	Repeater           & FM           & SKØRYG        & Uppl. Väsby  &           51.9700 &        -600kHz & 1750/77,0Hz     &       QRV       & JO89WM           \\
	Repeater           & FM/Fusion    & SK0MG/R         & Vårby        &          434.6875 &          -2MHz & 77,0Hz          &       QRV       & ?
\end{longtable}
\begin{itemize}
	\item[$^1$] Öppnas även med DTMF *
\end{itemize}
\normalsize

\clearpage

\subsubsection{Repeatrar distrikt 1}
\footnotesize
\begin{longtable}{llllrrlcl}
\textbf{Funktion} & \textbf{Typ} & \textbf{Call} & \textbf{QTH} & \textbf{Frekvens} & \textbf{Skift} & \textbf{Access} & \textbf{Status} & \textbf{Locator} \\ \hline \endhead
Repeater          & FM/C4FM      & SK1BL/R       & Endre	& 145.7750          & -600kHz        & 1750            & QRV             & JO97FO           \\
\end{longtable}
%\begin{itemize}
%	\item[$^1$] Öppnas även med DTMF *
%\end{itemize}
\normalsize

\subsubsection{Repeatrar distrikt 2}
\footnotesize
\begin{longtable}{llllrrlcl}
\textbf{Funktion}                 & \textbf{Typ}  & \textbf{Call} & \textbf{QTH}        & \textbf{Frekvens} & \textbf{Skift} & \textbf{Access}  & \textbf{Status} & \textbf{Locator} \\ \hline \endhead

Repeater & FM     & SK2AU/R   & Arjeplog           & 145.7000 & -600kHz & 1750             & QRV  & JP86XC \\
Repeater & FM     & SM2TBQ/R  & Fatmomakke         & 145.675  & -600kHz & 88,5Hz           & QRV  & JP75NC \\
Repeater & FM     & SK2TP/R   & Gällivare/Malmb.   & 145.6500 & -600kHz & 1750             & QRV  & KP07HC \\
Repeater & FM     & SK2AU/R   & Jörn/Storklinta    & 145.7500 & -600kHz & 1750             & QRV  & KP05BD \\
Repeater & FM     & SK2HG/R   & Kalix              & 51.9500  & -600kHz & 1750/100,0Hz$^1$ & QRV  & KP15NU \\
Repeater & FM     & SK2HG/R5  & Kalix/Raggdynan    & 145.7250 & -600kHz & 1750             & QRV  & KP15JV \\
Repeater & FM     & SK2RJH    & Kalix/Raggdynan    & 434.7500 & -2MHz   & 1750             & QRV  & KP15JV \\
Repeater & FM     & SK2HG/RU5 & Kalix-Vattent.     & 434.7250 & -2MHz   & 1750             & QRV  & KP15NU \\
Repeater & FM     & SK2RFR    & Kiruna             & 145.6250 & -600kHz & 1750             & QRV  & KP07DU \\
Repeater & FM     & SK2RFR    & Kiruna C           & 434.8250 & -2MHz   & 1750             & QRV  & KP07DU \\
Repeater & FM     & SM2KOT/R  & Kristineb./Hornb.  & 145.6750 & -600kHz & 1750             & QRV  & JP95GB \\
Repeater & FM     & SK2AZ/R   & Luleå              & 145.6500 & -600kHz & 1750             & QRV  & KP15CO \\
Repeater & FM     & SK2LY/R   & Lycksele           & 145.6000 & -600kHz & 1750             & QRT  & JP94IO \\
Repeater & FM     & SM2KXX    & Lycksele           & 434.7750 & -1,6MHz & 1750             & QRV  & JP94HO \\
Repeater & FM     & SK2RLE    & Lycksele/Alsb.     & 145.7750 & -600kHz & 1750             & QRT  & JP84VP \\
Repeater & FM     & SK2AT/R   & Nordmaling         & 434.6750 & -2MHz   & 67,0Hz           & QRV  & JP93RN \\
Repeater & FM     & SK2RME    & Piteå              & 145.6000 & -600kHz & 1750             & QRT  & KP05PH \\
Repeater & FM     & SK2RME    & Piteå              & 434.6000 & -1,6MHz & 1750             & QRV  & KP05RH \\
Repeater & FM     & SK2RME    & Piteå              & 29.6800  & -100kHz & 1750/67Hz        & QRV  & KP05RH \\
Repeater & FM     & SK2RUQ    & Saxnäs/Marsfj.     & 145.7500 & -600kHz & Mod.             & QRT  & JP74PX \\
Repeater & FM     & SK2HG/R3  & Seskarö            & 145.6750 & -600kHz & 1750             & QRV  & KP15UR \\
Repeater & FM     & SK2AU/R   & Skellefteå         & 145.7000 & -600kHz & 1750             & QRV  & KP04LS \\
Repeater & FM     & SK2AU/R   & Skellefteå         & 434.7000 & -2MHz   & 1750             & QRV  & KP05LS \\
Repeater & D-Star & SK2RFV    & Skellefteå         & 434.5250 & -2MHz   & DV Carrier       & Plan & KP04LS \\
Repeater & FM     & SK2RMD    & Sorsele            & 145.6000 & -600kHz & 1750             & QRV  & JP85SM \\
Repeater & FM     & SK2RMR    & Storuman           & 145.7250 & -600kHz & 1750             & QRV  & JP85NC \\
Repeater & FM     & SM2NLD/R  & Storuman           & 434.7500 & -1,6MHz & 1750             & QRV  & JP85NC \\
Repeater & FM     & SK2RLF    & Tärnaby            & 145.6250 & -600kHz & 1750             & QRV  & JP75PR \\
Repeater & D-Star & SK2AT-C   & Umeå               & 145.6875 & -600kHz & DV Carrier       & QRV  & KP03BU \\
Repeater & D-Star & SK2AT-B   & Umeå               & 434.9750 & -2MHz   & DV Carrier       & QRV  & KP03BU \\
Repeater & FM     & SL2ZA/R   & Umeå/Ersmarksb.    & 145.6000 & -600kHz & 1750             & QRT  & KP03EV \\
Repeater & FM     & SK2RLJ    & Umeå/Rödb.         & 145.6500 & -600kHz & 1750             & QRV  & KP03CU \\
Repeater & FM     & SK2RLX    & Vilhelmina         & 145.7000 & -600kHz & 1750             & QRT  & JP84HO \\
Repeater & FM     & SK2RLX    & Vilhelmina         & 434.7000 & -1,6MHz & 1750             & QRT  & JP84HO \\
Repeater & FM     & SK2RYI    & Vindeln/Åsträsk    & 145.6250 & -600kHz & 1750             & QRV  & KP04DP \\
Repeater & FM     & SK2RIU    & Vännäs/Granlundsb. & 145.7250 & -600kHz & 1750             & QRV  & JP93VU \\
Repeater & FM     & SK2RIU    & Vännäs/Granlundsb. & 434.7250 & -2MHz   & 1750             & QRV  & JP93VU \\
Repeater & FM     & SK2RWJ    & Älvsbyn            & 145.6750 & -600kHz & 1750             & QRV  & KP05LQ \\
Repeater & FM     & SK2HG/R6  & Överkalix          & 145.7500 & -600kHz & 1750             & QRV  & KP16KH

\end{longtable}
\begin{itemize}
	\item[$^1$] Öppnas även med DTMF 3
\end{itemize}
\normalsize
\subsubsection{Repeatrar distrikt 3}
\footnotesize
\begin{longtable}{llllrrlcl}

\textbf{Funktion} & \textbf{Typ}    & \textbf{Call}   & \textbf{QTH} & \textbf{Frekvens} & 
\textbf{Skift}    & \textbf{Access} & \textbf{Status} & \textbf{Locator} \\ \hline \endhead

Repeater & FM     & SK3KH/R  & Alfta            & 145.6000  & -600KHz  & 127,3Hz       & QRV & JP71XF \\
Repeater & D-Star & SK3KH    & Alfta            & 434.5500  & -2MHz    & DV Carrier    & QRV & JP71XE \\
Repeater & FM     & SK3RMG   & Bergsjö          & 145.7125  & -600KHz  & 1750          & QRV & JP81MX \\
Repeater & FM     & SK3RMG   & Bergsjö          & 434.9750  & -1,6MHz  & 1750          & QRV & JP81MX \\
Repeater & FM     & SK3RMG   & Bergsjö          & 1297.1000 & -6MHz    & 1750          & QRV & JP81MX \\
Repeater & D-Star & SK3RET   & Bollnäs          & 145.5875  & -600KHz  & DV Carrier    & QRV & JP81CL \\
Repeater & FM     & SK3BR/R  & Bollnäs/Arbrå    & 145.6500  & -600KHz  & 1750/127,3Hz  & QRV & JP81EI \\
Repeater & FM     & SK3BR/R  & Bollnäs/Arbrå    & 434.6500  & -2MHz    & 1750/127,3Hz  & QRV & JP81EI \\
Repeater & FM     & SK3RIN   & Borgsjö          & 145.7000  & -600KHz  & 1750Hz        & QRV & JP72WN \\
Repeater & FM     & SK3RIN   & Borgsjö          & 434.7000  & -1,6MHz  & 1750Hz        & QRT &        \\
Hotspot  & D-Star & SG3IVF-B & Enafors          & 434.5500  &          & DV Carrier    & QRV & JP63EG \\
Repeater & D-Star & SK3GY-B  & Gävle            & 434.9250  & -2MHz    & DV Carrier    & QRV & JP80JO \\
Repeater & FM     & SK3GW/R  & Gävle            & 434.8750  & -2MHz    & 1750/127,3Hz  & QRV & JP80NP \\
Hotspot  & D-Star & SG3CFY-B & Gävle            & 433.4750  &          & DV Carrier    & QRV & JP80NQ \\
Repeater & FM     & SK3YZ/R  & Hassela/Forsa    & 145.6750  & -600KHz  & 1750/74,4Hz   & QRV & JP82IC \\
Repeater & FM     & SK3RQE   & Hassela/Forsa    & 434.6750  & -2MHz    & 1750/127,3Hz  & QRV & JO81KS \\
Repeater & FM     & SK3YZ/R  & Hassela/Forsa    & 145.6125  & -600KHz  & 1750/127,3Hz  & QRV & JP82IC \\
Repeater & FM     & SK3RMX   & Hoting/Kyrktåsjö & 145.6000  & -600KHz  & 1750Hz        & QRV & JP74XF \\
Repeater & FM     & SK3RMX   & Hoting/Kyrktåsjö & 434.6000  & -1,6MHz  & 1750Hz        & QRT & JP84CC \\
Repeater & FM     & SK3GA/R  & Hudiksvall       & 145.7750  & -600KHz  & 1750Hz        & QRV & JP81NR \\
Repeater & FM     & SK3GA/R  & Hudiksvall       & 434.7750  & -1,6MHz  & 1750Hz        & QRV & JO81NR \\
Hotspot  & D-Star & SK3GA-C  & Hudiksvall       & 145.2375  &          & DV Carrier    & QRV & JP81NR \\
Hotspot  & D-Star & SK3GA-B  & Hudiksvall       & 434.4750  &          & DV Carrier    & QRV & JP81NR \\
Repeater & FM     & SL3ZB    & Härnösand        & 434.7250  & -2MHz    & 1750Hz        & QRV & JP82WQ \\
Repeater & FM     & SK3WH    & Högakustenbron   & 1297.2750 & -6MHz    & 1750Hz        & QRV & JP82XT \\
Repeater & FM     & SM3XRJ   & Kramfors         & 434.6000  & -2MHz    & 1750Hz        & QRV & JP82VW \\
Repeater & FM     & SK3IK/R  & K-fors/Nyland    & 145.6000  & -600KHz  & 1750Hz        & QRT & JP83UA \\
Repeater & FM     & SK3RAL   & Ljusdal          & 434.9000  & -1,6MHz  & 1750Hz        & QRV & JP81AV \\
Repeater & FM     & SK3MF/R  & Nordingrå/Rävsön & 145.6250  & -600KHz  & 1750Hz        & QRV & JP92FW \\
Repeater & FM     & SK3MF/R  & Nordingrå/Rävsön & 434.8500  & -2MHz    & 1750Hz        & QRV & JP92FW \\
Repeater & FM     & SM3VAC/R & Nyland           & 145.7500  & -600KHz  & 1750Hz        & QRV & JP83UA \\
Repeater & FM     & SM3VAC/R & Nyland           & 434.9500  & -1,6MHz  & 1750Hz        & QRV & JP83UA \\
Repeater & FM     & SK3GK/R  & Sandviken        & 434.7000  & -2MHz    & 127,3Hz/DTMF1 & QRV & JP80JO \\
Repeater & D-Star & SK3GY-C  & Sandviken        & 145.7625  & -600KHz  & DV Carrier    & QRV & JP80JO \\
Repeater & FM     & SK3GK/R  & Sandviken        & 145.7000  & -600KHz  & 127,3Hz       & QRV & JP80JO \\
Repeater & FM     & SK3GW/R  & S-viken/Kungsb.  & 434.8250  & -2MHz    & 1750/127,3Hz  & QRV &        \\
Repeater & FM     & SK3EK/R  & Sollefteå        & 145.6500  & -600KHz  & 1750Hz        & QRV & JP83PD \\
Repeater & FM     & SK3EK/R  & Sollefteå        & 434.6500  & -1,6MHz  & 1750Hz        & QRV & JP83DE \\
Repeater & FM     & SK3RFG   & Sundsvall        & 145.7250  & -600KHz  & 1750Hz        & QRV & JP82RJ \\
Repeater & FM     & SK3BG/R  & S-vall/Grundsjön & 51.8300   & -600KHz  & 127,3Hz       & QRT & JP72WH \\
Repeater & D-Star & SK3RFG-B & S-vall/Klissb.   & 434.8000  & -2MHz    & DV Carrier    & QRV & JP82OJ \\
Repeater & D-Star & SK3RFG-C & S-vall/Klissb.   & 145.5875  & -600KHz  & DV Carrier    & QRV & JP82OJ \\
Hotspot  & D-Star & SE3XCH-B & S-vall/Nolbyk.   & 433.0125  &          & DV Carrier    & QRV & JP82QH \\
Hotspot  & D-Star & SE3XCH-C & S-vall/Nolbyk.   & 144.8625  &          & DV Carrier    & QRV & JP82QH \\
Repeater & FM     & SK3RYK   & Söderhamn        & 145.7500  & -600KHz  & 1750Hz        & QRV & JP81NH \\
Repeater & FM     & SK3RYK   & Söderhamn        & 434.7500  & -1,6MHz  & 1750Hz        & QRV & JP81NH \\
Repeater & D-Star & SK3XX-C  & Söderhamn        & 145.5750  & -600KHz  & DV Carrier    & QRV & JP81NH \\
Repeater & D-Star & SK3XX-A  & Söderhamn        & 1297.0750 & -6MHz    & DV Carrier    & QRV & JP81NH \\
Repeater & D-Star & SK3XX-B  & Söderhamn        & 434.5250  & -2MHz    & DV Carrier    & QRV & JP81NH \\
Repeater & FM     & SI9AM/R  & Utanede          & 145.6125  & -600KHz  & 1750Hz        & QRV & JP82IX \\
Repeater & FM     & SK3RQC   & Vemdalen         & 145.6250  & -600KHz  & 1750/74,4Hz   & QRV & JP62WK \\
Repeater & FM     & SK3RNJ   & Åreskutan        & 145.7250  & -600KHz  & 127,3Hz       & QRV & JP63NK \\
Repeater & FM     & SM3LEI/R & Årsunda          & 434.6500  & +1,6MHz  & 1750/88,5Hz   & QRV & JP80IM \\
Repeater & FM     & SK3RKL   & Örnsköldsvik     & 434.7750  & -2MHz    & 1750Hz        & QRV & JP93IH \\
Repeater & FM     & SK3WH/R  & Örnsköldsvik     & 434.8750  & -2MHz    & 1750Hz        & QRV & JP93IH \\
Repeater & D-Star & SK3WH-C  & Örnsköldsvik     & 145.5750  & -600KHz  & DV Carrier    & QRV & JP93IH \\
Repeater & FM     & SK3RLO   & Ö-vik/Bågaliden  & 145.6750  & -600KHz  & 1750Hz        & QRT & JP93ES \\
Repeater & D-Star & SK3LH-B  & Ö-vik/Malmön     & 434.5750  & -2MHz    & DV Carrier    & QRV & JP93LF \\
Repeater & FM     & SK3RKL   & Ö-/Rutberget     & 145.7750  & -600KHz  & 1750Hz        & QRV & JP93GJ \\
Link     & FM     &          & Österfärnebo     &           & 144.5500 & 127,3Hz       & QRV & JP80JH \\
Repeater & FM     & SK3JR/R  & Östersund        & 145.7500  & -600KHz  & 1750Hz        & QRV & JP73JE \\
Repeater & FM     & SK3RIA   & Östersund        & 434.7500  & -2MHz    & 127,3Hz       & QRV & JP73JE \\
Repeater & D-Star & SK3JR    & Östersund        & 434.5625  & -2MHz    & DV Carrier    & QRV & JP73HC \\
Repeater & FM     & SK3JR/R2 & Ö-sund/Brattås.  & 145.7875  & -600KHz  & 127,3Hz       & QRV & JP73HC

\end{longtable}
\normalsize
\subsubsection{Repeatrar distrikt 4}
\footnotesize
\begin{longtable}{llllrrlcl}

\textbf{Funktion} & \textbf{Typ}    & \textbf{Call}   & \textbf{QTH} & \textbf{Frekvens} & 
\textbf{Skift}    & \textbf{Access} & \textbf{Status} & \textbf{Locator} \\ \hline \endhead

Repeater & FM     & SK4RWQ   & Arvika/Valfj.       & 434.7750 & -2MHz      & 1750            & QRV  & JO69ES \\
Repeater & FM     & SK4UH/R  & Avesta              & 434.8500 & -2MHz      & 1750            & QRV  & JP80CD \\
Repeater & FM     & SK4RVN   & Borlänge            & 434.8000 & -2MHz      & 74,4Hz          & QRV  & JP70RL \\
Hotspot  & D-Star & SG4UZM-B & Borlänge            & 434.5500 & DV Carrier &                 & QRV  & JP70RM \\
Hotspot  & D-Star & SG4YPG-B & Borlänge            & 433.6000 & DV Carrier &                 & QRV  & JP70RM \\
Repeater & D-Star & SK4BW-B  & Borlänge            & 434.9000 & -2MHz      & DV Carrier      & QRV  & JP70RJ \\
Hotspot  & D-Star & SG4AXV-C & Ekshärad            & 145.4250 & DV Carrier &                 & QRV  & JP60RE \\
Repeater & FM     & SK4AO/R  & Falun               & 145.6250 & -600kHz    & 1750            & QRV  &        \\
Repeater & FM     & SK4AO/R  & Falun               & 434.6250 & -2MHz      & 1750            & QRV  & JP70TO \\
Link     & FM     & SK4AV/R  & Filipstad/Klockarh. & 145.2000 &            & Carrier         & QRV  & JO79CR \\
Repeater & FM     & SM4EFQ   & Filipstad/Storh.    & 145.7000 & -600kHz    & 1750            & QRV  & JO79CR \\
Hotspot  & D-Star & SG4UOF-C & Glanshammar         & 145.3375 &            & DV Carrier      & QRV  & JO79RI \\
Repeater & FM     & SK4IL/R  & Grums               & 434.7250 & -2MHz      & 74,4Hz          & QRV  & JO69NI \\
Link     & FM     &          & Grängesberg         & 145.4500 &            &                 & QRV  & JP70LG \\
Repeater & FM     & SK4HV/R  & Hagfors/Värmullsås. & 145.6750 & -600kHz    & 1750/114,8Hz    & QRV  & JP60VA \\
Link     & FM     & SK4RJJ   & Hagfors/Värmullsås. & 145.2250 &            & 74,4 Hz         & QRV  & JP60UA \\
Repeater & FM     & SK4RKD   & Karlskoga           & 145.7500 & -600kHz    & Carrier         & QRV  & JO79FJ \\
Hotspot  & D-Star & SG4BYH   & Karlstad            & 145.3875 &            & DV Carrier      & QRV  & JO69RK \\
Repeater & FM     & SK4KS    & K-stad/Kronop.      & 434.8500 & -2MHz      & 1750            & QRV  & JO69TJ \\
Repeater & FM     & SK4KIL   & Kil                 & 145.6250 & -600kHz    &                 & QRV  & JO69QM \\
Repeater & D-Star & SG4JPK   & Kil                 & 145.5750 & -600kHz    & DV Carrier      & QRV  & JO69PM \\
Repeater & FM     & SK4KIL   & Kil                 & 434.9250 & -2MHz      & 74,4 Hz         & Plan & JO69NO \\
Repeater & FM     & SK4EA/R  & Kopparberg          & 145.6000 & -600kHz    & 1750            & QRV  & JP79MW \\
Repeater & FM     & SK4RWP   & Kristinehamn        & 434.7000 & -2MHz      & 1750            & QRV  & JO79AH \\
Repeater & FM     & SK4RUV   & Leksand             & 145.7750 & -600kHz    & 1750/85,4Hz     & QRV  & JP70KQ \\
Repeater & FM     & SK4EA/R  & Lindesberg          & 145.6875 & -600kHz    & 1750/74,4Hz     & QRV  & JO79NP \\
Repeater & FM     & SK4DM/R  & Ludvika             & 145.7250 & -600kHz    & 1750            & QRV  & JP70NC \\
Repeater & FM     & SK4DM/R  & Ludvika             & 434.7250 & -1,6MHz    & 1750/DTMF1      & QRV  & JP70NC \\
Repeater & FM     & SM4JDP/R & Mora                & 145.7000 & -600kHz    & 1750/118,8Hz    & QRV  & JP71DA \\
Repeater & FM     & SM4JDP   & Mora                & 434.6750 & -2MHz      & 71,9Hz          & QRV  & JP71GA \\
Repeater & FM     & SM4JDP   & Mora                & 434.8500 & -2MHz      & Carrier         & QRV  & JP71GA \\
Repeater & D-Star & SG4TYA   & Mora                & 145.5750 & -600kHz    & DV Carrier      & QRV  & JP71GE \\
Link     & FM     &          & Nyhammar            & 145.4250 &            &                 & QRV  & JP70LG \\
Repeater & FM     & SK4KO/R  & Orsa/Grönk.         & 145.7500 & -600kHz    & 1750            & QRV  & JP71GF \\
Repeater & FM     & SK4KO/R  & Orsa/Grönk.         & 434.7500 & -1,6MHz    & 1750            & QRV  & JP71GF \\
Hotspot  & D-Star & SG4AXQ   & Sunne               & 433.2000 &            & DV Carrier      & QRV  & JO69NU \\
Repeater & FM     & SK4RJJ   & Sunne/Blåbärsk.     & 145.7750 & -600kHz    & 1750/74,4Hz$^1$ & QRV  & JO69KU \\
Repeater & FM     & SK4ROI   & Särna               & 145.6750 & -600kHz    & 1750            & QRV  & JP61NQ \\
Link     & FM     & SK4RJJ   & Torsby/Hovfj.       & 145.2875 &            & 74,4Hz          & QRV  & JO69LH \\
Repeater & FM     & SK4RPK   & Torsby/Valb.        & 434.6250 & -2MHz      & 1750Hz          & QRV  & JP60LC \\
Repeater & D-Star & SK4NI-C  & Tossebergsklätten   & 145.7625 & -600kHz    & DV Carrier      & QRV  & JO69MX \\
Repeater & FM     & SK4WV/R  & Vansbro             & 145.6500 & -600kHz    & 1750 Hz         & QRV  & JP70AM \\
Repeater & FM     & SK4WV/R  & Vansbro             & 434.6500 & -1,6MHz    & 1750            & QRT  & JP70AM \\
Repeater & FM     & SK4RQF   & Årjäng              & 145.7250 & -600kHz    & 1750            & QRV  & JO69BJ \\
Repeater & FM     & SK4BX/R  & Örebro              & 145.6500 & -600kHz    & 1750/74.4Hz$^1$ & QRV  & JO79LH

\end{longtable}
\begin{itemize}
\item[$^1$] Kan även öppnas med DTMF 1
\end{itemize}
\subsubsection{Repeatrar distrikt 5}
\begin{longtable}{llllrrlcl}

\textbf{Funktion} & \textbf{Typ}    & \textbf{Call}   & \textbf{QTH} & \textbf{Frekvens} & 
\textbf{Skift}    & \textbf{Access} & \textbf{Status} & \textbf{Locator} \\ \hline \endhead

Repeater & FM     & SK5RTG   & Arboga/Kolsva     & 145.7375 & -600KHz & Carrier       & QRV  & JP79WO \\
Repeater & FM     & SK5RTG   & Arboga/Kolsva     & 434.9000 & -2MHz   & Carrier       & QRV  & JP79WO \\
Repeater & FM     & SK5WB/R  & Enköping          & 434.9500 & -2MHz   & 156,7Hz       & QRV  & JO89NP \\
Repeater & FM     & SK5VM/R  & Eskilstuna        & 434.9750 & -2MHz   & 82,5Hz        & QRV  & JO89GI \\
Repeater & FM     & SK5LW/R  & E-tuna/Torshälla  & 145.6125 & -600KHz & 82,5Hz/DTMF 0 & QRV  & JO89FJ \\
Repeater & D-Star & SK5LW/R  & E-tuna/Torshälla  & 434.8500 & -2MHz   & DV Carrier    & QRV  & JO89FJ \\
Repeater & FM     & SK5LW/R  & E-tuna/Ärla       & 51.8500  & -600KHz & 82,5Hz        & QRT  & JO89FJ \\
Repeater & FM     & SK5LW/R  & E-tuna/Ärla       & 145.5750 & -600KHz & 82,5Hz        & QRT  & JO89IG \\
Repeater & FM     & SK5LW/R  & E-tuna/Ärla       & 29.6400  & -100KHz & 82,5Hz        & QRV  & JO89IG \\
Repeater & FM     & SK5BN/R  & Finspång          & 434.9250 & -2MHz   & 107,2Hz       & QRV  & JO78VR \\
Hotspot  & D-Star & SG5TAH-C & Flen/Orrhammar    & 145.3375 &         & DV Carrier    & QRV  & JO89GB \\
Repeater & D-Star & SK5UM-B  & Flen/Vattent.     & 434.5500 & -2MHz   & DV Carrier    & QRV  & JO89HB \\
Repeater & FM     & SK5UM/R  & Flen/Vattent.     & 145.7500 & -600KHz & 91,5Hz        & QRV  & JO89GB \\
Repeater & FM     & SK5UM/R  & Flen/Öja          & 434.7500 & -2MHz   & 1750/91,5Hz   & QRV  & JO89HB \\
Hotspot  & D-Star & SG5UM    & Flen/Öja          & 145.2375 &         & DV Carrier    & QRV  & JO89HB \\
Repeater & FM     & SK5RLZ   & Katrineholm       & 145.7875 & -600KHz & 1750/DTMF 5   & QRV  & JO88CX \\
Hotspot  & D-Star & SA5RG    & Katrineholm       & 434.4125 &         & DV Carrier    & QRV  & JO88CX \\
Repeater & D-Star & SA5RG    & Katrineholm       & 434.5250 & -2MHz   & DV Carrier    & QRV  & JO88CX \\
Repeater & FM     & SK5RCQ   & Kisa              & 145.7000 & -600KHz & 1750Hz        & QRV  & JO77TX \\
Repeater & D-Star & SA5BCG/R & Knivsta           & 434.5250 & -2MHz   & 82,5Hz        & QRV  & JO89VR \\
Repeater & FM     & SK5BB/R  & Köping            & 145.6375 & -600KHz & 1750/82,5Hz   & QRT  & JO79XM \\
Repeater & FM     & SK5BB/R  & Köping            & 434.8750 & -2MHz   & Carrier       & QRT  & JO89AM \\
Repeater & FM     & SK5AS/R  & Linköping         & 145.7250 & -600KHz & 1750          & QRV  & JO78TJ \\
Repeater & FM     & SM5YMS/R & Linköping         & 434.8000 & -2MHz   & 1750          & QRV  & JO78SM \\
Repeater & FM     & SM5YMS/R & Linköping         & 51.8300  & -600KHz & 1750          & QRV  & JO78SM \\
Repeater & FM     & SM5DWC/R & Linköping         & 434.8750 & -2MHz   & 82,5Hz        & QRV  & JO78SM \\
Repeater & FM     & SM5DWC/R & Linköping         & 51.8900  & -600KHz & 82,5Hz        & QRT  & JO78SM \\
Repeater & FM     & SK5LF/R  & L-köping/Majelden & 434.7250 & -2MHz   & 82,5Hz        & QRV  & JO78TJ \\
Repeater & FM     & SM5YMT/R & Ljungsbro         & 145.6625 & -600KHz & 1750Hz        & QRV  & JO78SM \\
Repeater & FM     & SK5AS/R  & Ljungsbro         & 145.7375 & -600KHz & 1750/82,5     & QRV  & JO78SN \\
Repeater & FM     & SK5AS/R  & Ljungsbro         & 434.9000 & -2MHz   & 1750/82,5     & QRV  & JO78SN \\
Repeater & FM     & SL5ZAM/R & Motala            & 145.7625 & -600KHz & 1750/DTMF 5   & QRT  & JO78MN \\
Repeater & D-Star & SL5ZYT-C & Norrköping        & 145.5750 & -600KHz & DV Carrier    & QRV  & JO88BO \\
Repeater & FM     & SL5ZYT/R & Norrköping        & 434.9500 & -2MHz   & 82,5Hz        & QRV  & JO88DQ \\
Repeater & FM     & SK5BN/R  & Norrk./Kolmården  & 145.6000 & -600KHz & 1750/DTMF 5   & QRV  & JO88FQ \\
Repeater & FM     & SK5BN/R  & Norrk./Ö. Eneby   & 434.6000 & -2MHz   & 1750Hz        & QRV  & JO88BO \\
Repeater & FM     & SK5BE/R  & Nyköping          & 145.6375 & -600KHz & 82,5Hz        & QRV  & JO88LS \\
Link     & FM     & SM5RVH/L & Nyköping          & 145.4500 &         & 103,5Hz       & QRV  & JO88LQ \\
Hotspot  & D-Star & SG5RVH-B & Nyköping          & 434.4875 &         & DV Carrier    & QRV  & JO88LQ \\
Repeater & FM     & SK5RO/R  & Tierp             & 145.6000 & -600KHz & 1750          & QRT  & JP80SJ \\
Repeater & FM     & SK5RO/R  & Tierp             & 434.6000 & -1,6MHz & 1750          & QRT  & JP80SJ \\
Repeater & FM     & SK5DB/R  & Uppsala           & 145.7500 & -600KHz & 1750/82,5Hz   & QRV  & JO89VU \\
Repeater & FM     & SK5DB/R  & Uppsala           & 434.7500 & -2MHz   & 1750/82,5Hz   & QRV  & JO89VU \\
Hotspot  & D-Star & SC5SLU-C & Uppsala           & 145.3250 & Simplex & DV Carrier    & QRV  & JO89QW \\
Hotspot  & D-Star & SM5EZN-B & Uppsala           & 433.4875 & Simplex & DV Carrier    & QRV  & JO89QW \\
Hotspot  & D-Star & SG5BJY   & Uppsala           & 145.2875 &         & DV Carrier    & QRV  & JO89TT \\
Repeater & D-Star & SG5DV    & Uppsala           & 145.5875 & -600KHz & DV Carrier    & Plan & JO89TT \\
Repeater & FM     & SM5USM/R & Vingåker          & 434.6750 & -2MHz   & Carrier       & QRV  & JO79XB \\
Repeater & FM     & SK5RHQ   & Västerås          & 145.7750 & -600KHz & 1750/82,5Hz   & QRV  & JO89GO

\end{longtable}
\normalsize
\subsubsection{Repeatrar distrikt 6}
\footnotesize
\begin{longtable}{llllrrlcl}

\textbf{Funktion} & \textbf{Typ}    & \textbf{Call} & \textbf{QTH} & \textbf{Frekvens} & \textbf{Skift} & 
\textbf{Access}   & \textbf{Status} & \textbf{Locator} \\ \hline \endhead

Repeater & FM     & SK6RIC   & Alingsås            & 145.6250  & -600KHz & 1750/114,8Hz    & QRV  & JO67GW \\
Repeater & FM     & SK6RIC   & Alingsås            & 434.6250  & -2MHz   & 1750/114,8Hz    & QRV  & JO67GW \\
Repeater & FM     & SA6AR/R  & Angered             & 434.9250  & -2MHz   & 1750Hz          & QRV  & JO67AT \\
Repeater & FM     & SK6RFP   & Billingsfors        & 145.7000  & -600KHz & 118,8Hz         & QRV  & JO69CA \\
Repeater & FM     & SL6ZYW/R & Billingsfors        & 434.7000  & -2MHz   & 1750Hz          & QRV  & JO69CA \\
Repeater & FM     & SA6RP/R  & Björboholm          & 434.8250  & -2MHz   & 118,8Hz         & QRV  & JO67DV \\
Repeater & FM     & SK6LK/R  & Borås               & 145.7750  & -600KHz & 1750/114,8Hz    & QRV  & JO67MR \\
Repeater & FM     & SK6RBS   & Borås               & 434.8000  & -2Mhz   & 1750Hz          & QRV  & JO67MR \\
Hotspot  & D-Star & SG6BWX-C & Borås               & 145.2875  &         & DV Carrier      & QRV  & JO67LR \\
Repeater & DMR    & SM6TKT/R & Borås               & 434.5500  & -2MHz   & 240610          & QRV  & JO67LR \\
Repeater & FM     & SK6JX/R  & Falkenberg          & 434.6250  & -1,6MHz & 1750Hz          & QRT  & JO66FV \\
Repeater & FM     & SK6JX/R  & Falkenb./Arvidst.   & 145.6250  & -600KHz & 1750Hz/DTMF 1   & QRV  & JO66FV \\
Repeater & FM     & SA6AEG/R & Falköping           & 434.6000  & -1,6MHz & 1750Hz          & QRV  & JO68SE \\
Repeater & FM     & SK6HD/R  & Falköping           & 145.7250  & -600KHz & 1750Hz          & QRT  & JO68SE \\
Link     & FM     & SA6RP    & Floda               & 433.4750  &         & Carrier         & QRV  & JO67ET \\
Repeater & FM     & SK6GO/R  & Göteborg            & 145.7875  & -600KHz & 1750/114,8Hz    & QRV  & JO57XQ \\
Repeater & FM     & SK6GO/R  & Göteborg            & 145.6875  & -600KHz & 1750/118,8Hz    & QRT  & JO57XR \\
Repeater & FM     & SK6GO/R  & Göteborg            & 434.6500  & -2MHz   & 1750/114,8Hz    & QRT  & JO57XQ \\
Repeater & D-Star & SK6RKI-C & Göteborg            & 145.5875  & -600KHz & DV Carrier      & QRV  & JO67XQ \\
Repeater & FM     & SK6RFQ   & Göteborg            & 29.6800   & -100KHz & 1750/114,8Hz    & QRV  & JO57XQ \\
Repeater & D-Star & SK6SA-B  & Göteborg            & 434.5125  & -2MHz   & DV Carrier      & QRV  & JO57XQ \\
Hotspot  & D-Star & SE6H-B   & Göteborg            & 433.5250  & Simplex & DV Carrier      & QRV  & JO67AR \\
Repeater & DMR    & SG6DMR   & Göteborg            & 434.7625  & -2MHz   & 240699          & Plan & JO57XQ \\
Link     & FM     & SM6FZG   & Göteb./Aleklätt.    & 144.6125  & Simplex & 146,2Hz         & QRV  & JO67AV \\
Repeater & FM     & SK6RDG   & Göteb./Guldheden    & 434.9750  & -2MHz   & 1750/114,8Hz    & QRV  & JO57XQ \\
Repeater & FM     & SK6RKI   & Göteb./Guldheden    & 1297.1500 & -6MHz   & 1750            & QRV  & JO57XQ \\
Repeater & FM     & SK6RFQ   & Göteb./Guldheden    & 145.6500  & -600KHz & 1750/114,8 Hz   & QRV  & JO57XQ \\
Repeater & FM     & SK6RFQ   & Göteb./Guldheden    & 434.6000  & -2MHz   & 1750/114,8Hz    & QRV  & JO57XQ \\
Link     & FM     & SM6FZG   & Göteb./Guldheden    & 144.5750  & Simplex & 146,2Hz         & QRV  & JO57XQ \\
Repeater & FM     & SK6RKI   & Göteb./Hönö         & 145.7500  & -600KHz & 1750            & QRT  & JO57TQ \\
Repeater & FM     & SK6RKI   & Göteborg/Hönö       & 434.8500  & -1,6MHz & 1750            & QRT  & JO57TQ \\
Link     & FM     & SM6FZG   & Göteborg/Hönö       & 144.625   & Simplex & 146,2Hz         & QRV  & JO57TQ \\
Link     & FM     & SM6FZG   & Göteb./Kortedala    & 144.6000  & Simplex & 146,2Hz         & QRV  & JO67AS \\
Link     & FM     & SM6FZG   & Göteb./Landvetter   & 144.5625  & Simplex & 146,2Hz         & QRV  & JO67CQ \\
Link     & FM     & SM6FZG   & Göteb./Långedrag    & 144.5250  & Simplex & 146,2Hz         & QRV  & JO57WQ \\
Link     & FM     & SM6FZG   & Göteb./Mölnlycke    & 144.5875  & Simplex & 146,2Hz         & QRV  & JO67BP \\
Link     & FM     & SM6FZG   & Göteb./Skårsjön     & 144.5500  & Simplex & 146,2Hz         & QRV  & JO67AN \\
Link     & FM     & SM6FZG   & Göteb./Skårsjön     & 51.5500   & Simplex & 146,2Hz         & QRV  & JO67AN \\
Link     & FM     & SM6FZG   & Göteb./Skårsjön     & 29.5500   & Simplex & 146,2Hz         & QRV  & JO67AN \\
Repeater & FM     & SM6YOF/R & Göteb./Torslanda    & 434.9500  & -2MHz   & Carrier         & QRV  & JO57VS \\
Link     & FM     & SM6FZG   & Göteb./V. Frölunda  & 144.5375  & Simplex & 146,2 Hz        & QRV  & JO57XP \\
Link     & FM     & SM6CYJ   & Götene              & 145.3250  &         & 71,9 Hz         & QRV  & JO68RM \\
Repeater & FM     & SL6BH/R  & Halmstad            & 434.7500  & -2MHz   & 114,8 Hz        & QRV  & JO66KQ \\
Repeater & FM     & SK6RKG   & Halmstad            & 145.6750  & -600KHz & 114,8 Hz        & QRV  & JO66MS \\
Repeater & FM     & SK6RKG   & Halmstad            & 434.9250  & -2MHz   & 114,8 Hz        & QRV  & JO66MS \\
Hotspot  & D-Star & SG6RHB-B & Halmstad            & 433.4250  &         & DV Carrier      & QRV  & JO66LP \\
Hotspot  & D-Star & SG6JWU-C & Halmstad            & 145.3375  & Simplex & DV Carrier      & QRV  & JO66LP \\
Hotspot  & D-Star & SG6JWU-B & Halmstad            & 433.4750  & Simplex & DV Carrier      & QRV  & JO66LP \\
Repeater & FM     & SK6MA/R  & Hjo                 & 145.6375  & -600KHz & 1750            & QRV  & JO78DH \\
Hotspot  & D-Star & SK6MA-C  & Hjo                 & 145.2125  & Simplex & DV Carrier      & QRV  & JO78DH \\
Repeater & FM     & SK6BA/R  & Kinna               & 145.6000  & -600KHz & 1750/DTMF 1     & QRV  & JO67HM \\
Repeater & FM     & SK6BA/R  & Kinna               & 434.9500  & -2MHz   & 1750/DTMF 1     & QRV  & JO67HM \\
Hotspot  & D-Star & SK6BA-B  & Kinna               & 433.5625  &         & DV Carrier      & QRV  & JO67HL \\
Repeater & FM     & SK6QW/R  & Kinnekulle          & 434.9500  & -2MHz   & Carrier         & QRV  & JO68QO \\
Repeater & FM     & SK6ROY   & Kinnekulle          & 145.6000  & -600KHz & 1750/114,8Hz    & QRV  & JO68QO \\
Repeater & FM     & SK6RJW   & Kungsbacka          & 145.7250  & -600KHz & 1750/114,8Hz    & QRV  & JO67AL \\
Repeater & FM     & SK6RJW   & Kungsbacka          & 434.7250  & -2MHz   & 1750/114,8Hz    & QRV  & JO67AL \\
Repeater & FM     & SK6RPE   & Kungälv             & 145.6125  & -600KHz & 114,8 Hz        & Plan & JO57XU \\
Repeater & FM     & SK6RPE   & Kungälv             & 434.9000  & -2MHz   & 123,0 Hz        & QRV  & JO57XU \\
Repeater & FM     & SA6BXG/R & Kungälv/Romelanda   & 434.7375  & -2MHz   & 114,8 Hz        & QRV  & JO67AX \\
Repeater & FM     & SA6APY/R & Lidköping           & 434.7000  & -2MHz   & 118,8 Hz        & QRV  & JO68OM \\
Link     & FM     & SK6LR    & Lidköping           & 145.3500  &         & 118,8 Hz        & QRV  & JO68NM \\
Link     & FM     & SM6YRB   & Lidköping/Kållandsö & 145.3000  &         & 118,8 Hz        & QRV  & JO68NP \\
Hotspot  & D-Star & SK6KA-B  & Limmared            & 433.4375  &         & DV Carrier      & QRV  & JO67QM \\
Link     & FM     & SM6XIN   & Lundsbrunn          & 434.1250  &         & 118,8 Hz        & QRV  & JO68RK \\
Repeater & FM     & SK6IF/R  & Lysekil             & 145.6000  & -600KHz & 1750/118,8Hz    & QRV  & JO58TH \\
Repeater & FM     & SK6IF/R  & Lysekil             & 434.8000  & -2MHz   & 118,8 Hz        & QRV  & JO58RG \\
Repeater & D-Star & SK6IF-C  & Lysekil/Kungshamn   & 145.5750  & -600KHz & DV Carrier      & QRV  & JO58PI \\
Repeater & FM     & SK6QW/R  & Mariestad/Katrinef. & 434.9000  & -2MHz   & Carrier         & QRV  & JO68VQ \\
Repeater & FM     & SM6VBT/R & Mölndal             & 145.7000  & -600KHz & 118,8Hz         & QRV  & JO67AP \\
Repeater & FM     & SM6VBT/R & Mölndal             & 434.7000  & -2MHz   & 118,8Hz         & QRV  & JO67AP \\
Repeater & FM     & SM6VBT/R & Mölndal             & 29.6900   & -100KHz & 118,8Hz         & QRV  & JO67AP \\
Hotspot  & D-Star & SD6GB-B  & Mölndal             & 434.4625  &         & DV Carrier      & QRV  & JO67AQ \\
Hotspot  & D-Star & SD6GB-E  & Mölndal             & 51.5300   &         & DV Carrier      & QRV  & JO67AQ \\
Hotspot  & D-Star & SG6ZDO   & Sjövik              & 433.5750  &         & DV Carrier      & QRV  & JO67EV \\
Repeater & D-Star & SG6APY-B & Skara               & 434.5875  & -2MHz   & DV Carrier      & QRV  & JO68RJ \\
Repeater & FM     & SK6EI/R  & Skövde              & 145.6875  & -600KHz & 1750Hz          & QRT  & JO68VK \\
Repeater & FM     & SK6EI/R  & Skövde              & 434.8250  & -2MHz   & 114,8Hz         & QRV  & JO68VK \\
Repeater & FM     & SM6XTV/R & Stenungsund         & 145.7125  & -600KHz & 1750/114,8Hz    & QRV  & JO58VD \\
Repeater & D-Star & SK6QA-B  & Stenungsund         & 434.5375  & -2MHz   & DV Carrier      & QRV  & JO58UB \\
Repeater & FM     & SK6RWO   & Strömstad           & 434.8250  & -1,6MHz & 1750            & QRV  & JO58OW \\
Repeater & FM     & SK6RIP   & Tanumshede          & 145.6750  & -600KHz & 1750            & QRV  & JO58PR \\
Repeater & FM     & SK6DW/R  & Trollhättan         & 145.7625  & -600KHz & 114,8Hz         & QRV  & JO68DG \\
Repeater & FM     & SK6DW/R  & Trollhättan         & 434.8750  & -2MHz   & 114,8Hz         & QRV  & JO68BH \\
Repeater & D-Star & SK6DW-B  & Trollhättan         & 434.5250  & -2MHz   & DV Carrier      & QRV  & JO68DG \\
Repeater & D-Star & SL6ZAQ   & Uddevalla           & 434.5625  & -2MHz   & DV Carrier      & QRT  & JO58WH \\
Repeater & FM     & SK6GX/R  & Uddevalla           & 145.7375  & -600KHz & 114,8Hz         & QRV  & JO58WH \\
Repeater & DMR    & SL6ZAQ   & Uddevalla           & 145.6375  & -600KHz & 2406001/118,8Hz & QRV  & JO58WH \\
Repeater & FM     & SK6GX/R  & Uddevalla/Kungsh.   & 434.7750  & -2MHz   & 1750            & QRT  & JO58PI \\
Repeater & FM     & SM6UXW/R & Ulricehamn          & 434.6750  & -2MHz   & 118,8Hz         & QRV  & JO67RT \\
Repeater & FM     & SM6UXW/R & Ulricehamn          & 145.6750  & -600KHz & 118,8Hz         & QRV  & JO67ST \\
Repeater & FM     & SM6WVY/R & U-hamn/Blidsb.      & 434.6500  & -2MHz   & 114,8Hz         & QRT  & JO67RW \\
Link     & FM     &          & U-hamn/Tingsh.      & 434.6750  &         & 71,9Hz          & QRV  &        \\
Repeater & FM     & SK6DK/R  & Varberg             & 434.7000  & -1,6MHz & 1750            & QRV  & JO67EH \\
Repeater & FM     & SK6DK/R  & Varberg             & 145.7000  & -600KHz & 1750            & QRV  & JO67EH \\
Hotspot  & D-Star & SK6EP-C  & Varberg             & 145.3625  &         & DV Carrier      & QRT  & JO67BG \\
Repeater & FM     & SM6MFA/R & Vänersborg          & 434.7250  & -2MHz   & 1750            & QRV  & JO68DJ \\
Repeater & D-Star & SK6DZ-C  & Vårgårda            & 145.6625  & -600KHz & DV Carrier      & QRV  & JO68JA \\
Repeater & FM     & SK6DZ/R  & Vårgårda            & 434.8500  & -2MHz   & 118,8Hz         & QRV  & JO68JA

\end{longtable}
\normalsize
\subsubsection{Repeatrar distrikt 7}
\footnotesize
\begin{longtable}{llllrrlcl}

\textbf{Funktion} & \textbf{Typ} & \textbf{Call} & \textbf{QTH} & \textbf{Frekvens} & \textbf{Skift} & 
\textbf{Access} & \textbf{Status} & \textbf{Locator} \\ \hline \endhead

Repeater & FM     & SK7CA/R  & Algutsrum           & 145.6000  & -600KHz & 1750/79,7Hz  & QRV  & JO86GQ \\
Repeater & FM     & SK7CA/R  & Algutsrum           & 434.6000  & -2MHz   & 79,7Hz       & QRV  & JO86GQ \\
Repeater & FM     & SM7NTJ/R & Aneby               & 434.7250  & -2MHz   & 1750Hz       & QRV  & JO77HU \\
Repeater & FM     & SK7REZ   & Blentarp/Romeleås.  & 145.6750  & -600KHz & 79,7Hz       & QRV  & JO65TM \\
Repeater & FM     & SK7EM/R  & Blentarp/Romeleås.  & 434.8500  & -2MHz   & 79,7Hz       & QRV  & JO65SN \\
Repeater & DMR    & SM7SKI/R & Blentarp/Romeleås.  & 434.6125  & -2MHz   & DMR          & Plan & JO65TM \\
Repeater & FM     & SK7RN/R  & Borgholm            & 145.6625  & -600KHz & 1750Hz       & QRV  & JO86IU \\
Repeater & FM     & SK7RNO   & Borgholm/Köp:vik    & 434.6750  & -2MHz   & 1750Hz       & QRV  & JO86IU \\
Repeater & FM     & SK7RN/R  & Böda                & 434.9000  & -2MHz   & Carrier      & QRV  & JO87MG \\
Repeater & FM     & SK7UO/R  & Emmaboda            & 145.7750  & -600KHz & 1750Hz       & QRV  & JO76SP \\
Repeater & FM     & SM7GYT/R & Eslöv               & 434.8125  & -2MHz   & 88,5Hz       & QRV  & JO65PU \\
Repeater & D-Star & SK7RNQ-C & Gladsax             & 145.5875  & -600KHz & DV Carrier   & QRV  & JO75DN \\
Repeater & D-Star & SK7DS-B  & Glumslöv            & 434.5125  & -2MHz   & DV Carrier   & QRV  & JO65JW \\
Repeater & FM     & SK7RYR   & Gnosjö              & 145.6875  & -600KHz & 1750Hz       & QRV  & JO67UI \\
Repeater & FM     & SK7OL/R  & Hallandsåsen        & 145.7875  & -600KHz & 79,7Hz       & QRV  & JO66LJ \\
Repeater & D-Star & SK7RQX-B & Hallandsåsen        & 434.5750  & -2MHz   & DV Carrier   & QRV  & JO66LJ \\
Repeater & FM     & SK7CY    & Helsingborg         & 1297.2000 & -6MHz   & 1750Hz       & QRV  & JO66IB \\
Repeater & FM     & SK7REE   & H:borg/Söderåsen    & 145.6500  & -600KHz & 79,7Hz       & QRV  & JO66NB \\
Repeater & DMR/FM & SK7REE   & H:borg/Söderåsen    & 434.6500  & -2MHz   & DMR+ 240702  & QRV  & JO66NB \\
Repeater & FM     & SK7REE/R & H:borg/Söderåsen    & 51.8500   & -600KHz & 79,7Hz       & QRV  & JO66NB \\
Repeater & FM     & SK7GH/R  & Hillerstorp         & 434.6000  & -2MHz   & 1750         & QRV  & JO67WH \\
Link     & FM     & SM7KUY/R & Hosaby              & 145.4250  &         & 79,7Hz       & QRV  & JO76IA \\
Repeater & FM     & SK7RMV   & Hultsfred/Vimmerby  & 145.7625  & -600KHz & 1750         & QRV  & JO77WL \\
Repeater & FM     & SK7RBK   & Hässleh./Bjärnum    & 434.9500  & -2MHz   & 1750         & QRV  & JO66UH \\
Repeater & FM     & SA7ATK/R & Hässleh./Tormestorp & 434.9750  & -2MHz   & 79,7Hz       & QRV  & JO66UC \\
Repeater & FM     & SK7RBK   & Hässleh./Bjärnum    & 145.7625  & -600KHz & 1750         & QRV  & JO66UH \\
Repeater & FM     & SK7RGI   & Jönköping           & 434.7500  & -1,6MHz & 1750/DTMF 6  & QRV  & JO77CS \\
Repeater & FM     & SK7RGI   & Jönköping           & 29.6800   & -100KHz & 1750/DTMF 6  & QRV  & JO77BS \\
Link     & FM     & SM7TYU   & Jönköping           & 145.4000  &         & Carrier      & QRV  & JO77BS \\
Repeater & FM     & SK7RVZ   & Jönk./Huskvarna     & 145.7875  & -600KHz & 1750         & QRV  & JO77DT \\
Repeater & FM     & SK7RGI   & Jönk./Taberg        & 145.7500  & -600KHz & 1750         & QRV  & JO77CQ \\
Repeater & FM     & SM7JPI/R & Karlshamn           & 434.9250  & -2MHz   & 1750         & QRT  & JO76KE \\
Repeater & FM     & SK7RFJ   & Karlskrona          & 145.7500  & -600KHz & 1750         & QRV  & JO76TE \\
Repeater & FM     & SK7FK/R  & Karlskrona          & 434.7500  & -2MHz   & 1750         & QRV  & JO76TE \\
Repeater & FM     & SM7YWE/R & Karlskrona          & 434.8750  & -2MHz   & 79,7Hz       & QRV  & JO76RH \\
Repeater & FM     & SK7BQ/R  & Kristianstad        & 145.7375  & -600KHz & 79,7Hz       & QRV  & JO76AA \\
Repeater & FM     & SK7BQ/R  & Kristianstad        & 434.6000  & -2MHz   & 1750/79,7Hz  & QRV  & JO76CA \\
Repeater & D-Star & SG7TIX-B & Kristianstad        & 434.5250  & -2MHz   & DV Carrier   & QRV  & JO76DB \\
Repeater & D-Star & SK7RMQ-C & Linderöd            & 145.5750  & -600KHz & DV Carrier   & QRV  & JO65VW \\
Repeater & DMR    & SG7DMR   & Linderöd            & 434.5500  & -2MHz   & DMR240700    & QRV  & JO65VW \\
Repeater & FM     & SK7MO/R  & Ljungby             & 145.7250  & -600KHz & 1750Hz       & QRV  & JO66XT \\
Repeater & FM     & SK7RJL/R & Lund                & 434.7250  & -2MHz   & 79,7Hz       & QRV  & JO65OR \\
Repeater & DMR    & SK7RJL   & Lund                & 434.5875  & -2MHz   & DMR          & QRV  & JO65OR \\
Repeater & FM     & SM7XCO/R & Lyby                & 434.7000  & -2MHz   & 79,7Hz       & QRV  & JO65TU \\
Repeater & FM     & SK7RRV   & Lönsboda            & 434.9000  & -1,6MHz & 1750         & QRV  & JO76DJ \\
Hotspot  & D-Star & SK7RRV-C & Lönsboda            & 144.8125  &         & DV Carrier   & QRV  & JO76DJ \\
Repeater & FM     & SK7REP   & Malmö               & 434.7750  & -2MHz   & 1750         & QRV  & JO65MO \\
Repeater & FM     & SK7REP   & Malmö               & 145.7750  & -600KHz & 1750         & QRV  & JO65MO \\
Repeater & FM     & SK7REP   & Malmö               & 1297.1750 & -6MHz   & 1750         & QRV  & JO65MO \\
Link     & FM     & SK7DX    & Malmö               & 145.3000  &         & Simplex      & QRT  & JO65MO \\
Repeater & D-Star & SM7YKX-B & Malmö               & 434.9500  & -2MHz   & DV Carrier   & QRV  & JO65MN \\
Repeater & D-Star & SM7XAA   & Malmö               & 434.5250  & -2MHz   & DV Carrier   & QRV  & JO65MN \\
Repeater & FM     & SM7LNT/R & Mörrum              & 434.8250  & -2MHz   & 79,7Hz       & QRV  & JO76IE \\
Repeater & FM     & SL7ZXW/R & Nybro               & 145.6875  & -600KHz & 1750         & QRV  & JO76VQ \\
Repeater & FM     & SK7HR/R  & Nässjö              & 145.6500  & -600KHz & 1750         & QRV  & JO77IP \\
Repeater & FM     & SK7RFH   & Nässjö              & 434.8500  & -2MHz   & 1750/DTMF 6  & QRV  & JO77IP \\
Repeater & FM     & SK7RFH   & Nässjö              & 1297.0500 & -6MHz   & 1750         & QRT  & JO77IP \\
Repeater & FM     & SK7JC/R  & Olofström           & 434.6250  & -2MHz   & 1750         & QRT  & JO76KF \\
Repeater & FM     & SK7RGM   & O./Boafallsbacke    & 145.7000  & -600KHz & 79,7Hz       & QRV  & JO76FF \\
Repeater & FM     & SK7RIH   & Oskarshamn          & 145.7250  & -600KHz & 1750         & QRV  & JO87FG \\
Repeater & FM     & SK7RIH/R & Oskarshamn          & 434.7250  & -2MHz   & 1750         & QRV  & JO87EG \\
Repeater & FM     & SK7RIH   & Oskarshamn          & 51.9100   & -600KHz & 1750         & QRV  & JO87EG \\
Repeater & FM     & SA7CDA/R & Simrishamn          & 434.6375  & -2MHz   & 79,7Hz       & QRV  & JO75CL \\
Repeater & D-Star & SK7RDS-C & Svedala             & 145.5687  & -600KHz & DV Carrier   & QRV  & JO65OL \\
Hotspot  & D-Star & SK7JL-B  & Tving               & 433.4750  &         & DV Carrier   & QRV  & JO76RH \\
Repeater & FM     & SK7IJ/R  & Vetlanda            & 145.6250  & -600KHz & 1750/156,7Hz & QRV  & JO77OL \\
Repeater & FM     & SK7IJ/R  & Vetlanda            & 434.6250  & -2MHz   & 156,7Hz      & QRV  & JO77OL \\
Repeater & FM     & SK7RNQ   & Vitaby              & 145.6125  & -600KHz & 79,7Hz       & QRV  & JO75   \\
Repeater & FM     & SK7GH/R  & Värnamo             & 145.6000  & -600KHz & 1750/156,7Hz & QRV  & JO77BF \\
Repeater & FM     & SK7JD/R  & Västervik           & 145.6750  & -600KHz & Carrier      & QRV  & JO87HS \\
Repeater & D-Star & SK7HW-B  & Växjö               & 434.7000  & -2MHz   & DV Carrier   & QRV  & JO76KU

\end{longtable}
\normalsize

\end{landscape}



\begin{landscape}
\subsection{Bandplaner VHF--UHF}
\subsubsection{Bandplan 6m 50--52 MHz}
\begin{tabular}{rrrll}

\textbf{Frekvens} &  & \textbf{BW} & \textbf{Trafik} & \textbf{Noteringar} \\ \hline

50.000 & 50.100 & 500 Hz  & CW          & \textbf{CW anrp. 50.050 och 50.090 (interkont.)}             \\ \hline
50.100 & 50.130 & 2.7 kHz & CW, SSB     & Interkontinental DX-trafik. Ej QSO inom Europa               \\ \hline
50.100 & 50.200 & 2.7 kHz & CW,SSB      & DX 50.110--50.130, \textbf{50.110 50.150 anrop (interkont.)} \\ \hline
50.200 & 50.300 & 2.7 kHz & CW,SSB      & Generell användning. 50.285 för crossband                    \\ \hline
50.300 & 50.400 & 2.7 kHz & CW, MGM     & PSK 50.305, EME 50.310 – 50.320                              \\
       &        &         &             & MS 50.350 – 50.380                                           \\ \hline
50.400 & 50.500 & 1 kHz   & CW, MGM     & Endast fyrar, 50.401 ±500 Hz WSPR-fyrar                      \\ \hline
51.210 & 51.390 & 12 kHz  & FM          & Repeater Repeater in, 20/10 kHz kanalavstånd                 \\
       &        &         &             & RF81 – RF99                                                  \\ \hline
50.500 & 52.000 & 12 kHz  & Alla moder  & SSTV 50.510, RTTY 50.600, FM 51.510                          \\ \hline
51.810 & 51.990 & 12 kHz  & FM Repeater & Repeater ut, 20/10 kHz kanalavstånd                          \\
       &        &         &             & RF81 – RF99                                                  \\ \hline

\end{tabular}
\clearpage
\subsubsection{Bandplan 2m 144--146 MHz}
\begin{tabular}{rrrll}

\textbf{Frekvens} &  & \textbf{BW} & \textbf{Trafik} & \textbf{Noteringar} \\ \hline

144.0000 & 144.1100  & 500 Hz  & CW, EME      & \textbf{CW anrop 144.050}               \\
         &           &         &              & MS random 144.100                       \\ \hline
144.1100 & 144.1500  & 500 Hz  & CW, MGM      & EME MGM 144.120--144.160                \\
         &           &         &              & PSK31 cent. 144.138                     \\ \hline
144.1500 & 144.1800  & 2.7 kHz & CW, SSB, MGM & EME 144.150--144.160                    \\
         &           &         &              & MGM 144.160--144.180 anrop 144.170      \\ \hline
144.1800 & 144.3600  & 2.7 kHz & CW, SSB, MGM & MS SSB random 144.195--144.205          \\
         &           &         &              & \textbf{SSB anrop 144.300}              \\ \hline
144.3600 & 144.3990  & 2.7 kHz & CW, SSB, MGM & MS MGM random anrop 144.370             \\ \hline
144.4000 & 144.4900  & 500 Hz  & Fyr          & Exklusivt segment fyrar, ej QSO         \\ \hline
144.5000 & 144.7940  & 20 kHz  & All mode     & SSTV, RTTY, FAX, ATV                    \\
         &           &         &              & Linjära transpondrar                    \\ \hline
144.7940 & 144.9625  & 12 kHz  & MGM          & APRS 144.800                            \\ \hline
144.9750 & 145.19350 & 12 kHz  & FM, DV       & Rpt in 144.975--145.1935                \\
         &           &         &              & RV46–-RV63, 12.5 kHz, 600 kHz skift     \\ \hline
145.1940 & 145.2060  & 12 kHz  & FM rymd      & 145.200 för kom. m. bem. rymdfark.      \\ \hline
145.2060 & 145.5625  & 12 kHz  & FM, DV       & FM 145.2125-–145.5875  V17–V47          \\
         &           &         &              & \textbf{FM anrop 145.500}, RTTY 145.300 \\
         &           &         &              & FM simpl. INET GW 145.2375, 2875, 3375  \\
         &           &         &              & DV anrop 145.375                        \\ \hline
145.5750 & 145.7935  & 12 kHz  & FM, DV       & Rpt ut 145.575--145.7875                \\
         &           &         &              & RV46–RV63, 12.5 kHz kanalavstånd        \\ \hline
145.794  & 145.806   & 12 kHz  & FM Rymd      & 145.800, 145.200 dplx m. bem. rymdfark. \\ \hline
145.806  & 146.000   & 12 kHz  & All mode     & Exklusivt satellit                      \\ \hline

\end{tabular}
\clearpage
\subsubsection{Bandplan 70cm 432--438 MHz}
\begin{tabular}{rrrll}
	\textbf{Frekvens}         &               & \textbf{BW} & \textbf{Trafik} & \textbf{Anmärkning}                                          \\ \hline
	         432.0000         & 432.0250      & 500 Hz      & CW              & EME exklusivt.                                               \\ \hline
	         432.0250         & 432.1000      & 500 Hz      & CW, PSK31       & CW mellan 432.000--085, \textbf{CW anrop 432.050}            \\
                                  &               &             &                 & PSK31 432.088                                                \\ \hline
	         432.1000         & 432.3990      & 2.7 kHz     & CW, SSB, MGM    & \textbf{SSB anrop 432.200}                                   \\
                                  &               &             &                 & Mikrovåg talkback 432.350, FSK441 432.370                    \\ \hline
	         432.4000         & 432.4900      & 500 Hz      & Fyr             & Exklusivt segment för fyrar                                  \\ \hline
	         432.5000         & 432.5940      & 12 kHz      & All mode        & Linjära transpondrar IN 432.500--600                         \\ \hline
	         432.5000         & 432.5750      & 12 kHz      & All mode        & NRAU Digital rep. in 432.500--575 2 MHz skift                \\ \hline
	         432.5940         & 432.9940      & 12 kHz      & All mode        & Linjära transpondrar ut 432.600--800                         \\ \hline
	         432.5940         & 432.9940      & 12 kHz      & FM              & Rep. in 432.600--975 RU368--398 2 MHz skift                  \\ \hline
	         432.9940         & 433.3810      & 12 kHz      & FM              & Rep. in 433.000--375 RU368--398 1.6 MHz skift                \\ \hline
	         433.3940         & 433.5810      & 12 kHz      & FM              & SSTV (FM/AFSK) 433.400                                       \\
                                  &               &             &                 & FM simplex U272--286 \textbf{anrop 433.500}                  \\ \hline
	         433.6000         & 434.0000      & 20 kHz      & All mode        & RTTY (FM/AFSK) 433.600                                       \\
                                  &               &             &                 & FAX 433.700, APRS 433.800                                    \\ \hline
	         434.0000         & 434.4940      & 20 kHz      & All mode        & NRAU Dig. kanaler 433.450, 434.475                           \\ \hline
	    $^1$ 434.5000         & $^1$ 434.5940 & 20 kHz      & All mode        & NRAU Dig. rep. ut 434.500--575, 2 MHz skift                  \\ \hline
	         434.5940         & 434.9810      & 12 kHz      & FM              & NRAU Rep. ut 434.600--975 RU 368--RU398                      \\
                                  &               &             &                 & 12,5 kHz med 2 MHz skift                                     \\ \hline
	          435.000         & 438.000       & 20 kHz      & All mode        & Exklusivt satellit
\end{tabular}
%\clearpage
\subsubsection{Bandplan 23cm 1240--1300 MHz}
\begin{tabular}{rrrll}
	\textbf{Frekvens}         &               & \textbf{BW} & \textbf{Trafik} & \textbf{Anmärkning}                                          \\ \hline
	         1240.000         & 1243.250      & 20 kHz      & Alla moder      & 1240.000 - 1241.000 Digital kommunikation                    \\ \hline
	         1243.250         & 1260.000      & 20 kHz      & ATV och Data    & Repeater ut 1258.150-1259.350, R20--68                       \\ \hline
	         1260.000         & 1270.000      & 12 kHz      & Satellit        & Endast för satelliter alla moder                             \\ \hline
	         1270.000         & 1272.000      & 20 kHz      & Alla moder      & Repeater in, 1270.025-1270.700, RS1--28                      \\
                                  &               &             &                 & Packet RS29--50                                              \\ \hline
	         1272.000         & 1290.994      & 20 kHz      & ATV och Data    & Amatörtelevision ATV                                         \\ \hline
	         1290.994         & 1291.481      & 20 kHz      & FM och DV       & Repeater in Repeat. in 1291.000--1291.475                    \\
                                  &               &             &                 & RM0 – RM19, 25 kHz, 6 MHz skift                              \\ \hline
	         1291.494         & 1296.000      & 12 kHz      & Alla moder      &                                                              \\ \hline
	         1296.000         & 1296.150      & 500 Hz      & CW,  MGM        & EME 1296.000--025, \textbf{CW anrop 1296.050}                \\
                                  &               &             &                 & PSK31 1296.138 MHz                                           \\ \hline
	         1296.150         & 1296.400      & 2.7 kHz     & CW, SSB, MGM    & \textbf{SSB anrop 1296.200}                                  \\
                                  &               &             &                 & \textbf{FSK441 MS anrop 1296.370}                            \\ \hline
	         1296.400         & 1296.600      & 2.7 kHz     & CW, SSB, MGM    & Linjära transpondrar infrekvens                              \\ \hline
	         1296.600         & 1296.800      & 2.7 kHz     & CW, SSB, MGM    & SSTV/FAX 1296.500, MGM/RTTY 1296.600                         \\ \hline
	         1296.600         & 1296.800      & 2.7 kHz     & CW, SSB, MGM    & Linjära transpondrar utfrekvens                              \\
                                  &               &             &                 & 1296.750-.800 lokala fyrar max 10 W                          \\ \hline
	         1296.800         & 1296.994      & 500 Hz      & Fyrar           & Exklusivt segment för fyrar                                  \\ \hline
	         1296.994         & 1297.481      & 20 kHz      & FM              & Repeater ut Repeater ut 1297.000--1297.475                   \\
                                  &               &             &                 & RM0 – RM19, 25 kHz, 6 MHz skift                              \\ \hline
	         1297.494         & 1297.981      & 20 kHz      & FM simplex      & Simplex 25 kHz kanaler SM20--39                              \\
                                  &               &             &                 & \textbf{FM anrop 1297.500 SM20}                              \\ \hline
	         1299.000         & 1299.750      & 150 kHz     & Alla moder      & 5 st 150 kHz kanaler för DD,                                 \\
                                  &               &             &                 & 1299.075, 225, 375, 525, och 675 $\pm$75 kHz                 \\ \hline
	         1299.750         & 1300.000      & 20 kHz      & Alla moder      & 8 st FM/DV 25 kHz kan. 1299.775--1299.975
\end{tabular}
\end{landscape}

