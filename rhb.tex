% !TeX encoding = UTF-8
% !TeX spellcheck = sv_SE
\documentclass[10pt,swedish,a4paper,twoside]{article}
\usepackage[utf8]{inputenc}
\usepackage[swedish]{babel}
\usepackage{longtable}
\usepackage[a5paper,textwidth=12cm,
	textheight=17cm,bindingoffset=1cm]{geometry}
\usepackage{amsmath}
\usepackage{pdflscape}
\usepackage{longtable}
\usepackage{textcomp}

\begin{document}
\title{Trafikhandbok för Amatörradio\\
VHF/UHF}
\author{SM0UEI}
\maketitle

\begin{abstract}
Handbok för radiooperatörer innehåller diverse matnyttig information för amatörradiooperatörer i Sverige. Tanken är att samla information som t.ex. är bra att ha på resande fot eller annars som man saknar när man behöver den och inte har tillgång till Internet.

Skriv ut den på A4 med två sidor på samma papper och vik ihop till ett häfte. Ladda ner den i läsplattan eller surfplattan. Papperstypen är vald som A5 för att underlätta läsning på platta och mobiltelefon till och med.

Idéer tankar, korrigeringar och annat som behöver vara med är väl\-kom\-met att du skickar till anders@sikvall.se så kan jag se till att få med det i nästa utgåva. Denna bok uppdateras sporadiskt när det finns skäl och tid till detta.

Till en början är det VHF och UHF men kommer så småningom utökas med fler och fler band.
\end{abstract}

\clearpage
\tableofcontents
\clearpage

\setlength{\parskip}{0.5em}
\setlength{\parindent}{0pt}

% % % % % % % % % % % % % % % % % % % % % % %

\section{Signaler och distrikt}

\subsection{Svenska signaler}

Svenska signaler förekommer inom ett antal prefix. Enligt ITU disponerar Sverige förljande signalserier:
7SA--7SZ samt 8SA--8SZ och vidare de mer kända SAA--SMZ. Dessa har används till varierande ändamål, exempelvis har flyget signaler i serien SE-AAA--ZZZ. Polisen har tidigare använt signaler i serien SHA plus fyra siffror, detta är nu ersatt med nytt system i.o.m. RAKEL. Räddningstjänsten använde SDA med fyra siffror. Signaler som 7SA + 4 siffror används för mindre yrkesbåtar SC + 4 siffror för fritidsbåtar.

Amatörradion använder ett antal signaler, de viktigaste är:

\begin{tabular}{ll}
	SM & Amatörradiosignal utdelad av PTS (nya signaler tilldelas ej i serien) \\
	SA & Amatörradiosignal tilldelad av SSA, ESA eller FRO                     \\
	SK & Klubbsignaler (som regel tvåställiga efter distriktsiffran)           \\
	SL & Militära signaler (som regel tvåställiga efter distriktsiffran)
\end{tabular}

Dessa signaler följs av en \textit{distriktsiffra} se särskilt avsnitt och sedan 2-ställiga eller 3-ställiga bokstavskombinationer som är den personliga signalen. Exempel är SM0UEI som är min egen signal, distriktsiffran är 0 dvs hemmavarande i Stockholms län. Ett annat exempel kan vara SK5JV tidigare Fagersta amatörradioklubb.

Repeatrar som tillhör klubbar får ofta signal efter klubben med tillägg /R för repeater.

Det finns numera även ett stort antal signaler som är tillfälliga eller knutna till särskilda event, exempelvis scoutverksamhet som ibland sänder amatörradio och särskilda forskningsfartyg, flyg- och rymdfart mm.

Som suffix används följande:

\begin{tabular}{ll}
	/M  & Mobil (rörlig) sändaramatör, även portabel \\
	/MM & Mobil till sjöss (mobil maritime)          \\
	/AE & Mobil i luften (aeromobile)                \\
	/R  & Repeaterstation
\end{tabular}

\subsection{Utländska signaler}

Det är svårare att veta exakt hur utländska signaler är uppbyggda. Eftersom denna handledning är för VHF/UHF tar vi inte med alla DXCC-signaler utan de som är närmast oss i Sverige. Konsultera den fulla listan från ITU eller SSA för mer information.

\begin{tabular}{cl|cl|cl}
	\textbf{Signal} & \textbf{Land} & \textbf{Signal} & \textbf{Land} & \textbf{Signal} & \textbf{Land} \\ \hline
	   AMA--AOZ     & Spanien       &    C3A--C3Z     & Andorra       &    C4A--C4Z     & Cypern        \\
	   DAA--DRZ     & Tyskland      &    EAA--EHZ     & Spanien       &    EIA--EJZ     & Irland        \\
	   EKA--EKZ     & Armenien      &    EMA--EOZ     & Ukraina       &    EBA--ERZ     & Moldav.       \\
	   ESA--ESZ     & Estland       &    EUA--EWZ     & Vitryssl.     &    EXA--EXZ     & Ryssland      \\
	   EYA--EYZ     & Tajikist.     &    EZA--EZZ     & Turkmenist.   &    FAA--FZZ     & Frankrike     \\
	   GAA--GZZ     & Storbrit.     &    HAA--HZZ     & Ungern        &    HBA--HBZ     & Schweiz       \\
	   HEA--HEZ     & Schweiz       &    HFA--HFZ     & Polen         &    HGA--HGZ     & Ungern        \\
	   HVA--HVZ     & Vatikan.      &    HWA--HYZ     & Frankrike     &    H2A--H2Z     & Cypern        \\
	   IAA--IZZ     & Italien       &    JWA--JXZ     & Norge         &    J4A--J4Z     & Grekland      \\
	   LAA--LNZ     & Norge         &    LXA--LXZ     & Luxemburg     &    LYA--LYZ     & Litauen       \\
	   LZA--LZZ     & Bulgarien     &    MAA--MZZ     & Storbrit.     &    OEA--OEZ     & Österrike     \\
	   OFA--OJZ     & Finland       &    OKA--OLZ     & Tjeckien      &    OMA--OMZ     & Slovakien.
\end{tabular}

\subsection{Distriktsiffror}

Sverige delas in i följande distrikt efter sina län:

%\small
\begin{tabular}{cl}
	\textbf{Dist.} & \textbf{Län}                                     \\ \hline
	      0        & Stockholm                                        \\
	      1        & Gotland                                          \\
	      2        & Västerbotten, Norrbotten                         \\
	      3        & Gävleborg, Jämtland, Västernorrland              \\
	      4        & Örebro, Värmland, Dalarna                        \\
	      5        & Östergötland, Södermanland, Västmanland, Uppsala \\
	      6        & Halland, Västra götaland                         \\
	      7        & Skåne, Blekinge, Kronoberg, Jönköping, Kalmar    \\
	      8        & Speciella stationer utanför landets gränser
\end{tabular}

\section{Terminologi och trafik}
\subsection{Q-koder}
\begin{longtable}{ll}
	\textbf{Kod} & \textbf{Fråga / Svar}                                                 \\ \hline
	QRA          & Vad heter er station?                                                 \\
	             & Vår station heter ...                                                 \\ \hline
	QRB          & Hur långt bort från min station befinner ni er?                       \\
	             & Avståndet mellan oss är ungefär ...                                   \\ \hline
	QRG          & Kan ni ange min exakta frekvens?                                      \\
	             & Er exakta frekvens är ... (MHz/kHz)                                   \\ \hline
	QRH          & Varierar min frekvens/våglängd?                                       \\
	             & Er frekvens/våglängd varierar.                                        \\ \hline
	QRI          & Hur är min sändningston (CW)?                                         \\
	             & Er sändningston är 1--God, 2--Varierande, 3--Dålig                    \\ \hline
	QRK          & Vilken uppfattbarhet har mina signaler?                               \\
	             & Uppfattbarheten hos dina signaler är:                                 \\
	             & 1--Dålig, 2--Bristfällig, 3--Ganska god, 4--God, 5--Utmärkt           \\ \hline
	QRL          & Är ni upptagen?                                                       \\
	             & Jag är upptagen med ... (namn/signal) stör ej.                        \\ \hline
	QRM          & Är ni störd av annan station?                                         \\
	             & Störningarna är:                                                      \\
	             & 1--Obef., 2--Svaga, 3--Måttliga, 4--Starka, 5--Mycket starka          \\ \hline
	QRN          & Besväras ni av atmosfäriska störningar?                               \\
	             & Störningarna är:                                                      \\
	             & 1--Obef., 2--Svaga, 3--Måttliga, 4--Starka, 5--Mycket starka          \\ \hline
	QRO          & Kan jag (ska jag) öka sändareffekten?                                 \\
	             & Öka sändareffekten.                                                   \\ \hline
	QRP          & Kan jag (ska jag) minska sändareffekten?                              \\
	             & Minska sändareffekten.                                                \\ \hline
	QRQ          & Kan jag (får jag) öka sändningshastigheten?                           \\
	             & Öka sändningshastigheten.                                             \\ \hline
	QRS          & Kan jag (skall jag) sända långsammare?                                \\
	             & Sänd långsammare.                                                     \\ \hline
	QRT          & Skall jag avbryta sändningen?                                         \\
	             & Avbryt sändningen                                                     \\ \hline
	QRU          & Har ni något till mig?                                                \\
	             & Jag har inget till er. Se även QTC.                                   \\ \hline
	QRV          & Är ni redo?                                                           \\
	             & Jag är redo.                                                          \\ \hline
	QRX          & När anropar ni mig härnäst?                                           \\
	             & Jag anropar er kl ... (på ... MHz/kHz)                                \\ \hline
	QRZ          & Vem anropar mig?                                                      \\
	             & Ni anropas av ... (på ... MHz/kHz).                                   \\ \hline
	QSA          & Vilken styrka har mina signaler?                                      \\
	             & Era signaler är:                                                      \\
	             & 1--Ej uppf., 2--Svaga, 3--Ganska starka, 4--Starka, 5--Mycket starka  \\ \hline
	QSB          & Svajar styrkan på mina signaler?                                      \\
	             & Styrkan på era signaler svajar.                                       \\ \hline
	QSK          & Kan du höra mig mellan dina tecken och får jag avbryta dig? \\
	             & Jag kan höra dig mellan mina tecken och du får avbryta.               \\ \hline
	QSL          & Kan ni ge mig kvittens?                                               \\
	             & Jag kvitterar.                                                        \\ \hline
	QSO          & Ha ni förbindelse med ... eller ... (förmedlat)?                      \\
	             & Jag har förbindelse med ... (via ...)                                 \\ \hline
	QST          & Har tidigare använts som allmänt anrop men ersatts av CQ              \\ \hline
	QSY          & Skall jag övergå till att sända på annan frekvens?                    \\
	             & Gå över till att sända på annan frekvens (eller ... kHz/MHz).         \\ \hline
	QTC          & Hur många telegram har ni att sända?                                  \\
	             & Jag har ... telegram till dig (eller ...).                            \\ \hline
	QTH          & Vilken är er geografiska position?                                    \\
	             & Min geografiska position är ...                                       \\ \hline
	QTR          & Kan ni ge mig rätt tid?                                               \\
	             & Rätt tid är ...                                                       \\ \hline
\end{longtable}

\subsection{Bokstaveringsalfabetet (Svenska)}

\begin{longtable}{cl|cl|cl }
	A & Adam   & O & Olof    & 1 & Ett \\
	B & Bertil & P & Petter  & 2 & Tvåa \\
	C & Cesar  & Q & Qvintus & 3 & Trea \\
	D & David  & R & Rudolf  & 4 & Fyra \\
	E & Erik   & S & Sigurd  & 5 & Femma \\
	F & Filip  & T & Tore    & 6 & Sexa \\
	G & Gustav & U & Urban   & 7 & Sju \\
	H & Helge  & V & Viktor  & 8 & Åtta \\
	I & Ivar   & W & Wilhelm & 9 & Nia \\
	J & Johan  & X & Xerxes  & 0 & Nolla \\
	K & Kalle  & Y & Yngve   & . & Punkt\\
	L & Ludvig & Z & Zäta    & , & Komma\\
	M & Martin & Å & Åke     & - & Minus\\
	N & Niklas & Ä & Ärlig   & + & Plus\\
	  &        & Ö & Östen   &   & Mellanslag\\
\end{longtable}

\subsection{Bokstaveringsalfabetet (Internationella)}
\begin{longtable}{cl|cl|cl}
	A & Alfa     &  P   & Papa       & 0 & Zero    \\
	B & Bravo    &  Q   & Quebec     & 1 & One     \\
	C & Charlie  &  R   & Romeo      & 2 & Two     \\
	D & Delta    &  S   & Sierra     & 3 & Tree    \\
	E & Echo     &  T   & Tango      & 4 & Fower   \\
	F & Foxtrot  &  U   & Uniform    & 5 & Fife    \\
	G & Golf     &  V   & Victor     & 6 & Six     \\
	H & Hotel    &  W   & Whiskey    & 7 & Seven   \\
	I & India    &  X   & X-ray      & 8 & Ait     \\
	J & Juliet   &  Y   & Yankee     & 9 & Niner   \\
	K & Kilo     &  Z   & Zulu       & . & Stop    \\
	L & Lima     & Å/AA & Alfa-Alfa  & , & Decimal \\
	M & Mike     & Ä/AE & Alfa-Echo  & - & Minus   \\
	N & November & Ö/OE & Oscar-Echo & + & Plus    \\
	O & Oscar    &      &            &   & Space
\end{longtable}

\section{Signal och brus}

Effekter anges i W eller i decibel relaterat till 1 mW (dBm) eller relaterat 1W (dBW). Tabell över effekt och decibelwatt nedan:
\begin{center}
\begin{tabular}{rrr|rrr}
	            \textbf{W} & \textbf{dBW} & \textbf{dBm} &    \textbf{W} & \textbf{dBW} & \textbf{dBm} \\ \hline
	  1 \textmu W & -60 & -30 &  1 W &   0 &  30 \\
	 10 \textmu W & -50 & -20 &  3 W &   5 &  35 \\
	100 \textmu W & -40 & -10 &  5 W &   7 &  37 \\
	         1 mW & -30 &   0 & 10 W &  10 &  40 \\
	        10 mW & -20 &  10 & 20 W &  13 &  43 \\
	       100 mW & -10 &  20 & 50 W &  17 &  47
\end{tabular}
\end{center}

\subsection{S-värden}

Signalstyrkan i amatörradio uttrycks oftast som S-värden. Dessa fås i regel genom nivån på AGC hos mottagaren. Därför ser man sälla utslag vid riktigt låga signaler.

Standard kalibrering för S-metern är enligt följande skala:

\begin{center}
\begin{tabular}{r|rr|rr}
	     & \multicolumn{2}{c|}{$<$ 30 MHz} & \multicolumn{2}{c}{$>$ 30 MHz} \\
	   S &  dBm &                \textmu V &  dBm &               \textmu V \\ \hline
	   1 & -121 &                     0.21 & -141 &                    0.02 \\
	   2 & -115 &                     0.40 & -135 &                    0.04 \\
	   3 & -109 &                     0.80 & -129 &                    0.08 \\
	   4 & -103 &                     1.60 & -123 &                    0.16 \\
	   5 &  -97 &                     3.20 & -117 &                    0.32 \\
	   6 &  -91 &                     6.30 & -111 &                    0.63 \\
	   7 &  -85 &                    12.60 & -105 &                    1.26 \\
	   8 &  -79 &                    25.00 &  -99 &                    2.50 \\
	   9 &  -73 &                    50.00 &  -93 &                    5.00 \\
	9+10 &  -63 &                      160 &  -83 &                      16 \\
	9+20 &  -53 &                      500 &  -73 &                      50 \\
	9+30 &  -43 &                     1600 &  -63 &                     160 \\
	9+40 &  -33 &                     5000 &  -53 &                     500
\end{tabular}
\end{center}
\subsection{Termiska brusgolvet}
Termiska brusgolvet vid olika mottagarbandbredder beskrivs i tabellen. Generellt kan man beräkna brusgolvet genom följande formel:

$$N_0=k_BTB$$

Där $k_B$ är boltzmanns konstant $>>>>>>$

\begin{center}
\begin{tabular}{rr|rr|rr}
	\textbf{RBW} & \textbf{N$_0$} & \textbf{RBW} & \textbf{N$_0$} & \textbf{RBW} & \textbf{N$_0$} \\ \hline
	         0.5 &           -141 &         6.25 &           -136 &          100 &           -124 \\
	         1.0 &           -144 &        12.50 &           -133 &          200 &           -121 \\
	         3.0 &           -139 &        25.00 &           -130 &         5000 &           -107 \\
	         5.0 &           -137 &        50.00 &           -127 &        10000 &           -104
\end{tabular}
\end{center}

Mottagarbandbredden (RBW) anges i kHz och brusgolvet i dBm (dB relaterat en styrka om 1 mW).


\section{Frekvenser}

Dessa frekvenser är avsedda för allmänhet eller för specifika ända\-mål som anges. Det innebär att de kan brukas för de ändamål som anges i PTS för\-fatt\-nings\-sam\-ling\-ar och sammanställning över ej tillståndspliktiga frekvenser. Observera att du är skyldig att själv kontrollera bestämmelserna innan en frekvens brukas.

Effekten i tabellen är ustrålad effekt PEP om inte annat anges.

\subsection{Jakt och jordbruksfrekvenser 155 MHz}
\begin{tabular}{rlrl}
	\textbf{Frekvens} & \textbf{Benämning} & \textbf{Effekt} & \textbf{Användningsområde} \\ \hline
	          155,425 & Jakt K1            &             5 W & Jakt, Jordbruk [M]         \\
	          155,475 & Jakt K2            &             5 W & Jakt, Jordbruk [M]         \\
	          155,475 & Jakt K3            &             5 W & Jakt, Jordbruk [M]         \\
	          155,525 & Jakt K4            &             5 W & Jakt, Jordbruk [M]         \\
	          156,000 & Jakt K5            &             5 W & Jakt, Jordbruk, PMR        \\
	          155,400 & Jakt K6            &             5 W & Jakt, Jordbruk [M]         \\
	          155,450 & Jakt K7            &             5 W & Jakt, Jordbruk [M]
\end{tabular}

Jakt K5 är öppen att användas för andra ändamål och sammanfaller ej med marina VHF-bandet vilket de andra gör.


\subsection{Öppna PMR-bandet på 446 MHz}
\begin{tabular}{rlrl}
	\textbf{Frekvens} & \textbf{Benämning} & \textbf{Effekt} & \textbf{Användningsområde} \\ \hline
	        446,00625 & PMR446 K1          &          500 mW & PMR [N]                    \\
	        446,01875 & PMR446 K2          &          500 mW & PMR [N]                    \\
	        446,03125 & PMR446 K3          &          500 mW & PMR [N]                    \\
	        446,04375 & PMR446 K4          &          500 mW & PMR [N]                    \\
	        446,05625 & PMR446 K5          &          500 mW & PMR [N]                    \\
	        446,06875 & PMR446 K6          &          500 mW & PMR [N]                    \\
	        446,08125 & PMR446 K7          &          500 mW & PMR [N]                    \\
	        446,09375 & PMR446 K8          &          500 mW & PMR [N]
\end{tabular}

[M] Delas med marin VHF-radio. Får därför inte användas till sjöss inom landets gränser eller i svenska territorialvatten.

[N] Smalbandig FM-modulation skall användas pga tätt liggande kanaler.

\subsection{Kortdistansradio (KDR)}

Kallas även SRBR för Short Range Business Radio.

\begin{tabular}{rlrl}
	\textbf{Frekvens} & \textbf{Benämning} & \textbf{Effekt} & \textbf{Användningsområde} \\ \hline
	          444,600 & SRBR K1            &             2 W & Short range business radio \\
	          444,625 & SRBR K2            &             2 W & Short range business radio \\
	          444,800 & SRBR K3            &             2 W & Short range business radio \\
	          444,825 & SRBR K4            &             2 W & Short range business radio \\
	          444,850 & SRBR K5            &             2 W & Short range business radio \\
	          444,875 & SRBR K6            &             2 W & Short range business radio \\
	          444,925 & SRBR K7            &             2 W & Short range business radio \\
	          444,975 & SRBR K8            &             2 W & Short range business radio
\end{tabular}

SRBR är ett ej tillståndspliktigt frekvenssegment som används för yrkesmässig radiotrafik.

Rekommendationen är att man skall använda CTCSS eller motsvarande för att undvika störa och bli störd av andra stationer som delar frekvenserna.

\subsection{CTCSS subtoner}

På PMR446 och andra band samt för ett antal repeatrar på 70cm primärt men även ibland 2m och andra band används subtoner för att skapa virtuella grupper och sub-kanaler. De som används är följande toner och frekvenser:

\begin{tabular}{rr|rr|rr|rr|rr}
	 1 &  67,0 &  2 &  69,3 &  3 &  74,4 &  4 &  77,0 &  5 &  79,7 \\ \hline
	 6 &  82,5 &  7 &  85,4 &  8 &  88,5 &  9 &  91,5 & 10 &  94,8 \\ \hline
	11 &  97,4 & 12 & 100,0 & 13 & 103,5 & 14 & 107,2 & 15 & 110,9 \\ \hline
	16 & 114,8 & 17 & 118,8 & 18 & 123,0 & 19 & 127,3 & 20 & 131,8 \\ \hline
	21 & 136,5 & 22 & 141,3 & 23 & 146,2 & 24 & 151,4 & 25 & 156,7 \\ \hline
	26 & 162,2 & 27 & 167,9 & 28 & 173,8 & 29 & 179,9 & 30 & 186,2 \\ \hline
	31 & 192,8 & 32 & 203,5 & 33 & 210,7 & 34 & 218,1 & 35 & 225,7 \\ \hline
	36 & 233,6 & 37 & 241,8 & 38 & 250,3 &    &       &    &
\end{tabular}

\subsection{CTCSS Zoner i Sverige}

Rekommendationer för repeatrar i olika distrikt och län att använda CTCSS för att hindra att störningar uppkommer vid conds mm. Det ger också möjligheten för sändaramatörer att öppna just den repeater man önskar om man har flera på samma frekvens omkring sig.

\begin{longtable}{lcccc}
	\textbf{Område} & \textbf{Primär} & \textbf{Sek. 1} & \textbf{Sek. 2} & \textbf{Sek. 3} \\ \hline
	Distrikt 0      &      77,0       &      123.0      &      67.0       &      100.0      \\
	Distrikt 1      &      218.1      &      233.6      &                 &  \\
	Distrikt 2      &      107.2      &      146.2      &      162.2      &      186.2      \\
	Distrikt 3      &      127.3      &      141.3      &      250.3      &  \\
	Värml. / Örebro &      74.4       &      151.4      &                 &  \\
	Dalarna         &      85.4       &      151.4      &                 &  \\
	Distrikt 5      &      82.5       &      91.5       &      103.5      &      203.5      \\
	Distrikt 6      &      114.8      &      118,8      &      94.8       &      131.8      \\
	Distrikt 7      &      79.7       &      156.7      &      210.7      &
\end{longtable}




%%%%%%%%%% Bandplaner här roterade 90 för enklast läsning %%%%%%%%%%

\begin{landscape}
\section{Bandplan 2m 144--146 MHz}
\begin{tabular}{rrrll}
	\textbf{Frekvens} &           & \textbf{BW} & \textbf{Trafik} & \textbf{Noteringar}                     \\ \hline
	         144.0000 &  144.1100 &      500 Hz & CW, EME         & \textbf{CW anrop 144.050}               \\
	                  &           &             &                 & MS random 144.100                       \\ \hline
	         144.1100 &  144.1500 &      500 Hz & CW, MGM         & EME MGM 144.120--144.160                \\
	                  &           &             &                 & PSK31 cent. 144.138                     \\ \hline
	         144.1500 &  144.1800 &     2.7 kHz & CW, SSB, MGM    & EME 144.150--144.160                    \\
	                  &           &             &                 & MGM 144.160--144.180 anrop 144.170      \\ \hline
	         144.1800 &  144.3600 &     2.7 kHz & CW, SSB, MGM    & MS SSB random 144.195--144.205          \\
	                  &           &             &                 & \textbf{SSB anrop 144.300}              \\ \hline
	         144.3600 &  144.3990 &     2.7 kHz & CW, SSB, MGM    & MS MGM random anrop 144.370             \\ \hline
	         144.4000 &  144.4900 &      500 Hz & Fyr             & Exklusivt segment fyrar, ej QSO         \\ \hline
	         144.5000 &  144.7940 &      20 kHz & All mode        & SSTV, RTTY, FAX, ATV                    \\
	                  &           &             &                 & Linjära transpondrar                    \\ \hline
	         144.7940 &  144.9625 &      12 kHz & MGM             & APRS 144.800                            \\ \hline
	         144.9750 & 145.19350 &      12 kHz & FM, DV          & Rpt in 144.975--145.1935                \\
	                  &           &             &                 & RV46–-RV63, 12.5 kHz, 600 kHz skift     \\ \hline
	         145.1940 &  145.2060 &      12 kHz & FM rymd         & 145.200 för kom. m. bem. rymdfark.      \\ \hline
	         145.2060 &  145.5625 &      12 kHz & FM, DV          & FM 145.2125-–145.5875  V17–V47          \\
	                  &           &             &                 & \textbf{FM anrop 145.500}, RTTY 145.300 \\
	                  &           &             &                 & FM simpl. INET GW 145.2375, 2875, 3375  \\
	                  &           &             &                 & DV anrop 145.375                        \\ \hline
	         145.5750 &  145.7935 &      12 kHz & FM, DV          & Rpt ut 145.575--145.7875                \\
	                  &           &             &                 & RV46–RV63, 12.5 kHz kanalavstånd        \\ \hline
	          145.794 &   145.806 &      12 kHz & FM Rymd         & 145.800, 145.200 dplx m. bem. rymdfark. \\ \hline
	          145.806 &   146.000 &      12 kHz & All mode        & Exklusivt satellit                      \\ \hline
\end{tabular}
\end{landscape}

\begin{landscape}
\section{Bandplan 70cm 432--438 MHz}
\begin{tabular}{rrrll}
	\textbf{Frekvens} &               & \textbf{BW} & \textbf{Trafik} & \textbf{Anmärkning}                               \\ \hline
	         432.0000 &      432.0250 &      500 Hz & CW              & EME exklusivt.                                    \\ \hline
	         432.0250 &      432.1000 &      500 Hz & CW, PSK31       & CW mellan 432.000--085, \textbf{CW anrop 432.050} \\
	                  &               &             &                 & PSK31 432.088                                     \\ \hline
	         432.1000 &      432.3990 &     2.7 kHz & CW, SSB, MGM    & \textbf{SSB anrop 432.200}                        \\
	                  &               &             &                 & Mikrovåg talkback 432.350, FSK441 432.370         \\ \hline
	         432.4000 &      432.4900 &      500 Hz & Fyr             & Exklusivt segment för fyrar                       \\ \hline
	         432.5000 &      432.5940 &      12 kHz & All mode        & Linjära transpondrar IN 432.500--600              \\ \hline
	         432.5000 &      432.5750 &      12 kHz & All mode        & NRAU Digital rep. in 432.500--575 2 MHz skift     \\ \hline
	         432.5940 &      432.9940 &      12 kHz & All mode        & Linjära transpondrar ut 432.600--800              \\ \hline
	         432.5940 &      432.9940 &      12 kHz & FM              & Rep. in 432.600--975 RU368--398 2 MHz skift       \\ \hline
	         432.9940 &      433.3810 &      12 kHz & FM              & Rep. in 433.000--375 RU368--398 1.6 MHz skift     \\ \hline
	         433.3940 &      433.5810 &      12 kHz & FM              & SSTV (FM/AFSK) 433.400                            \\
	                  &               &             &                 & FM simplex U272--286 \textbf{anrop 433.500}       \\ \hline
	         433.6000 &      434.0000 &      20 kHz & All mode        & RTTY (FM/AFSK) 433.600                            \\
	                  &               &             &                 & FAX 433.700, APRS 433.800                         \\ \hline
	         434.0000 &      434.4940 &      20 kHz & All mode        & NRAU Dig. kanaler 433.450, 434.475                \\ \hline
	    $^1$ 434.5000 & $^1$ 434.5940 &      20 kHz & All mode        & NRAU Dig. rep. ut 434.500--575, 2 MHz skift       \\ \hline
	         434.5940 &      434.9810 &      12 kHz & FM              & NRAU Rep. ut 434.600--975 RU 368--RU398           \\
	                  &               &             &                 & 12,5 kHz med 2 MHz skift                          \\ \hline
	          435.000 &       438.000 &      20 kHz & All mode        & Exklusivt satellit                                \\ \hline
\end{tabular}

	\begin{itemize}
		\item[$^1$] SSA har ett fel i sin bandplan här där det står 433.500--433.5940 i stället.
	\end{itemize}
\end{landscape}
\end{document}

\begin{landscape}
\section{PTS Bandplan för VHF}
\end{landscape}
